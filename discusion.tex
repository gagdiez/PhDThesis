\chapter{Discusi\'on} 

A lo largo de este trabajo presentamos como parcelar la corteza cerebral
en base a un criterio estructural; estudiamos una de las m\'etodos
existentes actuales y presentamos un m\'etodo pr\'opio. En los capitulos
1 y 2 introdujimos la parcelaci\'on estructural y sus fundamentos 
te\'oricos. En las secciones \ref{sec:clustering_moreno} y 
\ref{sec:analisis_moreno} analizamos el reciente aporte de 
Moreno-Dominguez \cite{Moreno-Dominguez2014} al estado del arte. El mismo
se basa en combinar un algoritmo de tractograf\'ia y otro de 
\textit{clustering} para parcelar la corteza cerebral. Luego, bas\`andonos
en su trabajo, propusimos en la secci\'on \label{ch:nuestro} un nuevo
algoritmo de \textit{clustering}. En este \'ultimo capitulo discutiremos
los resultados obtenidos durante este trabajo. Comenzando por la robustes
del algoritmo de tractograf\'ia utilizado; siguiendo con una comparaci\'on
entre nuestro m\'etodo y el de Moreno-Dominguez; continuando con la 
validez biol\'ogica de la parcelaci\'on obtenida y propondremos objetos de
futuro estudio. \\

\section{Convergencia del algoritmo de tractograf\'ia}

Creamos los tractogramas usando el algoritmo \ref{alg:itract} de la 
secci\'on \ref{sec:convergencia}. Los tractogramas creados de esta manera
son inherentemente estoc\'asticos. Implementamos el algoritmo 
\ref{alg:bootstrap} de la secci\'on \ref{sec:convergencia} para estudiar
si la creaci\'on de los tractogramas era estable. Esto es, si dado un 
n\'umero suficientemente grande de part\'iculas el resultado converge
a un \'unico tractograma. \\

El algoritmo utilizado en el trabajo demostr\'o ser estable. Las figuras 
\ref{fig:s1}, \ref{fig:s2} y \ref{fig:s3} de la secci\'on 
\ref{sec:resultado_estabilidad} muestran que la desviaci\'on est\'andar es
casi nula al usar $15000$ part\'iculas. A su vez, la figura \ref{fig:mv}
muestra lo r\'apido que converge la media de tres voxels distintos. 
Como eran los de mayor desv\'iacion est\'andar, podemos asegurar que casi
no existe diferencia en la media de los tractogramas creados con dos mil
part\'iculas. \\

Nuestro m\'etodo de parcelamiento se basa en el clustering de 
tractogramas. Por ello es importante contar con un algoritmo de 
tractograf\'ia robusto. No encontramos estudios previos donde se analizara
este aspecto del algoritmo de tractograf\'ia usado. \\


\section{Comparaci\'on entra ambos m\'etodos de $clustering$}

En la secci\'on \ref{sec:clustering_moreno} presentamos el m\'etodo para
parcelar la corteza de Moreno-Dominguez. Luego, en la secci\'on
\label{ch:nuestro} presentamos nuestro m\'etodo. El resultado de parcelar
el \'area de Broca con estos se pueden ver en las secciones 
\ref{sec:moreno_broca} y \ref{sec:nuestro_broca}. En la secci\'on 
\label{sec:acercamiento} presentamos los resultados lado a lado en detalle
para compararlos. Los resultados son visualmente parecidos en todos los
casos. \\

Luego utilizamos ambos algoritmo para parcelar el hemisferio izquierdo del
mismo sujeto. Los resultados se encuentran en las secciones 
\label{sec:corteza_moreno} y \label{sec:corteza_nuestro}. En este caso 
resulta dif\'icil comparar visualmente las parcelaciones. M\'as a\'un, es
complicado encontrar cortes en los dendrogramas que generen
representaciones parecidas. Dado que los arboles fueron generados
de manera distinta, sus cortes no tienen por que corresponderse. Sin
embargo, la secci\'on \ref{sec:acercamiento_corteza} muestra algunas
similitudes en: la corteza motora y la somest\'esica; el l\'obulo frontal;
el l\'obulo occipital y el \'area de Broca. \\

Respecto a la eficiencia computacional, en la secci\'on 
\ref{sec:nuestro_clustering} comparamos ambos m\'etodos. 
Mostramos que nuestro m\'etodo permite mejorar la complejidad temporal y
espacial en al menos un orden de magnitud. Nuestro algoritmo de clustering
posee complejidad espacial $O(s^2)$ contra $O(s^2 + sm)$ del m\'etodo
Moreno-Dominguez. La complejidad temporal de nuestro algoritmo es $O(s^2m)$
contra $O(s^2)$ del m\'etodo Moreno-Dominguez. Recordemos de la secci\'on
\ref{sec:ralas} que dentro del contexto de este trabajo $m >> s$. En
general el n\'umero de semillas usadas es muy menor a la cantidad de
voxels de los tractogramas. En nuestro caso particular $s \cong 22000$ y
$m \cong 3500000$. $m$ es $159$ veces mas grande que $s$. \\


\section{Correlato con otras parcelaciones de la literatura}

Nuestro m\'etodo divide la corteza cerebral bas\'andonse en el 
agrupamiento de tractogramas. El criterio usado es puramente estructural.
Solo usamos informaci\'on sobre la estructura f\'sica de la materia
blanca. En la literatura actual existen parcelaciones basadas en otros
criterios. Desikan et al. \cite{Desikan2006} dividen la corteza usando
como referencia las circunvoluciones del cerebro. Peinfield 
\cite{Penfield1954} parcela la corteza en base a las respuestas a 
est\'imulos el\'ectricos sobre la misma. La parcelaci\'on de Desikan es
anat\'omica, mientras que la de Peinfield es funcional. Una pregunta
interesante es si nuestra parcelaci\'on posee correlaci\'on con otras
anat\'omicas o funcionales. La relaci\'on entre el aspecto anat\'omico o
funcional del cerebro y su aspecto estructural es un problema abierto en
neurociencia.  En el trabajo de Moreno-Dominguez solo hacen una breve
comparaci\'on entre sus resultados y un mapa citoarquitect\'onico. A 
continuaci\'on contrastaremos nuestros resultados con un atlas anat\'omico
y un estudio funcional realizado sobre el sujeto.  \\


\subsection{Correlaci\'on anat\'omica}

En la secci\'on \label{sec:corteza_nuestro} parcelamos el hemisferio 
izquiero de un sujeto con nuestro m\'etodo. Vamos a comparar nuestro
resultado con el atlas anat\'omico basado en Desikan et al. 
\cite{Desikan2006}
La figura \ref{fig:an2pa} muestra el resultado de proyectar algunas
regiones anat\'omicas sobre las obtenidas con nuestro m\'etodo. Nuestra
parcelaci\'on cre\'o \'areas bien delimitadas dentro de al menos 9
regiones anat\'omicas. Para explicar esto con mayor detalle incluimos la
figura \ref{fig:an2pa2}. All\'i solo dejamos pintadas las regiones que 
estaban en su contenidas mayor\'ia dentro de alguna proyecci\'on. Podemos
ver que los bordes entre las \'areas pintadas y las proyectadas son
similares. El $90\%$ del \'area de las 9 proyecciones est\'a pintada y 
solo un $5\%$ del \'area de nuestras parcelas queda fuera de las
proyecciones. Las parcelas de nuestro m\'etodo que se encuentran en una
sola proyecci\'on representan el $88\%$ del \'area pintada. Las parcelas
que est\'an en dos regiones representan el $12\%$ restante. Esto indica que casi todas nuestras parcelas se encuentran contenidas dentro de a lo
sumo una proyecci\'on. \\

Repetimos este procedimiento de proyectar el atlas anat\'omico sobre la
parcelaci\'on obtenida en la figura \ref{fig:moreno_corteza2}. Dicha
parcelaci\'on fue obtenida mediante el m\'etodo de Moreno-Dominguez.
En este caso el $75\%$ del \'area de las 9 proyecciones est\'a pintada y 
un $15\%$ del \'area de las parcelas quedan fuera de las proyecciones. 
Sin embargo esto no quiere decir que el m\'etodo de Moreno-Dominguez 
posea menor correlaci\'on con el atlas anat\'omico. La parcelaci\'on 
resultante en ambos m\'etodos depende de la altura a la cual se decida 
cortar el dendrograma. Cortando el mismo dendrograma a mayor profundidad
se pueden conseguir resultados similares a los de nuestro m\'etodo.\\


\begin{figure}[h!]
    \includegraphics[width=\textwidth]{img/anatomica2parcelation.png}
    \caption{Proyecci\'on del atlas anat\'omico (izquierda) a la 
             parcelaci\'on obtenida (derecha). Regiones: 
             1. \textit{pars triangularis}; 2. \textit{pars opercularis};
             3. \textit{caudal middle frontal}; 
             4. \textit{precentral (\'area motora)}; 
             5. \textit{post central}; 6. \textit{superior parietal}; 
             7. \textit{lateral occipital}; 8. \textit{fusiform};
             9. \textit{superior temporal}.  }
    \label{fig:an2pa}
\end{figure}

\begin{figure}[h!]
    \includegraphics[width=\textwidth]{img/anatomica2parcelation2.png}
    \caption{Proyecci\'on del atlas anat\'omico (izquierda) a la 
             parcelaci\'on obtenida (derecha). Solo las parcelas contenidas
             en mayor proporci\'on fueron pintadas. Regiones: 
             1. \textit{pars triangularis}; 2. \textit{pars opercularis};
             3. \textit{caudal middle frontal}; 
             4. \textit{precentral (\'area motora)}; 
             5. \textit{post central}; 6. \textit{superior parietal}; 
             7. \textit{lateral occipital}; 8. \textit{fusiform};
             9. \textit{superior temporal}.  }
    \label{fig:an2pa2}
\end{figure}

\subsection{Correlaci\'on funcional}

Para analizar la correlaci\'on funcional utilizamos datos provenientes
del estudio de Bach et al. \cite{Barch2013} hecho sobre el sujeto. En el
mismo utilizan Resonancia Magn\'etica Funcional (fMRI) para estudiar la
respuesta sobre la corteza a ciertos est\'imulos particulares. La 
Resonancia Magn\'etica Funcional es un tipo especial de resonancia 
magn\'etica donde se mide el nivel de oxigeno en sangre. Nosotros nos
enfocamos en las respuestas a los siguientes est\'imulos: mover la mano;
mover el pie; mover la lengua; reconocer formas y caras; comprender un
cuento y resolver problemas matem\'aticos simples. Por cada uno de estos
est\'imulos se cuenta con un mapa de \textit{z-scores} sobre la corteza.
Estos \textit{z-scores} representan la correlaci\'on entre la activaci\'on
de cada punto de la corteza con el est\'imulo. La figura \ref{fig:32k}
muestra nuestra parcelaci\'on interpolada sobre una superficie de menor
resoluci\'on que la usada hasta aqu\'i. Algunas regiones de inter\'es
fueron etiquetadas. La figura \ref{fig:32k_z5} muestra la proyecci\'on de
estas regiones sobre los \textit{z-scores} $ \geq 5$ de los distintos
mapas. La figura \ref{fig:32k_z7} muestra la proyecci\'on sobre los
\textit{z-scores} $ \geq 7$.
 
\begin{figure}[h!]
    \includegraphics[width=\textwidth]{img/32k_labels.png}
    \caption{Parcelas en baja resoluci\'on}
    \label{fig:32k}
\end{figure}


\begin{figure}[h!]
    \includegraphics[width=\textwidth]{img/32k_z5.png}
    \caption{Projecci\'on de las parcelas (izquierda) sobre los \textit{z-scores}
    mayores a $5$ de las activaciones funcionales (derecha)}
    \label{fig:32k_z5}
\end{figure}

\begin{figure}[h!]
    \includegraphics[width=\textwidth]{img/32k_z7.png}
    \caption{Projecci\'on de las parcelas (izquierda) sobre los \textit{z-scores}
    mayores a $7$ de las activaciones funcionales (derecha)}
    \label{fig:32k_z7}
\end{figure}


Tomamos las etiquetas de la figura \ref{fig:32k}. Los resultados muestran que 
la regi\'on $1$ es funcionalmente consistente con la resoluci\'on de problemas
matem\'aticos. Las regiones $2-10$ son consistentes anat\'omicamente con la corteza motora. A su vez, tambi\'en son consistentes con el hom\'unculo cortical.

El hom\'unculo cortical es una 
divisi\'on funcional de la corteza motora y de la corteza somastest\'esica.
Mapea distintas regiones del cuerpo con regiones de la corteza como puede
verse en la figura \ref{fig:homunculo}

\begin{figure}[h!]
    \includegraphics[width=0.8\textwidth]{img/homunculus.jpg}
    \caption{Hom\'unculo Cortical}
    \label{fig:homunculo}
\end{figure}


  La regiones $11$ y $12$ son
consistentes anat\'omica y funcionalmente con el l\'obulo temporal superior.
Finalmente, las regiones $13$ y $14$ son consistentes anat\'omica y funcionalmente
con el l\'obulo occipital.



\begin{table}[]
\centering
\label{tb:zscore5}
\begin{tabular}{|l|l|l|l|l|l|l|}
\hline
   & pie   & mano  & lengua & matemática & cuento & formas \\ \hline
1  & 0     & 0     & 0      & 0.276      & 0      & 0      \\ \hline
2  & 0     & 0     & 0.311  & 0          & 0      & 0      \\ \hline
3  & 0     & 0     & 0      & 0          & 0      & 0      \\ \hline
4  & 0     & 0.172 & 0      & 0          & 0      & 0      \\ \hline
5  & 0.177 & 0.321 & 0      & 0          & 0      & 0      \\ \hline
6  & 0.201 & 0     & 0      & 0          & 0      & 0      \\ \hline
7  & 0.231 & 0.044 & 0      & 0          & 0      & 0      \\ \hline
8  & 0     & {\bf 0.590} & 0.139  & 0          & 0      & 0      \\ \hline
9  & 0     & 0     & {\bf 0.610}   & 0          & 0      & 0      \\ \hline
10 & 0.33  & 0     & 0      & 0          & 0      & 0      \\ \hline
11 & 0     & 0     & 0      & 0          & {\bf 0.606}  & 0      \\ \hline
12 & 0     & 0     & 0      & 0          & {\bf 0.454}  & 0      \\ \hline
13 & 0     & 0     & 0      & 0          & 0      & {\bf 0.491}  \\ \hline
14 & 0     & 0     & 0      & 0          & 0      & 0.154  \\ \hline
\end{tabular}

\caption{Relaci\'on entre las \'areas de cada parcela y los mapas de
         activaci\'on con $zscore > 5$. Los valores mayores a $0.4$ fueron
         resaltados.}
\end{table}


\begin{table}[]
\centering


\label{tb:zscore7}
\begin{tabular}{|l|l|l|l|l|l|l|}
\hline
   & pie   & mano  & lengua & matemática & cuento & formas \\ \hline
1  & 0     & 0     & 0      & {\bf 0.419} & 0     & 0      \\ \hline
2  & 0     & 0     & 0.323  & 0          & 0      & 0      \\ \hline
3  & 0     & 0     & 0      & 0          & 0      & 0      \\ \hline
4  & 0     & 0.115 & 0      & 0          & 0      & 0      \\ \hline
5  & 0.141 & 0.341 & 0      & 0          & 0      & 0      \\ \hline
6  & 0.095 & 0     & 0      & 0          & 0      & 0      \\ \hline
7  & 0.377 & 0.040 & 0      & 0          & 0      & 0      \\ \hline
8  & 0     & {\bf 0.669} & 0.147  & 0          & 0      & 0      \\ \hline
9  & 0     & 0     & {\bf 0.629}  & 0          & 0      & 0      \\ \hline
10 & 0.240 & 0     & 0      & 0          & 0      & 0      \\ \hline
11 & 0     & 0     & 0      & 0          & {\bf 0.715}  & 0      \\ \hline
12 & 0     & 0     & 0      & 0          & {\bf 0.458}  & 0      \\ \hline
13 & 0     & 0     & 0      & 0          & 0      & {\bf 0.428}  \\ \hline
14 & 0     & 0     & 0      & 0          & 0      & 0.154 \\ \hline
\end{tabular}

\caption{Relaci\'on entre las \'areas de cada parcela y los mapas de
         activaci\'on con $zscore > 7$. Los valores mayores a $0.4$ fueron
         resaltados.}

\end{table}


\section{Trabajo Futuro}
El m\'etodo aqu\'i propuesto ha mostrado resultados con una fuerte correlaci\'on
anat\'omica y funcional sobre el sujeto estudiado. Dado este buen resultado es 
importante estudiar la estabilidad del mismo sobre una poblaci\'on mayor.
A su vez, el transformar los tractogramas a un espacio eucl\'ideo brinda la 
capacidad de calcular tractogramas medios. Ser\'ia interesante el estudiar
si es posible obtener una parcelaci\'on media dado un grupo de sujetos. 
En su trabajo \cite{Moreno-Dominguez2014}, Moreno-Dominguez propone una manera
de comparar dendrogramas para buscar diferencias entre sujetos. Contar con un
dendrograma medio podr\'ia llevar a una nueva manera de detectar patolog\'ias
en individuos.

