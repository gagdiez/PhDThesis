\documentclass[a4paper, 10pt]{article}
\parindent0pt  \parskip5pt             % make block paragraphs
\usepackage[utf8]{inputenc}
\usepackage[english]{babel}

\begin{document}

\begin{center}
{\Large \bf {\centering Title to be defined}} \\
{\vspace{1mm} \bf Gallardo Guillermo}
\end{center}

Composed by billions of interconnected neurons, the brain is the most complex
biological machine that we know. Understanding how brain connectivity is 
organized, and how this constrains brain functionality is one a key question
of neuroscience. Recent advances in acquisition and modeling techniques on
Diffusion Magnetic Resonance Imaging (dMRI) have facilitated to estimate and
study axonal connectivity in vivo. In this thesis, we leverage recent advances
in dMRI and tractography algorithms in order to study: how brain's connectivity
is organized; how connectivity is related to anatomy and function; how to
find correspondences when the connectivity variates across people, and how to
infer connectivity in the presence of a brain's pathology. The thesis is
divided in three major contributions:

Our first contribution is a parsimonious model for the long-range axonal
connectivity (extrinsic connectivity), and an efficient technique to divide the
brain in regions with homogeneous connectivity~\cite{Gallardo2017a}. Our
connectivity model is based on histological results obtained in the macaque
brain, and accounts for the across-subject variability in the human brain
connectivity. Our parceling technique uses a hierarchical clustering approach,
letting us comprise multiple granularities of the same brain parcellation.
Also, our technique can create both single subject and groupwise parcellations
of the whole cortex, allowing to study brain connectivity at the single subject
and at the population level. While our technique is solely based on the brain's
structural connectivity, our resulting parcels are in agreement with anatomical,
structural and functional parcellations extant in the literature. Our technique
helps to lower the gap between structural connectivity and brain function, since
some of our pure structural parcels show good overlapping with responses to
functional tasks.

Our second contribution is a technique to find correspondence between
structural parcellations of different subjects~\cite{Gallardo2018}. Even when
produced by the same technique, parcellations tend to differ in the
number, shape, and spatial localization of parcels across subjects. Matching
these parcels across subjects is an open problem in neuroscience. To solve
it, we propose a parcel matching method based on Optimal Transport. We test its
performance on different parcellations, and compare it against state of the
art matching techniques. We show that our method achieves the highest number
of correct matches. Our technique could help to study properties of structurally
defined areas, when they do not have high spatial coherence across subjects.
Also, it could help to understand the link between different brain atlases, and
improve the comparisons of cortical areas between higher primates.

Our third contribution is a multi-atlas technique to infer the location of 
white-matter bundles in patients with a brain pathology~\cite{Guillermo2018}.
Lesions in the cortex or white matter disrupt the normal functioning of the
brain. Some white matter pathologies, such as tumors or traumatic brain injury,
hampers tractography, difficulting to infer which pathways are affected. We 
present a technique that infers the affected tracts by aggregating spatial
information from healthy subjects while taking into account the diffusion
information of the patient. In particular, we register the tracts of each
healthy subject to the patient, and make each healthy patient 'vote' for a
tract on each voxel of the DWI image of the patient. Our technique weights the
vote of each subject based on how the voted pathway is supported by the patient's 
diffusion data. This is, if the diffusion data of our patient is consistent with
the direction of the voted pathway, the vote has a higher weight. We show that
our technique achieves better results that using the simple voting.

\bibliographystyle{ieeetr}
\bibliography{bibliography}
\end{document}
