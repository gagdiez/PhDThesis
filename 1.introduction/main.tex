\chapter{Introduction}

Composed of billions of interconnected neurons, the brain is a highly complex
biological machine. Understanding how the brain connectivity is organized,
and how cognition emerges from it, is a key problem in neuroscience. Given
current technological limitations, it is not feasible to study the brain at
cellular level. Instead, the brain is parcelled, and the bast network or neurons
is abstracted as a network of interacting parcels. Different brain parcellations
exist based on criteria such as anatomy, cytoarchitecture or functional
specialization~\cite{Brodmann1909, Collins1998, Yeo2011}. Studying the
agreement between existent parcellations is a hot topic in neuroscience.
So far, accumulating evidence suggest a strong relationship between brain
connectivity, cytoarchitecture, and brain function~\citep{Passingham2002, Johansen-Berg2004, Honey2009, Eickhoff2010}
Therefore, understanding how axonal connectivity is arranged can provide key
information in unraveling the mystery of concious life.

Cerebral dissections~\cite{Meynert1872, Brodmann1909, Gray1918},
and the injection of chemical markers~\cite{Schmahmann2006, Stephan2013} used to
be the only way to map brain axonal connectivity. Such methodologies are highly
invasive and can only be use in post-mortem studies. Nowadays, the advent of
Diffusion Magnetic Resonance Imaging (dMRI) allows to non-invasively quantify
the diffusion of water molecules in live tissue as the white-matter. Under
normal unhindered conditions, water particles diffuse randomly in a process
known as Brownian motion. However, in the white matter, water molecules are
constrained, and can only diffuse inside and along axonal bundles. Due to this
phenomenon, dMRI is able to capture information about the structural organization
of the brain. Building on top of this, tractography algorithms reconstruct
white matter bundles in the brain, allowing to estimate the brain's axonal
connectivity.


In this thesis, we leverage recent advances in dMRI and tractograpgt algorithms
in order to: study how the brain connectivity is organized; study the relationship
between brain connectivity, anatomy, and function; find correspondences between 
structurally-defined regions of different subjects, and infer connectivity in
the presence of a brain’s pathology. We present three major contributions.

Our first contribution is a parsimonious model for the long-range axonal
connectivity (extrinsic connectivity), and an efficient technique to divide the
brain in regions with homogeneous connectivity~\cite{Gallardo2017a}. Our
connectivity model is based on histological results obtained in the macaque
brain, and accounts for the across-subject variability in the human brain
connectivity. Our parceling technique uses a hierarchical clustering approach,
letting us comprise multiple granularities of the same brain parcellation.
Also, our technique can create both single subject and groupwise parcellations
of the whole cortex, allowing to study brain connectivity at the single subject
and at the population level. While our technique is solely based on the brain's
structural connectivity, our resulting parcels are in agreement with anatomical,
structural and functional parcellations extant in the literature. Our technique
helps to lower the gap between structural connectivity and brain function, since
some of our pure structural parcels show good overlapping with responses to
functional tasks.

Our second contribution is a technique to find correspondence between
structural parcellations of different subjects~\cite{Gallardo2018}. Even when
produced by the same technique, parcellations tend to differ in the
number, shape, and spatial localization of parcels across subjects. Matching
these parcels across subjects is an open problem in neuroscience. To solve
it, we propose a parcel matching method based on Optimal Transport. We test its
performance on different parcellations, and compare it against state of the
art matching techniques. We show that our method achieves the highest number
of correct matches. Our technique could help to study properties of structurally
defined areas, when they do not have high spatial coherence across subjects.
Also, it could help to understand the link between different brain atlases, and
improve the comparisons of cortical areas between higher primates.

Our third contribution is a multi-atlas technique to infer the location of 
white-matter bundles in patients with a brain pathology~\cite{Guillermo2018}.
Lesions in the cortex or white matter disrupt the normal functioning of the
brain. Some white matter pathologies, such as tumors or traumatic brain injury,
hampers tractography, difficulting to infer which pathways are affected. We 
present a technique that infers the affected tracts by aggregating spatial
information from healthy subjects while taking into account the diffusion
information of the patient. In particular, we register the tracts of each
healthy subject to the patient, and make each healthy patient 'vote' for a
tract on each voxel of the DWI image of the patient. Our technique weights the
vote of each subject based on how the voted pathway is supported by the patient's 
diffusion data. This is, if the diffusion data of our patient is consistent with
the direction of the voted pathway, the vote has a higher weight. We show that
our technique achieves better results that using the simple voting.

\section*{Organization of this Thesis}

This thesis is divided in two parts: Background and Contributions. In the
Background chapters we give a brief introduction to neuroanatomy, non-invasive
imaging techniques, and brain parcellation. In the Contributions chapters we
introduce our parceling technique, our matching technique and our multi-atlas
technique. We now present the outline of each chapter.

\subsection{Part I: Background}

\subsubsection{Chapter \ref{ch:intro_anato}: Introduction to Brain Cytoarchitecture, Neuroanatomy and Brain Function.}
In this chapter we cover the basic aspects of cellular composition, morphology
and function of the human brain. We start with a brief introduction to the
human nervous system, in order to understand the biological context of the brain.
Then, we study the brain from both a microscopic and a macroscopic view. In the
microscopic view, we explain the cellular composition of the brain and how it’s
organized. In the macroscopic view, we zoom out and make a review of the most 
important divisions and anatomical landmarks of the brain. Finally, we describe
the functional role of some of these gross anatomical divisions.

\subsubsection{Chapter \ref{ch:intro_mri}: Introduction to Non-invasive Imaging Techniques.}
In this chapter, we start by introducing some concepts in
nuclear physics and explain how they are applied in MRI to study the human
brain. Then, we explain how modifying the acquisition sequences allows to
study the physical process of diffusion, enabling to estimate the location
of tracts in the white-matter. Finally, we make a brief introduction to how to
detect brain activation in response to functional or cognitive tasks in the
brain using Functional MRI.

\subsubsection{Chapter \ref{ch:brain_mapping}: Mapping the Brain: A review of brain parcellations.}
Neuroscientist have long thought of the brain as a mosaic of interconnected
regions. This simplification allows to abstract the complex neuronal network
that underlies cognition. Until today, it's not clear that an unique and true
division of the human brain exists. Different modalities have been used to
study the brain, deriving in different parcellations. In this chapter, we make
a review of the different criteria to divide the brain; the hypothesis in
which they are based, and present the most notable parcellations of each.

\subsection{Part II: Contributions}

\subsubsection{Chapter \ref{ch:structural_parcellation}: Groupwise Structural Parcellation of the Whole Cortex: A Logistic Random Effects Model Based Approach}
In this chapter, we propose a parsimonious model for the extrinsic connectivity
and an efficient parceling technique based on clustering of tractograms. 
Our technique allows the creation of single subject and groupwise parcellations
of the whole cortex. We show that our technique creates parcellations in
agreement with anatomical, structural and functional parcellations extant in 
the literature.

\subsubsection{Chapter \ref{ch:matching}: Solving the Cross-Subject Parcel Matching Problem using Optimal Transport.}
In this chapter, we propose a parcel matching method based on Optimal Transport.
We test its performance by matching parcels of the Desikan atlas, parcels based
on a functional criteria and structural parcels. We compare our technique against
three other ways to match parcels which are based on the Euclidean distance, the
cosine similarity, and the Kullback-Leibler divergence. Our results show that
our method achieves the highest number of correct matches.

\subsubsection{Chapter \ref{ch:multiatlas}: Inferring the Localization of White-Matter Tracts using Diffusion Driven Label Fusion}
In this chapter, we introduce a way to infer the location of pathways, even
when it is not possible to use tractography to locate them. Our technique is
based on a methodology named label fusion. In particular, we show how to add
dMRI information to the label fusion in order to better estimate the location
of white matter pathways.


%I start by making a beautiful introduction to neuroanatomy, because the reader needs to know what is a brain.
%Here we talk about sulci, giri, white matter, gray matter, cellular composition, layers, etc.
%This chapter also talks about brain function, but from an old perspective.
%Then, I move to explain the state of the art of the non-invasive techniques that we're interested in.
%Since the thesis is going to be about brain parcellation, I have to explain why we are interested on it.
%So the next chapter is about well known parcellations: cytho, broadmann areas, desikan, functional... the classical ones.
%Then, I can basically take the paper of Vinod and explain that, even when we subdivide the brain in specific areas, the brain works as a network.
%There should be a big focus on Structural connectivity since it's the main theme of the thesis.
%We start again with why structural connectivity is important.
%Then, we explain again that it's important to divide the brain based on its connectivity, so we know the basic pieces of the brain that work together.
%We present my work (CDMRI + neuroimage), this includes state of the art, and I guess stuff that was made after.
%%In the previous chapter we explained how to divide the brain based on a structural criteria.
%Now, other chapter, we discuss that there's variability in the parcellation of different subjects.
%We insist in the fact that this is ok, because we are all different humans. 
%Then we present the MICCAI paper made in collaboration with Nathalie, this includes state of the art, etc etc etc
%%
%Another chapter, this time about multiatlas.
%%\section{Structure vs Function}
%Finally, something about structure vs function.
%In the intro we explained that structure is important, and that function is driven by function.
%It's time to show that this actually happens, and in which cases.
%We can start talking about Osher and others, however this didn't work for us.
%I think it's important to say it.
%%Gaston????????????
%Then, I can include the final part of my neuroimage, where we show the parcels are functionally specialized.
%Maybe something about the work with Nathalie.
%Stanford stuff goes here I guess
%New experiments.
%
%\section{Other projects?}
%Stanford vwfa? RTOP? Harvard? No idea...



%\section{Organization}
%
%\subsection{Neuroanatomy}
%
%I start by making a beautiful introduction to neuroanatomy, because the reader needs to know what is a brain.
%Here we talk about sulci, giri, white matter, gray matter, celular composition, layers...
%
%\subsubsection{Brain parcellation - Cytho}
%Since the thesis is going to be about brain parcellation, I have to explain why we are interested on it.
%I can explain about cytho, broadmann areas, desikan, functional... the classical ones
%
%\subsubsection{The brain as a network}
%Here I can basically take the paper of Vinod and explain that, even when we subdivide the brain in specific areas, the brain works as a network.
%There should be a big focus on Structural connectivity since it's the main theme of the thesis.
%
%\subsection{Non invasive methods}
%In this chapter we explain the state of the art of the non-invasive techniques that we're interested in.
% 
%\subsubsection{dMRI and tractography}
%How diffusion MRI works, from A to Z.
%Then I talk about tractography... when to use each type, advantages and disadvantages.
%
%\subsubsection{fMRI}
%A gentle introduction to fMRI, we don't need to focus a lot on this.
%
%\subsection{Background Methods}
%
%dividir en backgound y contributions
%
%\section{Structural Connectivity based parcellation}
%We start again with why structural connectivity is important.
%Then, we explain again that it's important to divide the brain based on its connectivity, so we know the basic pieces of the brain that work together.
%We present my work (CDMRI + neuroimage), this includes state of the art, and I guess stuff that was made after?
%
%\section{Matching parcels across different subjects}
%In the previous chapter we explained how to divide the brain based on a structural criteria.
%Now, we discuss that there's variability in the parcellation of different subjects.
%We insist in the fact that this is ok, because we are all different humans. 
%Then we present the MICCAI paper made in collaboration with Nathalie, this includes state of the art, etc etc etc
%
%\section{Structure vs Function}
%In the intro we explained that structure is important, and that function is driven by function.
%It's time to show that this actually happens, and in which cases.
%We can start talking about Osher and others, however this didn't work for us.
%I think it's important to say it.
%Gaston????????????
%Then, I can include the final part of my neuroimage, where we show the parcels are functionally specialized.
%Maybe something about the work with Nathalie.
%New experiments.
%
%\section{Other projects?}
%Stanford vwfa? RTOP? Harvard? No idea...
%
%\section{Conclusions}
\chapterbib
