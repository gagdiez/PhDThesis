\chapter{Introduction}

The human brain is the most complex biological machine known in the universe.
The brain works as a complex network.
Depending how you look at it, the brain can be divided following many different criteria (cytho, function).
In each one of these criteria, brain connectivity is super important and consistent.
We know this thanks to invasive stuff.
Since our brain is the result of biological evolution, we tortured monkeys, cats and rats in order to study and understand it.
We cannot do that anymore, so it's difficult to study brain connectivity.
Diffusion MRI is cool because is invivo and nobody suffers.
But it has strong limitations.
It has been used to divide the brain, but the techniques are not good enough.
We don't know how many parcels we need, or how to do it.
We propose one way to solve it.
We show it is super consistent with function, and this is amazing.
Now we would like to use this to study brain connectivity and function in different subjects.
Turns out that it's hard to match parcels across subjects.
Spatial overlap does not work here because of [cite].
We need to match stuff while imposing few constraints, specially spatial ones.
The existing techniques are nice and simple.
We propose to improve the matching by using optimal transport.
It actually works well.
Now that we have a way to parcellate the brain, and a way to map that across subjects, maybe we can better study the structure-function relationship.
I would like to talk about predicting function from structure.
This would be a negative chapter, since we were not able to predict anything from tractography.
But, what happens when there's a pathology in the white matter?
It's affecting brain function, but we cannot precisely predict what since it impedes tractography.
In that case, we need to do something to infear which bundles are affected.
We can do multi-atlas stuff.
Since bundles are related to dmri, we can add dMRI information to the multi-atlas to improve the localization of afected bundles.
Finally, the structure vs function in the brain does not necesarilly always have to be structural connectivity vs tfmri function.
We can also use microstructure vs cognitive.
RTOP -> cognitive

\section{Organization}
I start by making a beautiful introduction to neuroanatomy, because the reader needs to know what is a brain.
Here we talk about sulci, giri, white matter, gray matter, celular composition, layers, etc.
This chapter also talks about brain function, but from an old perspective.
Then, I move to explain the state of the art of the non-invasive techniques that we're interested in.
Since the thesis is going to be about brain parcellation, I have to explain why we are interested on it.
So the next chapter is about well known parcellations: cytho, broadmann areas, desikan, functional... the classical ones.
Then, I can basically take the paper of Vinod and explain that, even when we subdivide the brain in specific areas, the brain works as a network.
There should be a big focus on Structural connectivity since it's the main theme of the thesis.
We start again with why structural connectivity is important.
Then, we explain again that it's important to divide the brain based on its connectivity, so we know the basic pieces of the brain that work together.
We present my work (CDMRI + neuroimage), this includes state of the art, and I guess stuff that was made after.
%In the previous chapter we explained how to divide the brain based on a structural criteria.
Now, other chapter, we discuss that there's variability in the parcellation of different subjects.
We insist in the fact that this is ok, because we are all different humans. 
Then we present the MICCAI paper made in collaboration with Nathalie, this includes state of the art, etc etc etc
%
Another chapter, this time about multiatlas.
%\section{Structure vs Function}
Finally, something about structure vs function.
In the intro we explained that structure is important, and that function is driven by function.
It's time to show that this actually happens, and in which cases.
We can start talking about Osher and others, however this didn't work for us.
I think it's important to say it.
%Gaston????????????
Then, I can include the final part of my neuroimage, where we show the parcels are functionally specialized.
Maybe something about the work with Nathalie.
Stanford stuff goes here I guess
%New experiments.
%
%\section{Other projects?}
%Stanford vwfa? RTOP? Harvard? No idea...



%\section{Organization}
%
%\subsection{Neuroanatomy}
%
%I start by making a beautiful introduction to neuroanatomy, because the reader needs to know what is a brain.
%Here we talk about sulci, giri, white matter, gray matter, celular composition, layers...
%
%\subsubsection{Brain parcellation - Cytho}
%Since the thesis is going to be about brain parcellation, I have to explain why we are interested on it.
%I can explain about cytho, broadmann areas, desikan, functional... the classical ones
%
%\subsubsection{The brain as a network}
%Here I can basically take the paper of Vinod and explain that, even when we subdivide the brain in specific areas, the brain works as a network.
%There should be a big focus on Structural connectivity since it's the main theme of the thesis.
%
%\subsection{Non invasive methods}
%In this chapter we explain the state of the art of the non-invasive techniques that we're interested in.
% 
%\subsubsection{dMRI and tractography}
%How diffusion MRI works, from A to Z.
%Then I talk about tractography... when to use each type, advantages and disadvantages.
%
%\subsubsection{fMRI}
%A gentle introduction to fMRI, we don't need to focus a lot on this.
%
%\subsection{Background Methods}
%
%dividir en backgound y contributions
%
%\section{Structural Connectivity based parcellation}
%We start again with why structural connectivity is important.
%Then, we explain again that it's important to divide the brain based on its connectivity, so we know the basic pieces of the brain that work together.
%We present my work (CDMRI + neuroimage), this includes state of the art, and I guess stuff that was made after?
%
%\section{Matching parcels across different subjects}
%In the previous chapter we explained how to divide the brain based on a structural criteria.
%Now, we discuss that there's variability in the parcellation of different subjects.
%We insist in the fact that this is ok, because we are all different humans. 
%Then we present the MICCAI paper made in collaboration with Nathalie, this includes state of the art, etc etc etc
%
%\section{Structure vs Function}
%In the intro we explained that structure is important, and that function is driven by function.
%It's time to show that this actually happens, and in which cases.
%We can start talking about Osher and others, however this didn't work for us.
%I think it's important to say it.
%Gaston????????????
%Then, I can include the final part of my neuroimage, where we show the parcels are functionally specialized.
%Maybe something about the work with Nathalie.
%New experiments.
%
%\section{Other projects?}
%Stanford vwfa? RTOP? Harvard? No idea...
%
%\section{Conclusions}
