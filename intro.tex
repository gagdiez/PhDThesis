\chapter{Introducci\'on}

En la neurociencia moderna existe el consentimiento general de que el cerebro
puede ser dividido en \'areas de acuerdo a distintos criterios estructurales,
siendo la citoarquitectura de Broadmann \cite{Brodmann1909} una de la m\'as conocidas.
Actualmente existe evidencia de que es posible atribuir un criterio funcional
a cada una de estas regiones, como mover la mano o procesar el lenguaje
\cite{Greicius2003}. No obstante, el c\'omo se relacionan dichas \'areas con acciones
complejas sigue siendo un problema abierto \cite{Barch2013}. As\'i mismo, tampoco
se conoce si esta divisi\'on es \'unica, y en tal caso, qu\'e l\'imites definen
a cada \'area. Por ello existe la necesidad de entender c\'omo es que el cerebro
est\'a conectado estructuralmente, esto es, entender qu\'e regiones est\'an 
conectadas f\'isicamente por axones y cu\'anto influye esto en el aspecto funcional
del cerebro. La relaci\'on funci\'on-estructura posee diversas aplicaciones en la
medicina, como por ejemplo el planeamiento quir\'urgico \cite{Stufflebeam2011}
\cite{Oishi2010}; asistencia durante la cirug\'ia \cite{Bello2008} y 
rehabilitaci\'on \cite{Song2014} de pacientes. \\

El reciente desarrollo de la Resonancia Magn\'etica de Difusi\'on (dMRI) ha
aportado nuevas t\'ecnicas para estudiar la conectividad estructural.
En particular, el conocer la intensidad de difusi\'on que existe en cada punto del 
cerebro permite caracterizar los axones dentro de la materia blanca. Una forma 
de hacerlo es utilizando algoritmos de tractograf\'ia \cite{Descoteaux2009}
\cite{Jbabdi2007}, estos toman un punto del cerebro como semilla y devuelven un
tractograma. Un tractograma es una imagen donde cada voxel representa la
probabilidad de que ese punto del cerebro est\'e conectado a la semilla elegida.
Las probabilidades se estiman mediante un procedimiento Monte Carlo, simulando el
recorrido de un n\'umero de part\'iculas de agua por la materia blanca comenzando
desde dicha semilla. Recordemos que la corteza est\'a formada
por materia gris, la cual es densa en neuronas, mientras que la materia blanca
est\'a compuesta por axones que las conectan entre si. Como cada neurona posee
asociado un ax\'on, colocar semillas en la interfaz entre la materia gris y la
blanca permite caracterizar las regiones de la corteza \cite{Mori2002}
\cite{Anwander2006}. Distintos grupos han propuesto el agrupar estas semillas
empleando un algoritmo de clustering para definir nuevos criterios de parcelamiento. \\

\vspace{0.3cm}

Los algoritmos de clustering son una herramienta muy conocida en los
campos de \textit{Machine Learning} y \textit{Data Mining} ya que permiten
agrupar objetos en base a alg\'un criterio de similitud. Ejemplos de ellos son: 

\begin{itemize}
    \item K-means: Divide $n$ vectores en $k$ clusters diferentes, siendo $k$
                   un n\'umero predefinido. Cada cluster est\'a formado 
                   por los elementos que m\'as cerca est\'an al vector medio del
                   mismo. \cite{Hartigan1979}
    \item Agglomerative Hierarchical Clustering: Cada observaci\'on comienza en 
                   un cluster distinto. En cada paso une dos clusters siguiendo
                   alg\'un criterio de similitud y crea un elemento 
                   que representa al nuevo cluster. La jerarqu\'ia resultante 
                   es expresada como un dendrograma. \cite{Mining2009}
    \item Gaussian Mixture: Asume que todas las observaciones provienen de un 
                   n\'umero finito de distribuciones Gaussianas con par\'ametros
                   desconocidos. Implementa \textit{expectation-maximization} para
                   determinar dichos par\'ametros iterativamente. \cite{Mining2009}
\end{itemize}

Cada algoritmo utiliza distintos modelos de cluster, por lo que el espacio de
aplicaci\'on y el resultado de cada uno var\'ia significativamente. \\

Al querer utilizar el clustering de semillas para parcelar la corteza cerebral
se deben enfrentar una serie de problemas, ya que todos los pasos involucrados
son computacionalmente caros en t\'erminos espaciales y temporales. El primero
es c\'omo seleccionar y posicionar correctamente las semillas a utilizar. Es 
necesario contar con alguna manera de discriminar los voxels que solo pertenecen
a la corteza, a su vez, en base a qu\'e tan profundo se desea que est\'en las semillas 
en la materia blanca, ser\'a necesario implementar un algoritmo que elija las 
posiciones teniendo en cuenta la morfolog\'ia del cerebro. Al momento de generar
los tractogramas la performance es un factor importante a tener en cuenta. En la
literatura actual se utilizan hasta cien mil part\'iculas \cite{Moreno-Dominguez2014}
para generar cada tractograma, si se tiene en cuenta que en general hay mas de
cincuenta mil semillas (ver Secci\'on) entonces es claramente necesario el paralelizar
la creaci\'on de estos. Se debe utilizar una manera eficiente de almacenarlos, 
dado que si se utiliza una matriz del tama\~no de la dMRI suceder\'a que entre
el 40\% y el 80\% (ver Secci\'on X) de los valores guardados ser\'an nulos. El
clustering en si tambi\'en es importante de analizar. Dependiendo el algoritmo,
la m\'etrica y el tipo de \textit{linkage} que se utilice tanto los costos
computacionales como los resultados variar\'an significativamente. Si adem\'as 
se eligi\'o alguna estructura distinta a la implementaci\'on est\'andar de una
matriz para representar las \textit{features} va a ser necesario modificar las
implementaciones de clustering para que utilicen est\'as de manera eficiente.
Finalmente, una vez que las semillas fueron agrupadas, a\'un es necesario mapear
cada semilla con su respectivo voxel en la corteza cerebral. \\

En los \'ultimos a\~nos se han estudiado los resultados de aplicar distintos 
tipos de clustering sobre tractogramas para parcelar la corteza.  Por
ejemplo Behrens et. al \cite{Behrens2003} utilizan \textit{Target-Based Clustering},
este algoritmo define como restricci\'on que cada \'area solo puede estar
conectada con alguna otra perteneciente a un conjunto predefinido. Anwander et.
al \cite{Anwander2006} parcelan el \'Area de Broca utilizando \textit{k-means},
para lo cual se necesita definir un n\'umero de parcelas a encontrar. 
Moreno-Dominguez et al. \cite{Moreno-Dominguez2014} sit\'uan semillas en la
interfaz entre materia blanca y materia gris, a partir de las cuales se crean
tractogramas. Dichos tractogramas son luego agrupados utilizando el algoritmo
\textit{Agglomerative Hierarchical Clustering} con la distancia coseno como
medida de similitud, y un centroide como representante de cada Cluster. La gran
ventaja de este \'ultimo caso es que la parcelaci\'on resultante no posee
restricciones respecto al n\'umero de parcelas a generar, o sobre como las mismas 
deber\'ian estar conectadas. Por el contrario, dependiendo la forma en que se
desee cortar el dendrograma se obtendr\'an distintos grados de granularidad en
la parcelaci\'on. En resumen, se han propuesto varios m\'etodos para parcelar la
corteza cerebral, actualmente el trabajo de Moreno-Dominguez es el que posee menor 
cantidad de restricciones. Sin embargo pareciera haber un problema formal en \'el,
ya que no queda claro si el criterio que utilizan para medir distancias entre
clusters y la forma de representar la uni\'on de los mismos son compatibles. \\

El objetivo de este trabajo es analizar los m\'etodos de clustering jer\'arquico
actuales para parcelar toda la corteza y proponer un nuevo enfoque. Utilizamos
la base de datos provista por Human Connectome Project, \'esta posee la dMRI de
varios sujetos organizada por sexo y edad, con la ventaja de que todos los datos
est\'an ya preprocesados. Comenzamos analizando distintos algoritmos de
tractograf\'ia, dada la naturaleza estoc\'astica de los mismos es importante
comprobar si el resultado se estabiliza al utilizar un n\'umero suficientemente
grande de semillas. Tambi\'en es importante determinar si algoritmos diferentes
convergen a una misma soluci\'on. En particular mostramos que para dos
implementaciones distintas, ambas basadas en la librer\'ia \textit{dipy}, los
tractogramas convergen, y lo hacen a resultados similares. Luego comparamos distintos 
m\'etodos para posicionar las semillas dentro de la materia blanca a cierta 
distancia de la corteza cerebral. Concluimos que el mejor m\'etodo se basa en 
utilizar una implementaci\'on de \textit{Fast Marching Method} para generar un
mapa de distancias con signo de cada voxel a la materia gris, respetando la estructura
de la materia blanca. Las posiciones son entonces seleccionadas partiendo desde
cada punto de la corteza y caminando por el gradiente del mapa de distancias una
determinada distancia. Esto nos permite a su vez poder guardar un mapeo directo
desde las semillas hasta la corteza cerebral. Una vez que tenemos las semillas
procedemos a generar los tractogramas, para ello paralelizamos los experimentos
Monte Carlo, corriendo cada uno en un nodo distinto de un cluster. Cada tractograma 
es almacenado por su cuenta, pero al mismo tiempo se guarda una matriz de 
dimensiones $NxM$, siendo $N$ el n\'umero de tractogramas y $M$ la longitud 
de los mismos al transformarlos a una sola dimensi\'on. Mostramos que la mejor
forma de almacenar los mismos es utilizando matrices ralas. Finalmente analizamos
el m\'etodo de clustering descrito por Moreno-Dominguez y presentamos una mejora 
utilizando una transformaci\'on entre el espacio de los tractogramas
y un espacio vectorial donde es posible utilizar la m\'etrica euclidiana. 
Mostramos que nuestro m\'etodo permite salvar los problemas formales que el
anterior presenta. Durante este \'ultimo paso tambi\'en presentamos una 
implementaci\'on propia del algoritmo \textit{Agglomerative Hierarchical
Clustering} que aprovecha fuertemente las propiedades de las matrices ralas tanto
en el espacio de los tractogramas como en el espacio de la transformaci\'on, 
optimizando as\'i el costo temporal y espacial.
