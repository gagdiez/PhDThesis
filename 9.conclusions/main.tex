\chapter{Conclusion}

\section{Overview}
In this thesis we tackled the problem of parcellation the brain based on its axonal
connectivity, estimated by means of Diffusion MRI. The first contribution we reported
is a parcellation
technique that creates both single subject and groupwise multi-scale parcellations
of the brain~\cite{Gallardo2017a} based on long range connectivity. Then, we proposed
a new method to find correspondence between structural parcellations of different 
subjects~\cite{Gallardo2018}. Finally, we introduced a new label fusion technique
to infer the location of  white-matter bundles in patients with brain pathology~\cite{Guillermo2018}.
In this last chapter, we discuss our three contributions in this thesis,
and provide some perspectives for the field.

\section{Discussion}
Brain parcellation is a way of dimensionality reduction. Parceling the brain
allows to abstract the interaction between billions of neurons into a tractable
number of interacting regions. Accumulated evidence suggest that regions with
distinct function or cytoarchitecture also possess distinct axonal connectivity~\citep{Passingham2002, Johansen-Berg2004, Honey2009, Eickhoff2010}.
Hence, understanding how the cortex is arranged based on its axonal connectivity
could provide key information in unraveling brain organization. This motivates
the first contribution of our thesis.

\subsection{Structural Connectivity Based Parcellation}
Our first contribution is a parsimonious model for the long-range axonal
connectivity (extrinsic connectivity), and an efficient technique to divide the
brain in regions with homogeneous extrinsic connectivity~\cite{Gallardo2017a}.
Our model assumes that the cortex is divided in patches of homogeneous extrinsic
connectivity and, using logistic random effects, accounts for intra-patch and
across subject variability in the connections.

Leveraging our proposed model, we presented an efficient technique to parcellate
the cortex based on its extrinsic connectivity. Our parcellation technique presents
many advantages. First, it works directly with the dMRI information, not needing
to rely on an initial parcellation. Second, our technique can create both
single subject and groupwise parcellations of the whole cortex. Third, inspired
by Moreno-Dominguez et al.~\cite{Moreno-Dominguez2014}, our technique uses Hierarchical
Clustering to comprise multiple granularities of the same parcellation in a
dendrogram. This allows us to overcome the need to specify an expected number
of clusters. Fourth, thanks to our model, we can retrieve a parcellation from
the dendrogram using a simple technique: horizontal cut~\cite{Murtagh2011}.
Finally, also thanks to our model, our technique works with tractograms
in the Euclidean space~\cite{Pohl2007}, which allows us to use Ward clustering
to compute clusters with minimum intra-cluster variance.

It is important to also point the trade-offs in which our technique incurs. First,
our groupwise parcellation technique relies on anatomical seed-correspondence across
subjects. Given anatomical differences across-subjects, a purely anatomical
matching of seeds is probably sub-optimal. However, this allows us to compute
single and groupwise parcellations independently, without the need to impose
constraints between both. Second, in comparing high-dimensional vectors with the
Euclidean distance, we are probably affected by the curse of dimensionality~\cite{Beyer1999}.
However, working with the Euclidean distance allows us to generate clusters with
minimum intra-cluster variance by means of  Ward's Clustering.

The biggest limitations of our technique come from tractography, the tool
we used to estimate the brain's extrinsic connectivity. Tractography suffers
from at least three problems: gyral bias, distance bias, and false positives.
The gyral bias refers to the fact that streamlines tend to terminate preferentially
in gyri, in contrast with the general widely distributed connections obtained with
tracers~\cite{VanEssen2014}. The distance bias reflects that connectivity strength
that is usually inferred from tractography tends to decrease with distance
between the source and target area~\cite{Jbabdi2013}. Finally, recent studies
show that state-of-the-art tractography algorithms create four times more false
positives than true positives~\cite{Hein2016}. However, these limitations do not
imply that tractography is plain wrong~\cite{Asimov1988}. In fact, it has been
shown that tractography can recover major white matter bundles~\cite{Catani2008}.
Also, studies found correlation tractography-based tracts and tracer-based
ones in Old World monkeys~\cite{Donahue2016}.

The most important point to highlight about our parcellation technique, is that
it created parcels consistent with brain function. Particularly, some of our
parcels showed a high spatial overlap with responses to functional and cognitive
tasks measured with functional MRI. This was specially the case with motor functions as
hand movement, foot movement and tongue movement. Other cognitive functions that
show a good overlap were: face recognition, shape recognition and short-story
categorization. These results reflect the strong relationship between extrinsic
connectivity and functional specialization in the human brain cortex. 

Something noticeable in our resulting parcels is that the single subject parcellations
show big variability across subjects. This is in fact not specific of our methodology,
brain parcellations tend to differ in the number, shape, and spatial localization
across subjects~\cite{Jbabdi2013}. This does not necessarily reflect a problem
with the parcellation methodology, human brains are different from each other,
and ``individual variability is not noise''~\cite{Zilles2013}. Studying brain
connectivity therefore faces the challenge of locating homogeneous regions while
accounting for this variability. This motivates our second contribution.

\subsection{Matching Structural Parcels Across-Subjects}
Our second contribution, is a method to find correspondence between structural
parcellations based on Optimal Transport. Given two parcellations, we aim to
find the mapping between the parcels in both brains, based on their connectivity
fingerprints. Particularly, our method tries to find the optimal way to transport
connectivity fingerprints in one brain to the other. In this way, parcels with
similar connectivity fingerprints are matched.

Our technique allows matching two parcels based on how they are connected,
no matter if they do not spatially overlap after registration. The main advantages
of our method is that it is highly performant, and outperforms state-of-the-art
matching techniques in our experiments. Using Optimal Transport has an important
advantage over existent matching techniques. Existing matching techniques define
a similarity (e.g. euclidean or cosine), and match a parcel to that with the
most similar connectivity fingerprint. Optimal Transport does the same, while
taking into account all the parcels at the same time. In this way, the
obtained match is that which maximizes the similarity between parcels globally.

The major limitation of our technique is the necessity to define coherent
fingerprints across subjects. In order to compare two connectivity fingerprints,
both have to be defined over the same target regions. If the first fingerprint
is a 3-dimensional vector representing the connectivity to regions A, B and C,
then the second fingerprint must represent the same. This limitation is not
specific to our method. All current techniques that we know of have it. However,
studies have shown how to define such correspondences, allowing even to match
fingerprints across species~\cite{Mars2016, Mars2018}.

\subsection{Inferring White-Matter Tracts in the Presence of Pathology}
Our third contribution is a label-fusion techniques that leverages dMRI information
in order to infer white matter structures in the brain. We profit of the fact
that water particles diffuse along white matter bundles to improve Majority
Voting~\cite{Xu1992}. More specifically, we weight each vote based on how the voted pathway
is supported by the target's diffusion data. In this way, voted
pathways that better resemble the white matter of the target subject obtain a 
higher weight.

Our technique presents many advantages. First, our technique can achieve accurate
segmentations even when inferring from few subjects. Second, our technique
allows infering white matter pathways without the need of tracking. This is of
special importance in the presence of brain pathology. Third, while we weight the
votes of each tract based on dMRI information, we do not need to register
DWIs. Registering DWIs is a highly time consuming task~\cite{ODonnell2017}.
Instead, we register tracts across subjects and use them to estimate the diffusion
information. Fourth, our technique obtains a higher precision than Majority
Voting, as a trade-off, our technique possess a lower sensitivity. Meaning,
our technique has less false positives at the cost of more true negatives.
Fifth, in our simulated experiments, out technique did not label regions with
severe pathologies.

As with every label-fusion technique, the main limitation of our technique
is the necessity to do registration. While registration of healthy subjects
is a well known and studied task, registering subject with a brain pathology
is still an open problem. Most registering algorithms fail or return inaccurate
registrations, hampered by the pathology~\cite{Lutkenhoff2014}. In such cases
where the registration is highly inaccurate, our technique will most probably fail.

\section{Perspectives}

We believe that some contributions from the thesis can now be applied
to answer more neuroscientific questions. Our parcellation technique divides
the brain in regions of homogeneous extrinsic connectivity. This can give
a sound basis for studies based on connectivity. One example is the collaboration
with Samuel Deslauriers-Gauthier, from Inria. Briefly, we parcelled a subject's
cortex based on its extrinsic connectivity and then detect and estimate the information
flow between parcels using connectivity and MEG data~\cite{Gallardo2017b}.
Given that information flows through axons, our structural parcellation is suitable to predefine the
interconnected regions through which information flows. Another example is the
collaboration with Matteo Frigo et al., from Inria. We parcelled 3 subjects
and used the resulting parcels to filter false positives streamlines using a
functionally informed COMMIT~\cite{Frigo2018}. A final example is our collaboration
with Gaston Zanitti, student from the University of Buenos Aires. The aim of
our work is to study if specific responses to functional activation measured
with fMRI can be predicted using structural connectivity. For this, we define
cortical regions with our parcellation technique, and see if the structural
connectivity of the vertices within a region can predict its functional activation.

These applications aside, we expect to be able to use our parcellation technique to
further characterize the relationship between brain structure and brain function.
In particular, in further work we will try to estimate the overlap between
our parcels and more functional activations from fMRI. We will also focus on
studying the functional properties of the structural networks derived from
our parcellation. Meaning, we will study if the structural and functional
connections between regions generated by our technique are correlated. 
As a long term project, we expect that improvements in tractography as well as
in dMRI acquisition allow us to infer an atlas of human connections in the brain.
We believe this to be a doable yet arduous task, hindered presently by the
limitations of tractography.

Our matching technique could have major implications in the study of brain
connectivity and its relationship with brain function, allowing for the location
of parcels with similar connectivity but not high spatial coherence. Also, it
could improve results in the areas where matching techniques are
being used. This is, our matching technique could help to better understand
the link between different brain atlases, and improve the comparisons of cortical
areas between higher primates.

Finally, our label-fusion technique needs more testing in clinical data before
being able to use it to infer bundles affected by pathologies. Future work
will aim to keep developing tests with which to benchmark our technique.

\section{Conclusion}
In this thesis we tackled the problem of brain parcellation based on structural
connectivity. During the whole thesis, We putted emphasis on the importance
of the structural connectivity, and how it is strongly related to brain function.
However, we fell important to conclude the thesis reminding the reader that,
if a universal parcellation of the brain exists, it has not been found yet.
Presently, different atlases and techniques to divide the brain coexist, each
one with different and disadvantages. At the end, which parcellation to use in
practice will heavily depend on the hypothesis and the goal of the study to be
done.


% intro
%
% there are many parcellations, but this is ok
% we propose a parcellation, it has all of this good
% but in the single subject level we cannot match subjects
% we propose how to match subjects
% we propose how to find tracts in the presence of lesions
%
% the problem of brain atlas is not finished
% the problem of the connectivity atlas is not finished
% we need new parcellations, its not only important to be reproducible, also we need relationship
% microstructure maybe can give new parcellations
% we need better ways to characterize function vs structure
% 

%In Chapter \ref{ch:brain_mapping} we reviewed Cytoarchitectonic, Anatomical, Functional,
%Structural, and Multi-modal parcellations, highlighting their advantages and disadvantages.
%Cytoarchitectonic parcellations denote the brain's cellular composition,
%making them perfect to abstract populations
%of neurons. However, they can only be estimated post mortem and require years
%of work~\cite{Amunts2007}. Anatomical parcellations are highly reproducible across
%subjects, but incur in the trade-off of having coarse regions. Functional
%parcellations map cognitive functions into the brain. This has the limitation
%of having to define and isolate cognitive functions before being mapped, which
%could impossible~\cite{Bressler2010}. Structural parcellations are based on
%axonal connectivity, which is strongly related to cytoarchitecture and function~\citep{Passingham2002,Muircheartaigh2018}.
%The main downside is that tractography, the underlying technique of structural
%parcellations, is dominated by false positives~\cite{Hein2016}. Finally, Multi-modal
%parcellations combine information from different neuroimaging methodologies.
%Their main limitation is that sometimes, regions tend to over represent one modality.
%

%Brain parcellation is a way of dimensionality reduction. Parceling the brain
%allows us to abstract the interaction between billions of neurons into a tractable
%number of interacting regions. Such abstraction is supported by diverse studies
%showing that neurons tend to organize and activate in spatially coherent regions~\cite{Meynert1872, Brodmann1909, VonEconomo1925, Penfield1954, VonderMalsburg1994, Schmahmann2006, Stephan2013}.
%Brain regions can be defined based on criteria such as cytoarchitecture, function,
%anatomy or structural connectivity. Accumulating evidence suggest that there is
%a strong relationship between brain cytoarchitecture, brain function and brain
%structure. Particularly, regions with distinct
%function or cytoarchitecture also possess distinct structural connectivity~\citep{Passingham2002, Johansen-Berg2004, Honey2009, Eickhoff2010}.
%Therefore, understanding how the cortex is arranged based on its structural
%connectivity could provide a great insight on brain organization.


\chapterbib
