\section{Clustering de semillas: Moreno-Dominguez.}
\label{sec:clustering_moreno}

En la secci\'on anterior mostramos como crear tractogramas. El cuarto paso
para parcelar la corteza cerebral es agrupar estos tractogramas.
Moreno-Dominguez et al. \cite{Moreno-Dominguez2014} implementan el
algoritmo \textit{Agglomerative Hierarchical Clustering}. En este
algoritmo, cada \textit{feature} comienza en un
$cluster$ distinto. Luego, el algoritmo selecciona iterativamente dos
$clusters$ siguiendo alg\'un criterio de similitud; los agrupa en un nuevo
$cluster$ y crea un elemento representativo de este. La jerarqu\'ia
resultante de agrupar todos los $clusters$ es expresada como un
dendrograma.
En el trabajo de Moreno-Dominguez utilizan como medida de similitud la
distancia coseno (ecuaci\'on \ref{eq:cosine}) y como criterio de 
\textit{linkage} el centroide (ecuaci\'on \ref{eq:centroide}).

\begin{figure}[h!]
                                                                                                                        
\begin{minipage}[b]{0.49\textwidth}
    \begin{equation}
        \label{eq:cosine}
        simil_{cos}(X,Y) = 1 - \frac{ X \cdot Y }{||X|| ||Y||}
    \end{equation}
\end{minipage} ~
\hfill
\begin{minipage}[b]{0.49\textwidth}
    \begin{equation}
        \label{eq:centroide}
        centroide(X,Y) = \frac{ n_X X + n_Y Y}{n_X + n_Y}
    \end{equation}
    %\caption{\small $X, Y \in R^m; n_Z = #Z$}
\end{minipage} ~

\centering
\vspace{0.5cm}
\small{$X, Y \in R^m$, $n_z = \#z$}

\end{figure}  

Para crear los tractogramas transforman los mapas de visitas mediante la
ecuaci\'on \ref{eq:normalizacion}. Luego cambian a cero todos los voxels
que poseen un valor menor a $0.4$. Si se usan 15000 part\'iculas, entonces 
estos voxels fueron visitados por menos del 0.3\% de ellas. \\

\settowidth\mylen{procedure Clustering(}
\addtolength\mylen{\parindent}

\begin{algorithm}[h]
\caption{Modificaciones al algoritmo Agglomerative Hierarchical Clustering.}
\label{alg:morenoahc}
\begin{algorithmic}[1]

\Procedure{Clustering(k\_pasos: primeros K pasos, \\ \hspace*{\mylen}
                      tractogramas: mat. de tractogramas, \\ \hspace*{\mylen}
                      vecinos: mat. de vecinos, \\ \hspace*{\mylen}
                      distancias: mat. de distancia clusters ) }{}
                      
    \State \emph{jerarquia} $\gets \emptyset$
                      
\For{\emph{k} in [1, cant(\emph{tractogramas})] }

    \If{\emph{k} $>$ \emph{k\_pasos}}

        \State \emph{$C_x$, $C_y$} $\gets$ clusters tales que $x$ e $y$
                                           poseen \emph{distancia}
                                           minima      
            
    \Else{}

        \State \emph{$C_x$, $C_y$} $\gets$ $x$ e $y$ son \emph{vecinos}; 
                                   de \emph{distancia} minima y de tama\~no
                                   similar.

    \EndIf
    
    \State {tractogramas} $\gets$ eliminar clusters $C_x$, $C_y$

    \State {centroide} $\gets$ computar explicitamente el centroide $(C_x,C_y)$ 

    \State {tractogramas} $\gets$ agregar \emph{centroide}
    
    \State \emph{jerarquia} $\gets$ agregar la union $(C_x,C_y)$
    
    \For{ $C_z$ in \emph{tractogramas} }
        \State \emph{D} $\gets$ computar explicitamente distancia coseno
                                de $C_z$ al \emph{centroide}
        \State \emph{distancias} $\gets$ actualizar distancia entre
                                         $(C_x,C_y)$ y \emph{centroide}
                                         con \emph{D}
    \EndFor            
    
\EndFor

\State \Return \emph{jerarquia} 
 
\EndProcedure 

\end{algorithmic}
\end{algorithm}


Moreno-Dominguez et al. realizan varias modificaciones al algoritmo 
\textit{Agglomerative Hierarchical Clustering} para mejorar su resultado.
Aqu\'i daremos solo una breve descripci\'on de los mas relevantes, para
mayores detalles favor de referirse al paper.\\

Una de las mayores modificaciones es agregar un par\'ametro
para cambiar las primeras iteraciones del algoritmo.
Dado un n\'umero $k$, las primeras $k$ uniones son entre $clusters$
vecinos y de tama\~no similar. Esto es, solo los $clusters$ que se
encuentran a menos de cierta distancia f\'isica en el cerebro pueden ser
unidos. A su vez, solo se unen los $clusters$ que poseen un tama\~no
similar
para que el dendrograma crezca de manera balanceada. Estos cambios se
pueden ver reflejados en el algoritmo \ref{alg:morenoahc}. \\

Otra modificaci\'on al algoritmo de $clustering$ les permite quitar
\textit{outliers} del dendrograma resultante. Evitan que los $clusters$ de
un solo elemento se unan a otros $clusters$ si la (di)similitud es mayor a
cierto \textit{threshold}. Al hacer este paso durante el 
\textit{clustering} previenen que los  \textit{ouliers} afecten la forma
de los nuevos centroides.\\

\begin{figure}[h!]
                                                                                                                        
\begin{minipage}[b]{0.49\textwidth}
    \includegraphics[width=\textwidth]{img/inversion_0.png}
    \caption{\small Dendrograma con una inversi\'on.}
     \label{fig:inversion}
\end{minipage} ~
\hfill
\begin{minipage}[b]{0.49\textwidth}
    \includegraphics[width=\textwidth]{img/inversion_1.png}
    \caption{\small Dendrograma con inversi\'on corregida. }
    \label{fig:no_inversion}
\end{minipage} ~

\end{figure}  

Una vez obtenido el dendrograma proceden a eliminar las inversiones dentro
del mismo. Una inversi\'on sucede cuando se unen dos $clusters$ con una
distancia interna mayor a la distancia entre ellos. Las inversiones no
cambian la jerarqu\'ia de los $clusters$, sino que solo complican la
interpretaci\'on visual de los datos \cite{Murtagh1985}. Una forma de
eliminarlas es colapsando las ramas que la componen en una sola jerarqu\'ia
con mas de dos elementos. Un ejemplo de inversi\'on y el resultado de
quitarla se muestran en las Figuras  \ref{fig:inversion} y 
\ref{fig:no_inversion} respectivamente. \\

Concluidos todos estos pasos se obtiene un dendrograma que representa 
el agrupamiento de los tractogramas de manera jer\'arquica. Para parcelar
la corteza solo es necesario seleccionar una altura en la cual cortar
dicho dendrograma. Los $clusters$ que est\'en por debajo de ese
corte formaran las distintas parcelas. \\
