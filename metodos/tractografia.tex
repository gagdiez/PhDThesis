\section{Creando tractogramas a partir de las semillas}

Un tractograma es una imagen donde cada voxel representa la probabilidad de que
ese punto del cerebro est\'e conectado a la semilla elegida mediante un conjunto
de axones. Una forma de crear el tractograma de una semilla es generar un gran
n\'umero de streamlines desde ella y luego calcular la frecuencia de visitas por
cada voxel. Se denomina \textit{streamline} al camino que puede realizar una
part\'icula de agua siguiendo un mapa probabil\'istico de transiciones entre voxels.
Es importante destacar que el estimar los tractogramas de esta manera genera un 
sesgo respecto a la distancia. Cuanto m\'as lejos est\'a un voxel mayor es el 
n\'umero de transiciones probabil\'isticas necesarias para llegar a \'el. Esto 
provoca que los voxels lejanos tengan intrinsecamente valores peque\~nos. \\

Usando el m\'etodo descrito en la secci\'on anterior situamos semillas en la 
materia blanca de cada sujeto a $3mm$ de profundidad. Por cada semilla simulamos
el recorrido de quince mil part\'iculas de agua a trav\'es de la materia blanca. 
En base a los \textit{streamlines} resultantes se crearon mapas de visitas para
cada semilla. Un mapa de visitas es una imagen donde cada voxel posee el n\'umero
de \textit{streams} que lo conectan con la semilla. Luego usamos la transformaci\'on
propuesta en Moreno-Dominguez et al. \cite{Moreno-Dominguez2014}:

$$ T_i = \frac{ log(M_i + 1)}{log(p+1)} $$

Donde $T_i$ es el valor del voxel $i$ en el tractograma resultante; $M_i$ es 
el valor del voxel $i$ en el mapa de visitar y $p$ es el n\'umero de part\'iculas
usadas. Esto nos permite evitar el sesgo producto de la distancia que obtendr\'iamos
al dividir por la cantidad de part\'iculas utilizadas. Dado que 
$M_i \leq p, \forall i \in [1..N]$ cada voxel del tractograma tendr\'a un valor
entre uno y cero. $T_i = 1$ representa que todos los \textit{streams} pasaron por
el voxel, mientras que $T_i = 0$ representa que ninguno pas\'o. \\
