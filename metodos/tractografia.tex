\section{Estabilidad algor\'itmica}
\label{sec:estabilidad}

Un tractograma es una imagen donde cada voxel representa la probabilidad de que
ese punto del cerebro est\'e conectado a la semilla elegida mediante un conjunto
de axones. Una forma de crear el tractograma de una semilla es generar un gran
n\'umero de streamlines desde ella y luego calcular la frecuencia de visitas por
cada voxel. Se denomina \textit{streamline} al camino que puede realizar una
part\'icula de agua siguiendo un mapa probabil\'istico de transiciones entre voxels.
Es importante destacar que el estimar los tractogramas de esta manera genera un 
sesgo respecto a la distancia. Cuanto m\'as lejos est\'a un voxel mayor es el 
n\'umero de transiciones probabil\'isticas necesarias para llegar a \'el. \\

Dipy es una librer\'ia para python que contiene, entre otras herramientas,
varios algoritmos para generar \textit{streamlines}. Para este trabajo elegimos
utilizar la implementaci\'on de \textit{LocalTracking} (LT de aqu\'i en mas) que
se encuentra en el paquete \textit{dipy.tracking.local}; y una implementaci\'on 
propia (MSL de aqu\'i en m\'as). Ambos algoritmos poseen una estructura 
similar: Encuadran la imagen de difusi\'on en un modelo; crean un objeto que les
permita seleccionar una direcci\'on hacia donde caminar en base a la posici\'on
actual y se mueven hasta cumplir un criterio de parada.\\

En ambos casos modelamos la informaci\'on de dMRI usando el \textit{Constrained
Spherical Deconvolve Model}. La principal diferencia entre los algoritmos surge
en la forma en que seleccionan c\'omo avanzar. Dado el conjunto de direcciones 
iniciales, esto es, las direcciones posibles a tomar desde la semilla,
LT intenta en sucesivas repeticiones del experimento elegir una distinta, 
usando as\'i todas al menos una vez. MSL por otro lado selecciona una al
azar cada vez que repite el experimento. A su vez, LT utiliza un criterio de 
parada basado en la anisotrop\'ia de la difusi\'on; MSL usa una mascara ya 
predefinida. Para mayores detalles referirse al Anexo\\

Algunas preguntas interesantes a realizar sobre los tractogramas son: ¿Al repetir
el experimento, podremos obtener el mismo tractograma?; ¿Cu\'antas part\'iculas son
necesarias para ello? y ¿Qu\'e tanto difieren los resultados entre los distintos 
algoritmos de tractograf\'ia?\\

Para determinar si los algoritmos se estabilizaban y el n\'umero de part\'iculas
necesario para que eso suceda utilizamos la t\'ecnica estad\'istica de
\textit{bootstrap} \cite{Efron1982}. Bootstrap es una forma de aproximar la
distribuci\'on del muestreo de un estad\'istico en base a calcular el mismo
utilizando sucesivos remuestreos de los datos con repeticiones. Esto es
especialmente \'util cuando el n\'umero de muestras que se posee de la poblaci\'on
no es significativamente alto. En nuestro caso situamos mas de setecientas
semillas en el \'Area de Broca y luego generamos quince mil streamlines por cada
una. Luego calculamos el tractograma medio y la varianza de cada voxel utilizando
mil submuestras aleatorias del mismo tama\~no. Esto se repiti\'o con varios
tama\~nos de submuestra para estudiar as\'i la variabilidad a medida que la
cantidad de part\'iculas crec\'ia.\\
