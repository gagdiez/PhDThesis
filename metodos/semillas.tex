\section{Seleccionando voxels que ser\'an semillas}
\label{sec:semillas}

El segundo paso para parcelar la corteza cerebral es seleccionar que voxels
ser\'an semillas.
La materia gris est\'a compuesta principalmente por neuronas, y la materia
blanca por axones que las comunican \cite{Dale2008}. Como cada
neurona posee asociado un ax\'on, colocar semillas en la interfaz entre la
materia gris y la blanca permite caracterizar las neuronas de la corteza
\cite{Mori2002} \cite{Anwander2006}. Cibu et al. \cite{Thomas2014} muestran
que la materia blanca cercana a la materia gris est\'a interconectada por
peque\~nos axones. Como estamos interesados en realizar un estudio de las
conexiones entre regiones distantes del cerebro, decidimos situar las
semillas a \textit{3mm} de la corteza, evitando as\'i el efecto de \'estos
axones locales. El problema es que la corteza del cerebro no es uniforme,
sino que est\'a llena de surcos y circunvoluciones. Calcular la distancia
entonces no es inmediato, necesita un m\'etodo que tome estas propiedades
en cuenta. A continuaci\'on presentamos el m\'etodo \textit{Fast Marching
Method} y como es posible utilizarlo para posicionar las semillas
respetando la forma de la materia blanca. \\

\textit{Fast Marching Method} es un m\'etodo para resolver num\'ericamente
una versi\'on restringida de la ecuaci\'on \textit{Eikonal}. La misma, en
su forma general, es una ecuaci\'on diferencial no lineal que se encuentra
com\'unmente en problemas de propagaci\'on de onda. Tiene la forma: 

$$ V(x) | \nabla u(x) | = F(x) , x \in \Omega $$ 

Donde $\Omega$ es un subconjunto abierto de $R^n$ con un \textit{buen
comportamiento} en su borde. $F(x)$ se denomina el costo temporal y $V(x)$
es la velocidad de la onda en cada punto. En el caso particular que
queremos resolver $u(x_\omega) = 0, x \in \delta\Omega$;  $F(x)=1$ y 
$V(x)=1$, por lo que la ecuaci\'on se resume a:

$$ | \nabla u(x) | = 1 , x \in \Omega $$ 

$u(v)$ en este caso representa el tiempo que tarda la onda en llegar desde
alg\'un elemento del borde hasta el punto $v$ movi\'endose a velocidad
constante de una unidad de espacio por unidad de tiempo. Dada la forma de
la velocidad, $u(v)$ tambi\'en representa \textbf{la distancia mas corta
que existe entre cualquier punto $v$ de la imagen y el borde de $\Omega$}.
Dependiendo la orientaci\'on que se elija, las distancias a los puntos
internos de la superficie ser\'an negativas y las distancias a los puntos
externos positivas (figura \ref{fig:fmm}). \textit{FMM} encuentra estas 
distancias en tiempo $O(n log(n))$ \cite{Sethian2001}, siendo $n$ la 
cantidad de voxels de la imagen.\\

\begin{figure}[h!]

\centering
\begin{minipage}[b]{0.7\textwidth}
    \includegraphics[width=\textwidth]{img/fmm.png}
    \caption{\small FMM sobre el hemisferio derecho. El borde la materia
                    blanca fue resaltado intencionalmente. Las distancias
                    a los puntos internos son negativas y las distancias a
                    los puntos externos positivas.}
    \label{fig:fmm}
\end{minipage} ~

\end{figure}  

Es posible utilizar este algoritmo para seleccionar voxels a cierta
profundidad en la materia blanca. Usando como borde la corteza cerebral
podemos crear un mapa de distancias en la materia blanca. El gradiente de
este mapa de distancias es un campo vectorial donde cada vector apunta
hacia el interior de la materia blanca. Caminar partiendo desde los puntos
en la siguiendo este campo permite adentrarse respetando la morfolog\'ia de
la materia blanca. Una ventaja de este m\'etodo es que permite guardar un
mapeo entre cada coordenada de la superficie y la semilla que la
representa. Otra ventaja es que es posible realizar todo el proceso en 
tiempo $O(n log(n))$. \\
