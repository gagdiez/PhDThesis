\chapter{Parcelando la Corteza Cerebral}

En el cap\'itulo anterior presentamos los fundamentos f\'isicos de la Resonancia
Magn\'etica Nuclear y como es posible utilizar estos para caracterizar la 
difusi\'on en el cerebro. Nuestro objetivo ahora es parcelar la corteza cerebral
haciendo uso de un criterio estructural. En particular, utilizando una imagen de
difusi\'on, queremos generar una parcelaci\'on mediante el agrupamiento de
tractogramas. Para ello es necesario primero seleccionar los voxels que ser\'an
utilizados como semilla de cada tractograma; luego generarlos y finalmente 
agruparlos usando alg\'un algoritmo de clustering. Dejamos el siguiente diagrama
para ser utilizado como referencia de los pasos a seguir: \\

\begin{figure}[h!]

\centering
\begin{minipage}[b]{\textwidth}
    \includegraphics[width=\textwidth]{img/diagrama.png}
    \caption{\small \textit{Pipeline} del proceso de parcelaci\'on }
    \label{fig:diagrama}
\end{minipage} ~

\end{figure}  

\section{Materiales utilizados en el estudio}

Todos nuestros experimentos fueron realizados con sujetos descargados de la base
de datos de \textit{Human Connectome Project}. Las ventajas de utilizar estos
datos son muchas: Tanto la imagen de difusi\'on como la anat\'omica se encuentran
ya preprocesadas \cite{Glasser2013}; cada sujeto posee una parcelaci\'on que, entre
otras cosas, permite separar la materia blanca de la gris; y cada sujeto posee una
superficie que representa su corteza cerebral.\\

\section{Seleccionando voxels que ser\'an semillas}
\label{sec:semillas}

El segundo paso para parcelar la corteza cerebral es seleccionar que voxels
ser\'an semillas.
La materia gris est\'a compuesta principalmente por neuronas, y la materia
blanca por axones que las comunican \cite{Dale2008}. Como cada
neurona posee asociado un ax\'on, colocar semillas en la interfaz entre la
materia gris y la blanca permite caracterizar las neuronas de la corteza
\cite{Mori2002} \cite{Anwander2006}. Cibu et al. \cite{Thomas2014} muestran
que la materia blanca cercana a la materia gris est\'a interconectada por
peque\~nos axones. Como estamos interesados en realizar un estudio de las
conexiones entre regiones distantes del cerebro, decidimos situar las
semillas a \textit{3mm} de la corteza, evitando as\'i el efecto de \'estos
axones locales. El problema es que la corteza del cerebro no es uniforme,
sino que est\'a llena de surcos y circunvoluciones. Calcular la distancia
entonces no es inmediato, necesita un m\'etodo que tome estas propiedades
en cuenta. A continuaci\'on presentamos el m\'etodo \textit{Fast Marching
Method} y como es posible utilizarlo para posicionar las semillas
respetando la forma de la materia blanca. \\

\textit{Fast Marching Method} es un m\'etodo para resolver num\'ericamente
una versi\'on restringida de la ecuaci\'on \textit{Eikonal}. La misma, en
su forma general, es una ecuaci\'on diferencial no lineal que se encuentra
com\'unmente en problemas de propagaci\'on de onda. Tiene la forma: 

$$ V(x) | \nabla u(x) | = F(x) , x \in \Omega $$ 

Donde $\Omega$ es un subconjunto abierto de $R^n$ con un \textit{buen
comportamiento} en su borde. $F(x)$ se denomina el costo temporal y $V(x)$
es la velocidad de la onda en cada punto. En el caso particular que
queremos resolver $u(x_\omega) = 0, x \in \delta\Omega$;  $F(x)=1$ y 
$V(x)=1$, por lo que la ecuaci\'on se resume a:

$$ | \nabla u(x) | = 1 , x \in \Omega $$ 

$u(v)$ en este caso representa el tiempo que tarda la onda en llegar desde
alg\'un elemento del borde hasta el punto $v$ movi\'endose a velocidad
constante de una unidad de espacio por unidad de tiempo. Dada la forma de
la velocidad, $u(v)$ tambi\'en representa \textbf{la distancia mas corta
que existe entre cualquier punto $v$ de la imagen y el borde de $\Omega$}.
Dependiendo la orientaci\'on que se elija, las distancias a los puntos
internos de la superficie ser\'an negativas y las distancias a los puntos
externos positivas (figura \ref{fig:fmm}). \textit{FMM} encuentra estas 
distancias en tiempo $O(n log(n))$ \cite{Sethian2001}, siendo $n$ la 
cantidad de voxels de la imagen.\\

\begin{figure}[h!]

\centering
\begin{minipage}[b]{0.7\textwidth}
    \includegraphics[width=\textwidth]{img/fmm.png}
    \caption{\small FMM sobre el hemisferio derecho. El borde la materia
                    blanca fue resaltado intencionalmente. Las distancias
                    a los puntos internos son negativas y las distancias a
                    los puntos externos positivas.}
    \label{fig:fmm}
\end{minipage} ~

\end{figure}  

Es posible utilizar este algoritmo para seleccionar voxels a cierta
profundidad en la materia blanca. Usando como borde la corteza cerebral
podemos crear un mapa de distancias en la materia blanca. El gradiente de
este mapa de distancias es un campo vectorial donde cada vector apunta
hacia el interior de la materia blanca. Caminar partiendo desde los puntos
en la siguiendo este campo permite adentrarse respetando la morfolog\'ia de
la materia blanca. Una ventaja de este m\'etodo es que permite guardar un
mapeo entre cada coordenada de la superficie y la semilla que la
representa. Otra ventaja es que es posible realizar todo el proceso en 
tiempo $O(n log(n))$. \\


\section{Estabilidad algor\'itmica}
\label{sec:estabilidad}

Un tractograma es una imagen donde cada voxel representa la probabilidad de que
ese punto del cerebro est\'e conectado a la semilla elegida mediante un conjunto
de axones. Una forma de crear el tractograma de una semilla es generar un gran
n\'umero de streamlines desde ella y luego calcular la frecuencia de visitas por
cada voxel. Se denomina \textit{streamline} al camino que puede realizar una
part\'icula de agua siguiendo un mapa probabil\'istico de transiciones entre voxels.
Es importante destacar que el estimar los tractogramas de esta manera genera un 
sesgo respecto a la distancia. Cuanto m\'as lejos est\'a un voxel mayor es el 
n\'umero de transiciones probabil\'isticas necesarias para llegar a \'el. \\

Para este trabajo elegimos utilizar una implementaci\'on de tractograf\'ia ya 
existente llamada \textit{LocalTracking} (LT de aqu\'i en mas) y un algor\'itmo
propio (MSL de aqu\'i en m\'as). Ambos algoritmos poseen una estructura 
similar: Encuadran la imagen de difusi\'on en un modelo; en base a ese modelo 
crean un mapa de transiciones probabil\'isticas entre voxels y lo recorren 
aleatoriamente hasta cumplir un criterio de parada. Para detalles sobre la 
implementaci\'on, por favor referirse al Anexo. \\

Algunas preguntas interesantes a realizar sobre los tractogramas son: ¿Al repetir
el experimento, podremos obtener el mismo tractograma?; ¿Cu\'antas part\'iculas son
necesarias para ello? y ¿Qu\'e tanto difieren los resultados entre los distintos 
algoritmos de tractograf\'ia?. \\

Para determinar si los algoritmos se estabilizaban y el n\'umero de part\'iculas
necesario para que eso suceda utilizamos la t\'ecnica estad\'istica de
\textit{bootstrap} \cite{Efron1982}. Bootstrap es una forma de aproximar la
distribuci\'on del muestreo de un estad\'istico en base a calcular el mismo
utilizando sucesivos remuestreos de los datos con repeticiones. Esto es
especialmente \'util cuando el n\'umero de muestras que se posee de la poblaci\'on
no es significativamente alto. En nuestro caso situamos mas de setecientas
semillas en el \'Area de Broca y luego generamos quince mil streamlines por cada
una. Luego calculamos el tractograma medio y la varianza de cada voxel utilizando
mil submuestras aleatorias del mismo tama\~no. Esto se repiti\'o con varios
tama\~nos de submuestra para estudiar as\'i la variabilidad a medida que la
cantidad de part\'iculas crec\'ia.\\


\section{Clustering de Semillas: Estado del Arte.}

Moreno-Dominguez et al. \cite{Moreno-Dominguez2014} implementan el algoritmo
\textit{Agglomerative Hierarchical Clustering} para agrupar los tractogramas. 
En este algoritmo, cada \textit{feature} comienza en un cluster distinto. Luego,
el algoritmo selecciona iterativamente dos clusters siguiendo alg\'un criterio
de similitud; los agrupa en un nuevo cluster y crea un elemento representativo
de este. La jerarqu\'ia resultante de agrupar todos los clusters es expresada
como un dendrograma. En el trabajo de Moreno-Dominguez utilizan como medida de
similitud la distancia coseno (Ecuaci\'on \ref{eq:cosine}) y como criterio de
\textit{linkage} el centroide (Ecuaci\'on \ref{eq:centroide}).

\begin{figure}[h!]
                                                                                                                        
\begin{minipage}[b]{0.49\textwidth}
    \begin{equation}
        \label{eq:cosine}
        similarity(X,Y) = 1 - \frac{ X \cdot Y }{||X|| ||Y||}
    \end{equation}
\end{minipage} ~
\hfill
\begin{minipage}[b]{0.49\textwidth}
    \begin{equation}
        \label{eq:centroide}
        centroide(X,Y) = \frac{ n_X X + n_Y Y}{n_X + n_Y}
    \end{equation}
    %\caption{\small $X, Y \in R^m; n_Z = #Z$}
\end{minipage} ~

\centering
\vspace{0.5cm}
\small{$X, Y \in R^m$, $n_z = \#z$}

\end{figure}  

Para mejorar los resultados del \textit{clustering} realizan distintos tipos de
preprocesamiento en varias etapas. Aqu\'i daremos solo una breve descripci\'on 
de los mas relevantes, para mayores detalles favor de referirse al paper. \\

Una de las primeras modificaciones es al algoritmo 
\textit{Agglomerative Hierarchical Clustering}. Dado un n\'umero $k$,
las primeras $k$ iteraciones sean entre clusters vecinos y de tama\~no similar.
Esto es, solo los clusters que se encuentran a menos de cierta distancia f\'isica
en el cerebro pueden ser unidos. A su vez, solo se unen los clusters que poseen
un tama\~no similar para que el dendrograma crezca de manera balanceada. 
Una vez obtenido el dendrograma proceden a eliminar las inversiones dentro del
mismo. Una inversi\'on sucede cuando se unen dos clusters con una distancia 
interna mayor a la distancia entre ellos. Las inversiones no cambian la 
jerarqu\'ia de los clusters, sino que solo complican la interpretaci\'on visual
de los datos \cite{Murtagh1985}. Una forma de eliminarlas es colapsando las 
ramas que la componen en una sola jerarqu\'ia con mas de dos elementos. Un 
ejemplo de inversi\'on y el resultado de quitarla se muestran en las Figuras 
\ref{fig:inversion} y \ref{fig:no_inversion} respectivamente. 


\begin{figure}[h!]
                                                                                                                        
\begin{minipage}[b]{0.49\textwidth}
    \includegraphics[width=\textwidth]{img/inversion_0.png}
    \caption{\small Dendrograma de inversi\'on.}
     \label{fig:inversion}
\end{minipage} ~
\hfill
\begin{minipage}[b]{0.49\textwidth}
    \includegraphics[width=\textwidth]{img/inversion_1.png}
    \caption{\small Dendrograma con inversi\'on corregida. }
    \label{fig:no_inversion}
\end{minipage} ~

\end{figure}  

\vspace{0.1cm}

Otro paso de preprocesamiento es el quitar \textit{outliers} del dendrograma.
Esto en realidad lo hacen durante la etapa de \textit{clustering}. Evitan que los
clusters de un solo elemento se unan a otros clusters si la (di)similitud es 
mayor a cierto \textit{threshold}. Al hacer este paso durante el \textit{clustering}
previenen que los \textit{ouliers} afecten la forma de los nuevos centroides.\\

Una vez finalizados todos los pasos el resultado es un dendrograma. Para parcelar
la corteza solo es necesario seleccionar una altura en la cual cortar dicho
dendrograma. Los clusters que est\'en por debajo de ese corte ser\'an las distintas
parcelas. \\

