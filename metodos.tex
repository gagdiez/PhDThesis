\chapter{M\'etodos}

Nuestro objetivo es parcelar la corteza cerebral mediante un criterio estructural.
En particular, utilizando una imagen de difusi\'on, queremos generar \'areas 
mediante el agrupamiento de tractogramas. Para ello es necesario primero seleccionar
los voxels que ser\'an utilizados como semilla de cada tractograma; luego generarlos
y finalmente agruparlos usando alg\'un algoritmo de clustering.  \\

\section{Adquisici\'on de datos}

Descargamos lo datos de X sujetos, todos (caracter\'isticas) del sitio web 
de \textit{Human Connectome Project}. Las ventajas de utilizar estos datos son muchas:
Tanto la imagen de difusi\'on como la anat\'omica se encuentran ya preprocesadas;
cada sujeto posee una parcelaci\'on que entre otras cosas permite separar la materia
blanca de la gris; cada sujeto posee un archivo en formato \textit{Gifti} que
representan la corteza, permitiendo visualizar un modelo en tres dimensiones
de la misma sencillamente, y una representaci\'on tambi\'en en \textit{Gifti} 
de la parcelaci\'on de dicha corteza. \\

\section{Estabilidad algor\'itmica}
\label{sec:estabilidad}

Un tractograma es una imagen donde cada voxel representa la probabilidad de que
ese punto del cerebro est\'e conectado a la semilla elegida mediante un conjunto
de axones. Una forma de crear el tractograma de una semilla es generar un gran
n\'umero de streamlines desde ella y luego calcular la frecuencia de visitas por
cada voxel. Se denomina \textit{streamline} al camino que puede realizar una
part\'icula de agua siguiendo un mapa probabil\'istico de transiciones entre voxels.
Es importante destacar que el estimar los tractogramas de esta manera genera un 
sesgo respecto a la distancia. Cuanto m\'as lejos est\'a un voxel mayor es el 
n\'umero de transiciones probabil\'isticas necesarias para llegar a \'el. \\

Para este trabajo elegimos utilizar una implementaci\'on de tractograf\'ia ya 
existente llamada \textit{LocalTracking} (LT de aqu\'i en mas) y un algor\'itmo
propio (MSL de aqu\'i en m\'as). Ambos algoritmos poseen una estructura 
similar: Encuadran la imagen de difusi\'on en un modelo; en base a ese modelo 
crean un mapa de transiciones probabil\'isticas entre voxels y lo recorren 
aleatoriamente hasta cumplir un criterio de parada. Para detalles sobre la 
implementaci\'on, por favor referirse al Anexo. \\

Algunas preguntas interesantes a realizar sobre los tractogramas son: ¿Al repetir
el experimento, podremos obtener el mismo tractograma?; ¿Cu\'antas part\'iculas son
necesarias para ello? y ¿Qu\'e tanto difieren los resultados entre los distintos 
algoritmos de tractograf\'ia?. \\

Para determinar si los algoritmos se estabilizaban y el n\'umero de part\'iculas
necesario para que eso suceda utilizamos la t\'ecnica estad\'istica de
\textit{bootstrap} \cite{Efron1982}. Bootstrap es una forma de aproximar la
distribuci\'on del muestreo de un estad\'istico en base a calcular el mismo
utilizando sucesivos remuestreos de los datos con repeticiones. Esto es
especialmente \'util cuando el n\'umero de muestras que se posee de la poblaci\'on
no es significativamente alto. En nuestro caso situamos mas de setecientas
semillas en el \'Area de Broca y luego generamos quince mil streamlines por cada
una. Luego calculamos el tractograma medio y la varianza de cada voxel utilizando
mil submuestras aleatorias del mismo tama\~no. Esto se repiti\'o con varios
tama\~nos de submuestra para estudiar as\'i la variabilidad a medida que la
cantidad de part\'iculas crec\'ia.\\


\section{Seleccionando voxels que ser\'an semillas}
\label{sec:semillas}

El segundo paso para parcelar la corteza cerebral es seleccionar que voxels
ser\'an semillas.
La materia gris est\'a compuesta principalmente por neuronas, y la materia
blanca por axones que las comunican \cite{Dale2008}. Como cada
neurona posee asociado un ax\'on, colocar semillas en la interfaz entre la
materia gris y la blanca permite caracterizar las neuronas de la corteza
\cite{Mori2002} \cite{Anwander2006}. Cibu et al. \cite{Thomas2014} muestran
que la materia blanca cercana a la materia gris est\'a interconectada por
peque\~nos axones. Como estamos interesados en realizar un estudio de las
conexiones entre regiones distantes del cerebro, decidimos situar las
semillas a \textit{3mm} de la corteza, evitando as\'i el efecto de \'estos
axones locales. El problema es que la corteza del cerebro no es uniforme,
sino que est\'a llena de surcos y circunvoluciones. Calcular la distancia
entonces no es inmediato, necesita un m\'etodo que tome estas propiedades
en cuenta. A continuaci\'on presentamos el m\'etodo \textit{Fast Marching
Method} y como es posible utilizarlo para posicionar las semillas
respetando la forma de la materia blanca. \\

\textit{Fast Marching Method} es un m\'etodo para resolver num\'ericamente
una versi\'on restringida de la ecuaci\'on \textit{Eikonal}. La misma, en
su forma general, es una ecuaci\'on diferencial no lineal que se encuentra
com\'unmente en problemas de propagaci\'on de onda. Tiene la forma: 

$$ V(x) | \nabla u(x) | = F(x) , x \in \Omega $$ 

Donde $\Omega$ es un subconjunto abierto de $R^n$ con un \textit{buen
comportamiento} en su borde. $F(x)$ se denomina el costo temporal y $V(x)$
es la velocidad de la onda en cada punto. En el caso particular que
queremos resolver $u(x_\omega) = 0, x \in \delta\Omega$;  $F(x)=1$ y 
$V(x)=1$, por lo que la ecuaci\'on se resume a:

$$ | \nabla u(x) | = 1 , x \in \Omega $$ 

$u(v)$ en este caso representa el tiempo que tarda la onda en llegar desde
alg\'un elemento del borde hasta el punto $v$ movi\'endose a velocidad
constante de una unidad de espacio por unidad de tiempo. Dada la forma de
la velocidad, $u(v)$ tambi\'en representa \textbf{la distancia mas corta
que existe entre cualquier punto $v$ de la imagen y el borde de $\Omega$}.
Dependiendo la orientaci\'on que se elija, las distancias a los puntos
internos de la superficie ser\'an negativas y las distancias a los puntos
externos positivas (figura \ref{fig:fmm}). \textit{FMM} encuentra estas 
distancias en tiempo $O(n log(n))$ \cite{Sethian2001}, siendo $n$ la 
cantidad de voxels de la imagen.\\

\begin{figure}[h!]

\centering
\begin{minipage}[b]{0.7\textwidth}
    \includegraphics[width=\textwidth]{img/fmm.png}
    \caption{\small FMM sobre el hemisferio derecho. El borde la materia
                    blanca fue resaltado intencionalmente. Las distancias
                    a los puntos internos son negativas y las distancias a
                    los puntos externos positivas.}
    \label{fig:fmm}
\end{minipage} ~

\end{figure}  

Es posible utilizar este algoritmo para seleccionar voxels a cierta
profundidad en la materia blanca. Usando como borde la corteza cerebral
podemos crear un mapa de distancias en la materia blanca. El gradiente de
este mapa de distancias es un campo vectorial donde cada vector apunta
hacia el interior de la materia blanca. Caminar partiendo desde los puntos
en la siguiendo este campo permite adentrarse respetando la morfolog\'ia de
la materia blanca. Una ventaja de este m\'etodo es que permite guardar un
mapeo entre cada coordenada de la superficie y la semilla que la
representa. Otra ventaja es que es posible realizar todo el proceso en 
tiempo $O(n log(n))$. \\


\section{Generando tractogramas}

Partiendo desde la corteza situamos semillas en la materia blanca de cada 
sujeto a $3mm$ de profundidad. Por cada semilla simulamos el recorrido de quince
mil part\'iculas de agua a traves de la materia blanca. En base a los \textit{streamlines}
resultantes se crearon mapas de visitas para cada semilla. Un mapa de visitas es
una imagen donde cada voxel posee el n\'umero de \textit{streams} que lo conectan
con la semilla. Si uno dividiera cada valor de este mapa por la cantidad de
part\'iculas utilizadas obtendr\'ia un tractograma. Sin embargo, por lo dicho en
la secci\'on \ref{sec:estabilidad}, este m\'etodo posee un sesgo respecto a la
distancia. Para evitarlo usamos la transformaci\'on propuesta en Moreno-Dominguez et
al. \cite{Moreno-Dominguez2014}:

$$ T_i = log(M_i + 1)/log(p+1) $$

Donde $T_i$ es el valor del voxel $i$ en el tractograma resultante; $M_i$ es 
el valor del voxel $i$ en el mapa de visitar y $p$ es el n\'umero de part\'iculas
usadas. Dado que $M_i \leq p \forall i$ cada voxel del tractograma tendr\'a un 
valor entre uno y cero. Uno representa que todos los \textit{streams} pasaron 
por el voxel, mientras que cero representa que ninguno pas\'o. \\

\section{Matrices Ralas}

Una forma posible de almancenar los tractogramas es convertir cada uno a un 
vector fila y almacenarlos todos juntos en una misma matriz. En nuestro caso,
hacer esto con los tractogramas del \'Area de Broca gener\'o una matriz de dimensiones
$762x3587328$. Asumiendo que cada valor se representa usando $8$ Bytes en memoria, 
esta matriz ocupa un total de aproximadamente $20$ Gigabytes. Sin embargo, solo
un $1\%$ de los datos almancenados resultaron ser no nulos. Casi toda la matriz
resulta ser espacio desperdiciado.\\

Es posible mejorar esto si eliminamos de la matriz las columnas que poseen solo
elementos nulos. La Figura \ref{fig:densa} muestra la matriz que resulta de aplicar
este procedimiento a los tractogramas del \'Area de Broca. Si bien nuestra nueva
matriz posee dimensiones $762x121045$, a\'un solo el $27\%$ de los valores 
almacenados son no nulos. Manteniendo la representaci\'on de $8$ Bytes redujimos
el espacio necesario de $20\%$ Gigabites a aproximadamente $700$ Megabytes.\\

\begin{figure}[h!]
   \centering
    \includegraphics[width=\textwidth]{img/densa_broca.png}
    \caption{Semillas en el hemisferio izquierdo. }
    \label{fig:densa}
\end{figure}

El problema de este \'ultimo m\'etodo es que a\'un desperdicia mucho espacio. En
el caso de utilizar todas los tractogramas de un hemisferio, la matriz pasa a ser
de dimensiones $21657x3587328$ con un $1\%$ de valores no nulos. Para almacenar
dicha matriz es necesario utilizar $587$ Gigabytes. Por esto es necesario
utilizar estructuras mas eficiente, como un \textit{Dictionary of Keys}, o
una matriz \textit{Compressed Sparse Row} (CSR). \\

%eliminando columnas inutiles nos queda 21657x145574, son 23.5 Gn

El \textit{Dictionary of Keys} (DOK) representa una matriz mediante un diccionario.
Almacena los valores de la matriz indexandolos con las coordenadas de su matriz.
Todas aquellas coordenadas que no poseen un valor asociado se asumen nulas. Esta 
estructura se suele utilizar para crear las matrices ralas. Luego, para realizar
operaciones con las mismas, resulta mas eficiente transformarlas a CSR. \\

\textit{Compressed Sparse Row} es otra manera de representar matrices
ralas. En este caso se utilizan tres vectores comunmente denominados $A$, $IA$ y
$JA$. $A$ posee todos los valores no nulos; los valores de $A$ entre los indices
$IA_i$ e $IA_{i+1}-1$ son los valores no nulos de la fila $i$. Finalmente, $JA_i$
posee el numero de columna al cual pertenece $A_i$. Por ejemplo, para la siguiente
matriz $M$:

$$
    M =
    \begin{pmatrix}
             0 & 0 & 0 & 0 \\
             5 & 8 & 3 & 0 \\
             0 & 6 & 0 & 0    
    \end{pmatrix}
$$    

$$ A  = \begin{pmatrix} 5 & 8 & 3 & 6     \end{pmatrix} $$
$$ IA = \begin{pmatrix} 0 & 0 & 2 & 3 & 4 \end{pmatrix} $$
$$ JA = \begin{pmatrix} 0 & 1 & 2 & 1     \end{pmatrix} $$

CSR permite realizar sumas, multiplicaciones y operaciones con sus filas de manera
eficiente. 

BUSCAR REFERENCIAS PARA TODO ESTO.


\section{Implementando el m\'etodo Moreno-Dominguez}
Ac\'a explico todo lo que hacen. Preprocesamiento de tractogramas, modificaciones
al algoritmo de clustering, preprocesamiento del clustering, preprocesamiento
de dendrograma.\\


\section{Transformaci\'on LogOdds}

Utilizar \textit{Hierarchical Agglomerative Clustering} con la distancia 
coseno y el \textit{linkage} centroide presenta al menos dos desventajas. 
Primero, la distancia coseno obliga a tener que comparar explicitamente cada
centroide con los clusters existentes. Notemos que esto implica mantener los
clusters en memoria. Por otro lado, el promedio de probabilidades no
necesariamente representa una probabilidad \cite{Pohl2007}. Esto quiere decir
que el centroide de un grupo de tractogramas no necesariamente representa un
tractograma. \\

Para solucionar estos inconvenientes proponemos transformar los tractogramas 
al espacio euclidiano utilizando la funci\'on LoggOdds. Sea $P_M$ el espacio 
de una distribuci\'on discreta para $M$ etiquetas: 

$$P_M = \left\{  p | p = (p_1,\dots p_n) \in (0,1)^M , \sum{p_i} = 1 \right\}$$

La funci\'on \textit{logit}:$P_M \rightarrow R^{M-1}$ define una transformaci\'on
entre el espacio $P_M$ y el espacio eucl\'ideo $R^{M-1}$. Dados los vectores $Q \in P^M$ y
$S \in R^{M-1}$:

$$S_i = logit(Q_i) = \frac{log(Q_i)}{log(Q_M)}$$

Para el caso de $M=2$ permite transformar la distribuci\'on Bernoulli
discreta al espacio eucl\'idio:

$$logit(p) = \frac{log(p)}{log(1-p)}$$

Trabajar en el espacio eucl\'ideo nos asegura que la suma y la multiplicaci\'on
por escalares est\'a contenido en el mismo espacio. Pohl et al. \cite{Pohl2007} 
utilizan esta propiedad para realizar operaciones lineales sobre mapas
probabil\'isticos. A su vez, demuestran que la suma y multiplicaci\'on
por escalares en el espacio eucl\'ideo poseen un significado en el espacio de la
distribuci\'on. \\

Asumiendo que cada voxel $v$ proviene de una variable aleatoria:

 $$X_v= \textrm{``La semilla est\'a conectada con el voxel v''}$$
 
Es posible aplicar la funci\'on \textit{logit} en cada voxel de los tractogramas.
El resultado es un vector donde cada coordenada se encuentra en el espacio
eucl\'ideo. Esto permite realizar operaciones lineales entre los mismos voxels
de distintos tractogramas, ya que todos provienen de una distribuci\'on
Bernoulli.\\

Nuestra segunda propuesta es cambiar la funci\'on de similitud usada durante el 
\textit{clustering}. Luego de aplicar la transformaci\'on los tractogramas
se encuentran en el espacio eucl\'ideo. Podemos entonces utilizar la m\'etrica
euclidea. Las ventajas respecto a usar la distancia coseno son varias. A 
continuaci\'on nombraremos algunas.\\



\begin{wrapfigure}{r}{0.35\textwidth}
    \begin{center}
        \vspace{-1cm}
        \includegraphics[width=0.35\textwidth]{img/3pop.png}
        \caption{Tres cluster colineales\-}
        \label{fig:3clusters}
    \end{center}
\end{wrapfigure}

La distancia coseno es una forma de medir correlaci\'on entre vectores. Por ello,
cuando lo que se intenta agrupar son vectores colineales el resultado es aleatorio.
Por ejemplo, cuando se aplica el procedimiento sobre los puntos de la Figura
\ref{fig:3clusters} el resultado del clustering es la Figura \ref{fig:3moreno}.
Usando LogOdds y la m\'etrica euclidiana se consigue el \textit{clustering} de 
la Figura \ref{fig:3logit}.\\

\vspace{1.5cm}

\begin{figure}[h!]

\centering                                                                                                          
\begin{minipage}[h]{0.8\textwidth}
    \includegraphics[width=\textwidth]{img/3pop_moreno.png}
    \caption{Clustering resultado de utilizar el m\'etodo Moreno}
    \label{fig:3moreno}
\end{minipage} ~

\end{figure}  

\begin{figure}[h!]

\centering
\begin{minipage}[h]{0.8\textwidth}
    \includegraphics[width=\textwidth]{img/3pop_logit.png}
    \caption{Clustering resultado de utilizar el m\'etodo logit}
    \label{fig:3logit}
\end{minipage} ~

\end{figure}  


\subsection{Similitud vs Linkage}

La Figura \ref{fig:cos_cen} muestra cuatro vectores, sus posiciones en coordenadas
polares son: 

$$ p_1 = (0.4, 45^\circ);  p_2 = (0.3, 25^\circ); p_3 = (0.4, 66^\circ); p_4 = (0.4, 4.5^\circ)$$

Podemos apreciar que al principio $d(p_2,p_3) < d(p_3,p_4) < d(p_1,p_2)$, siendo
$d(x,y)$ la distancia coseno. Sin embargo, luego de utilizar el 
\textit{linkage centroid} sucede que $d(p_1,p_c) < d(p_4,p_c)$, esto es, $p_4$
es ahora el punto que mas lejos est\'a del centroide. Si crearamos el representante
del nuevo cluster usando el \'angulo medio entre $p_2$ y $p_3$, entonces $p_4$
seguir\'ia siendo el punto mas cercano a la uni\'on. \\

\begin{figure}[h!]
                                                                                                                        
\begin{minipage}[b]{\textwidth}
    \includegraphics[width=\textwidth]{img/cosine_centroid.png}
    \caption{Contraejemplo para linkage centroide con la distancia coseno}
    \label{fig:cos_cen}
\end{minipage} ~

\end{figure}  

\textbf{Esto sucede tambi\'en con la distancia euclidea en logit}. En la Figura
\ref{fig:euc_cen}, para $l>2$, se cumple que $p_1$ est\'a m\'as cerca del centroide
que $p_4$. Recordemos que los valores dentro de los vectores luego de ser 
transformados usando LogOdds dejan de est\'ar acotados.

\begin{figure}[h!]
                                                                                                                        
\begin{minipage}[b]{\textwidth}
    \includegraphics[width=\textwidth]{img/euclidean_centroid.png}
    \caption{Contraejemplo para linkage centroide con la distancia euclidea 
             luego de aplicar logit}
    \label{fig:euc_cen}
\end{minipage} ~

\end{figure}  

\subsection{Complejidad algor\'itmica}

Como ya explicamos, por cada iteraci\'on del algoritmo 
\textit{Agglomerative Herarchical Clustering} es necesario calcular un 
representante de la uni\'on y luego computar su distancia al resto. Sin embargo, 
al usar la m\'etrica euclideana junto con el \textit{linkage} centroide es posible
simplificar este paso. La formula de Lance y Williams permite computar las nuevas
distancias sin comparar explicitamente los clusters. Esto baja significativamente
la complejidad. Cada iteraci\'on pasa a costar $O(c^2)$ en vez de $O(c^2 m)$,
siendo $c$ la cantidad de clusters y $m$ la longitud de los mismos. La complejidad
temporal total del \textit{clustering} baja a $O(n^3)$ contra $O(n^3 m)$, siendo
$n$ la cantidad de semillas. Recordemos que en el contexto que estamos utilizando
este algoritmo $m>>n$.  





