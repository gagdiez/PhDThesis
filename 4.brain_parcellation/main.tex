\chapter{Mapping the Brain: A review of the brain divisions}

\section{Overview}
The brain is composed of billions of neurons interacting at the same time
between them. Given the complexity of this network, a dimensionality reduction
is needed in order to study its properties. Since its beginning, neuroscientist
have divided the brain based on different criteria. This allows to study the
brain as a set of interacting regions, allowing to abstract the underlying
neuronal complexity. However, it's not clear that a unique and truth division
of the human brain exists. In this chapter, we make a review of the different
type of parcellations that exist; explain their advantages, and the best
scenarios where to use each.

\section{Introduction}
Composed by billions of interconnected neurons, the brain is the most complex
biological machine that we know. Untangling this network is of special importance
to understand the underlying process of concious life. However, studying the 
brain at the cellular level, alongside their interactions is simply not feasible.
Existent techniques to study brain microstructure are highly invasive, and can
only be used in postmortem tissue [cite]. Furthermore, while this techniques
succefully help to characterize the cellular composition of the tissue, they
do not allow to account for the interaction between cells. Specially for whole
brain data, the obtained dataset are simply too huge to be handled [cite].
Because of all of this, some way to reduce dimentionallity is needed.

Thanks to the study of cythoarquitectonics, brain function, and tracers we know
that neighboring neurons tend to work together [many cites]. In particular,
the brain behaves as a small world graph. While which populations
of neurons, and how they interact between is highly subject dependant, this
studies give us a solid support to subdivide the brain before studying it.
Nowadays, subdividing the brain in regions homogeneous respect to some criteria
is a necessary first step of every study in neuroscience. By doing so, it's not
only possible to reduce the dimensionality of the network, but also to derive
rules more general than in the cellular case. Furthermore, having consistent
divisions across subjects allows to infer properties from a population. The 
problem recides in the fact that there's not a gold standard/unique division of
the human brain. How to divide the brain will heavily depend on the task desired
to achieve, since using the wrong parcellation can introduce a bias in the
results. 

Historically, the brain has been divided following different criteria. Furthermore,
with every new technological advance, new parcellations arise. Perhaps the first
parcellation to exist was the anatomical one. Still something from Demians thesis
here. Based solely in the brain morphology, gross divisions of the brain can be
made, as lobes and etc. Then, through the study of brain lesions, brain surgery
and EEG allowed to study brain function, from which gross functional divisions
arise. A couple of years later, Brodmann creates his map based on the
cythoarchitecture. Enter MRI to improve anatomical divisions, PET/fMRI for functional
and more recently dMRI for structural. Then, semantical, and finally, the
multischeme ones.

In this chapter, we make a review of the most important techniques of which
we know about for each modality, giving a brief introduction to how they work
and their main strenghts. All the reviewed parcellations are comprised in the
table X. The chapter is divided in the following: Introduction, Parcellations
and Discussion.

\section{All}
where I published mine:
https://www.sciencedirect.com/science/article/pii/S1053811917310091

Surface-Based and Probabilistic Atlases of Primate Cerebral Cortex
David, in Mendeley.

\section{Anatomical}
The regions on an Anatomical divisions of the brain are defined by their shape
or relative position in the brain. For example, in the Desikan atlas, the
pars opercularis region is defined as 'the first gyrus from the precentral gyrus'.
The division of the brain in lobes is a coarse anatomical parcellation, finer
divisions can be created based on the sulci. AAL is one of the oldest one,
check this. One of the most used anatomical atlases is that of Desikan~\cite{Desikan2006}.
The desikan atlas divides each in 34 coarse structures (Fig X), and each
area is delimited mostly by sulcus. Desikan atlas is built on the cortical surface by: projecting the cortical folding
patters into a spherical surface, aligning them with an atlas computed from the
average of 40 subjects, and then transfering the labels from the atlas to the
cortical surface.
This map was further refined by Destrieux. Destrieux works in a similar way,
but makes further subdivisions of giry and stuff. AAL is older than them?
Anyway, no one recomends AAL.

https://biomedia.doc.ic.ac.uk/brain-parcellation-survey/


\section{Cythoarquitecture}
\label{sec:cyto_maps}
Only the borders of a few architectonically defined areas show a sufficiently precise association with sulci~\cite{Amunts2007}
As shown with modern techniques, borders between cytoarchitectonic areas are functionally relevant, to the point that many functional regions are refered using cythoarquitectonic labels.

Cythoarquitectonic divisions of the brain are based solely in the cellular
composition of the cortex. In this atlases, a brain region possess a consistent
thickness and cellular organization. For example, Brodmann Area 4 has an unusually
thick cortex; possess giant pyramidal cells, and lacks an internal and external
granular layer. The pioneers in the area are Meynert [CITE4-Thiarhou],
Campbell [CITE5] who divided the cortex into 14 areas, and Elliot Smith [CITE6]
into 50.

The most known cythoarchitectonic division of the brain is that
of Broadmann~\cite{Brodmann1909}. Broadmann defines 52 cortical regions, based
on the inspection trough microscope of cortical sections on post-mortem
primates brains. In 1925 von Economo and Koskinas publish the 'Atlas of 
Cytoarchitectonics of the Adult Human Cerebral Cortex' [CITE]. Economo and
Koskinas [cite] recognized 54 fundamental cytoarchitectonic areas with 76
variants and 107 modifications~\cite{Triarhou2007}. Their atlas is based in the
examination of mentally healthy subjects in the range of 30–40 years of age
through improved microscopes. More recent work includes the study of
neurotransmitter receptors to map divisions in the visual cortex~\cite{Eickhoff2008},
and the introduction of microstructural metrics, as the gray level index
[CITE](Schleicher and Zilles, 1990), to create observer-independent parcellation
methods.

Extant cytoarchitectonic maps do not cover the complete cerebral cortex. 
Only approximately 40\% of the cortical surface has been mapped up to now. 
This is caused by the difficulties of: studying sections; 3D reconstructing the
histological sections, and registration of post-mortem data to a reference space.
This is a very time-consuming process, requiring as minimum as 1 person year for
an area.


\section{Functional}
Functional parcellations of the brain denotes how different human functions 
are localized in the brain. This is, each region is in charge of a specific
function (i.e. movement, speach, vision), or is part of a greater system with
a specific functional or cognitive goal. The first functional maps where derived
from lessions. A classical example is that of Broca and Wenicke areas, related
to meaningfull speach production and spech comprehension. The areas are named
after Broca [CITE] and Wernicke [CITE], who in the late 1800 reported lesions in
that region for aphasic patients. Another example is the human homunculus,
a representation of motor and sensory functions, which Penfied [cite] mapped
trough experimenting with electrical stimulation of different brain areas of
patients undergoing open brain surgery.

As stated in the previous section (sec. \ref{sec:cytho_maps}, Cythoarchitectonic
parcellations can be seen as functional ones, since they show close relation 
to function. However, they do not cover the whole brain, and we can only relate
them after many patients show specific impediments and have lesions in those
regions. The advenment of functional MRI (fMRI) allowed to measure the level of
oxygen in blood. Knowing that the neurons need oxygen as fuel to fire, we can
find wich region of the brain are being activated for specific tasks. This
allowed to refine and validate in vivo the functional specialization of 
cythoarchitectural parcels and to better define anatomical ones as the human
humunculus \cite{Lashkari2010, Michel2011}. Even more interesting is the fact
that there are some regions that activate while in a resting state. From the
no-task state (resting state), is possible to derive parcellations by seing
which systems activate simultaneusly. While we have no idea what this means,
people uses it. By applying different clustering algorithms to this matrix, we
obtain different parcellations. We can use ward clustering [cites from Thirion];
k-means clustering [cites from thirion]; spectral clustering [cites from thirion],
and principal component analysis (PCA) [cites from thirion]. There are also
local gradient approaches, that detect abrupt transitions in functional
connectivity patterns \cite{Wig2014, Schaefer2017}.

add this one \cite{Craddock2011}.

add https://www.sciencedirect.com/science/article/pii/S1053811913002668.

Good review in:
https://www.biorxiv.org/content/biorxiv/early/2017/06/06/135632.full.pdf

https://biomedia.doc.ic.ac.uk/brain-parcellation-survey/

\subsection{EEG}
No idea... i have to read it.
https://www.sciencedirect.com/science/article/pii/S1053811917307474

\section{Brain Networks}
Vinod, brain as a system
The lobes and subcortical structures do not function in isolation, in fact they are heavily connected through the fibre bundles which compose the white matter.

[VINOD]
In his view, the human brain contains at least five major core functional networks: (i) a spatial attention network anchored in posterior parietal cortex and frontal eye fields;
(ii) a language network anchored in Wernicke’s and Broca’s areas; (iii) an explicit memory network anchored in the hippocampal–entorhinal complex and inferior parietal cor- tex; (iv) a face-object recognition network anchored in
midtemporal and temporopolar cortices; and (v) a working memory-executive function network anchored in prefron- tal and inferior parietal cortices. 


Gael (which fmri clustering... in mendeley)
https://www.frontiersin.org/articles/10.3389/fnins.2014.00167/full

\section{Semantic}
Semantic parcellations try to map semantic processing to the cortex.
In this maps, a region of the brain is ligated to a specific 'definition?/semantic'.

https://www.nature.com/articles/nature17637
% Mendeley \cite{Glover2011}
Where Is the Semantic System? A Critical Review and Meta-Analysis of 120 Functional Neuroimaging Studies
% In mendeley also

\section{Structural Connectivity}
In a structural parcellation, regions have a homogeneous pattern of connectivity
with the rest of the brain. The first maps come from tracers in monkey. The new
ones come from tractography. Yeah, we know that tractography is not perfect,
but we cannot torture any more monkeys. We name all of the principal players.
Gallardo.
Moreno-Dominguez.
Alfred Awander PCA.
Michel Thiebaut.
Parisot. 
https://arxiv.org/abs/1610.03809

\section{Multimodal}
Finally, more and more people is doing this. Maybe a precursor of this is
anatomy+cytoarchitecture. But here we talk about really different modalities,
as T1+T2+fMRI+goodknowswhat. We present at least two. I'm sure there are more.
Parisot.
Van Essen.
Braintome\cite{Fan2016}
...?

\section{Discussion}
The brain atlas concordance problem: quantitative comparison of anatomical parcellations.

Here we name them, some people profile them:

https://www.frontiersin.org/articles/10.3389/fnins.2014.00167/full % \cite{Thirion2014}
Chris (the guy from stanford, that we met in many conferences) 

There's not such a thing as a unique brain parcellation. While we cannot say
that it doesn't exist. We know that all of our techniques have a limitation,
either in resolution, SNR. At the same time, we cannot study things at the
neuron level, because we have billions of neurons in the brain. It makes a
lot of sense that each parcellation is different, because they're based
in different criteria. Even when trying to parcellate data coming from a
same machine (fmri, dmri, etc), changing the hypothesis will change the 
resulting parcellation. For example, in structural gradient vs structural
parcel. The important thing is to be able to select the right parcellation
for the right study.

\section{Resume of all of the parcellations presented}
Here I put a beautiful table resuming everything.

\section{Conclusion}
In order to reduce the complexity of the brain many studies start by subdividing
the brain. Depending on the hypothesis driving the study, different type
of parcellations can be used. In this chapter we presented parcellations based
on different criteria: cythoarquitecture; anatomy, function, semantics and
structural. We also presented some parcellation which are driven by a mixture
of suche criteria. Each parcellation possess its own advantages, and should be
used in the right context. In the following chapter we will introduce the
first contribution of this thesis: a technique to parcellate the cortex based
on its structural connectivity. Our technique allows to create parcellations
at the single-subject and group level, while having a good correlation with
known divisions of the brain.


\chapterbib
