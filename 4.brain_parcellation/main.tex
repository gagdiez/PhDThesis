\chapter{Mapping the Brain: A review of the brain divisions}

\section{Overview}
Neuroscientist have long thought of the brain as a mosaic of interconnected
regions. This simplification allows to abstract the complex neuronal network
that underlies cognition. Until today, it's not clear that an unique and truth
division of the human brain exists. This is also true even when divisions are 
constrained to only one modality: anatomy, structure or function. In this
chapter, we make a review of the different type of parcellations that exist;
explain their advantages, and the best scenarios where to use each.

%The brain is a higly complex network composed by billions of neurons interacting at the same time
%between them. Given the complexity of this network, a dimensionality reduction
%is needed in order to study its properties. Since its beginning, neuroscientist
%have divided the brain based on different criteria. This allows to study the
%brain as a set of interacting regions, allowing to abstract the underlying
%neuronal complexity. 

\section{Introduction}
Composed by billions of interconnected neurons, the brain is the most complex
biological machine that we know. Untangling this network is of special importance
to understand the process underlying concious life, a task that has proven to be
arduous. In fact, with nowadays technology, is simply not feasible
to study the whole neural network at a cellular level. Extant techniques to
study brain microstructure are highly invasive, and can only be used in
postmortem brain~\cite{Swanson1982, Schmahmann2006, Amunts2007, Ding2016}.
While useful to quantify the cellular composition of the tissue, these techniques
do not allow to study neuronal communication (synapses) nor neuronal connectivity.
Even when whole brain dissections are made \cite{Amunts2013, Ding2016}, the size
of the acquired data, and the damage induced in the tissue while dissecting it,
makes hard to coregister and study continuity between slices. These problems
highlight the need to compress information from the brain in a coarser network.

Studies in cytoarchitecture\cite{Brodmann1909, VonEconomo1925}, brain
function\cite{Penfield1954, VonderMalsburg1994}, and tracers~\cite{Schmahmann2006, Stephan2013}
show that neurons tend to organize and activate in spatially coherent groups.
These studies give us evidence to support subdividing the brain before studying
it. By doing so, it's not only possible to reduce the dimensionality of the
cellular network, but also to derive rules for population of neurons. Furthermore,
having consistent divisions across subjects allows to infer properties about the
human brain in general. Then the problem is, how to subdivide the brain. Until
today, there's not a gold standard division of the human brain. Moreover, how
to divide the brain depends heavily on the modality being use to study it.

%Historically, the brain has been divided following different criteria such as
%anatomical landmarks, or cellular composition. With every new technological
%advance, new divisions arise. Perhaps the first type of parcellations to exist
%were the anatomical ones. Already in the year 3,000 BCE, an Egyptian papyrus
%references anatomical landmarks in the head when talking about brain 
%lesions\cite{Elsberg1945}. During the 17th century, studies in postmortem
%brains helped to further identify the lobes, gyrus and sulci~\cite{Collice2008}.
%At the end of the 1800, Broca and Wernicke define regions in the brain related
%to speach understanding and production, making one of the first functional 
%divisions that survives until today. In 1905 Campbell~\cite{Campbell1905} 
%presents a cortical parcellation of the brain in 14 regions based on its cellular
%composition, this is the first cytoarchitectural division of the brain. Four
%years later Brodmann presents its seminal cytoarchitectural~\cite{Brodmann1909} division of the
%brain which is still used until today. In 1954 Penfield~\cite{Penfield1954}
%maps motor and sensory functions to the cortex and shows the consistency across
%different subjects, defining a functional atlas over the precentral and poscentral
%gyrus. something about tracers, maybe In 1982 Swanson et al.~\cite{Swanson1982} started 

%The advent of Magnetic Resonance Imaging (MRI) allowed to rapidly
%progress the field of brain mapping. MRI permitted to acquire massive amount
%of brain images and create brain templates. By computing brain templates neuroscientist
%were able to define a single brain image that is representative of the population~\cite{Talairach1988, Collins1998, Holmes1998}.
%Since the regions defined on these templates should also be representative,
%the parcellations defined in templates are considered atlases. functional MRI (fMRI)
%did the same for mapping function in the brain. idem dMRI. And finally, the
%multischeme ones.

Historically, the brain has been divided following different criteria such as
anatomical landmarks, functional localization, axonal connectivity or
cellular composition. A valid question is, why are there different parcellations?.
Why not just use one?. The answer is because each type of parcellation incurs
on a trade off. Based on brain landmarks, Anatomical parcellations are highly
reproducible across subjects. This is, once the rules to define a parcel are
set, they are easy to automatize. However, this regions tend to be coarse,
and do not account correctly for brain function. Cytoarchitectural divisions
of the brain are functionally relevant, to the point that many functional
regions are referred using cytoarchitectonic labels. But, in terms of location,
the borders of only a few cytoarchitectonically defined areas show a 
sufficiently precise association with sulci~\cite{Amunts2007}. This highlights
the need of cytoarchitectonic parcellations. The downside, is that the study
of cellular composition can only be done in post-mortem brains. At the same time,
extant cytoarchitectonic maps do not cover the complete cerebral cortex. 
Only approximately 40\% of the cortical surface has been mapped up to now. 
This is caused by the difficulties of: studying sections; 3D reconstructing the
histological sections, and registration of post-mortem data to a reference space.
This is a very time-consuming process, requiring as minimum as 1 person year for
an area. Functional parcellations are inferred based in relative differences of
some sort, generally a region is defined as functionally specialized because:
it has a lesion and that function was affected; or it shows more activation 
while the subject is doing the function. The problem of this approach, known
as the modular paradigm, is that it does not allow to map higher cognitive
functions to the brain~\cite{Fuster2000}. As proposed by Bressler and Menon~\cite{Bressler2010},
is necessary to study the brain as a network, taking into account axonal
connectivity. Structural parcellations of the brain define regions with homogeneous
connectivity. While many studies show that the obtained parcels are functionally
relevant, the underlying technique, tractography, is not trusted.

In recent years we witness a rapidly growing in the field
of brain mapping, driven by advances in Magnetic Resonance Imaging (MRI) and
computational power. Computational
power gave us better imaging in cytoarchitecture. MRI, fMRI and
dMRI gave us non-invasive ways to permitted to acquire massive amount
of brain images. This opened the possibility of doing studies with big populations.
Computational power let us use more complex methods. 




In this chapter, we make a
review of the most important techniques of which we know about for each modality,
giving a brief introduction to how they work and their main strenghts. All the
reviewed parcellations are comprised in the table X. 




\section{Anatomical}
On an anatomical parcellation, a region is characterize by its shape or its
relative position in the brain. For example, in the Desikan atlas, the
pars opercularis region is defined as 'the first gyrus from the precentral gyrus'.

The first anatomical atlases of the brain where solely based on the study of
post-mortem brains. The division of the brain in lobes can be see as a coarse
atlas of the cerebral cortex. A notable example of post-mortem atlas is the 
Talairach Atlas~\cite{Talairach1988}, which is based on the dissection of one
human brain, and defines a standarized system of coordinates for neurosurgery.

In moderns times, the advent of MRI permitted to acquire massive amount
of brain images and create brain templates. A brain template is an image
which is a representative of the different brains on a population. Two
examples of this are the Colin27 atlas~\cite{Collins1998}, based on the average
of 27 healthy subjects images, and the Montreal Neurological Institute (MNI) brain~\cite{Holmes1998},
which latest version (ICBM152) is based on the average of 152 healthy subjects images.
The parcellation of a brain templates is, by definition, anatomically
consistent across subjects, and therefore is considered as atlases. The
most used anatomical atlas, AAL~\cite{Landeau2002}, is based on the manual
parcellation of the MNI brain template. It divides it in 45 anatomical
volumes of interest.

When working only on the cortical surface, the most used atlas is that of
Desikan~\cite{Desikan2006}. The atlas is based on the segmentation of a surface
template, generated from the average cortical folding of 40 healthy subjects
projected on a sphere. The template was manually segmented into 34 coarse 
structures per hemisphere. In order to label a new subject, they cortical 
folding is projected into the sphere and aligned to the template, then 
the labels are mapped from the atlas to the cortical surface. The automatic
labeling method developped by Desikan shows a great accuracy when labeling
new subjects as shown in their paper (rand index > 0.8). The methodology
of Destrieux~\cite{Destrieux2010} presents a finer division of the cortex in
74 labels per hemisphere. In the case of Destrieux, not only the cortical
folding is taken into account in order to label a region. Indeed, the
underlying probabilistic model takes into account information as: curvature
and average convexity of the cortica surface, prior labeling probability for
that vertex, as well as the labels of vertices in a local neighborhood.

The MarsAtlas~\cite{Auzias2016} uses the superior temporal and inferior frontal
sulcus as orthogonal axis to defines a grill over the cortex. The rest of the
sulcus in the brain are aligned to this grill, and used to divide the cortical
surface in 41 regions. They show that their parcels have a good correspondence
with specific functional activations.


\section{Cytoarchitecture}
\label{sec:cyto_maps}

Cytoarchitectonic divisions of the brain are based solely in the cellular
composition of the cortex. In this atlases, a brain region possess a consistent
thickness and cellular organization. For example, Brodmann Area 4 has an unusually
thick cortex; possess giant pyramidal cells, and lacks an internal and external
granular layer. The pioneers in the area are Meynert~\cite{Meynert1872} who
who noted regional differences in cellular structure among different areas of the human cerebral cortex,
Campbell~\cite{Campbell1905} who divided the cortex into 14 areas, and Elliot Smith\cite{Smith1907}
into 50.

The most known cytoarchitectonic division of the brain is that
of Broadmann~\cite{Brodmann1909}. Broadmann defines 52 cortical regions, based
on the inspection trough microscope of cortical sections on post-mortem
primates brains. In 1925 von Economo and Koskinas publish the 'Atlas of 
Cytoarchitectonics of the Adult Human Cerebral Cortex' [CITE]. Economo and
Koskinas [cite] recognized 54 fundamental cytoarchitectonic areas with 76
variants and 107 modifications~\cite{Triarhou2007}. Their atlas is based in the
examination of mentally healthy subjects in the range of 30–40 years of age
through improved microscopes. We also have the Jubrain~\cite{Mohlberg2012}, a
cytoarchitectonic probabilistic maps viever, made by Juelich and Duesseldorf
institutes by analizing the histological sections of ten human postmortem brains.
More recent work includes the study of
neurotransmitter receptors to map divisions in the visual cortex~\cite{Eickhoff2008}, 
and the introduction of microstructural metrics, as the gray level index
[CITE](Schleicher and Zilles, 1990), to create observer-independent parcellation
methods, 

Creates an atlas of the human ventral visual stream from the postmortem brain
of 11 subjects\cite{Rosenke2018}.
    


\section{Functional}
Functional parcellations of the brain denote how different human functions 
are localized in the brain. This is, each region is in charge of a specific
function (i.e. movement, speach, vision), or is part of a greater system with
a specific functional or cognitive goal. The first functional maps where derived
from lessions. A classical example is that of Broca and Wenicke areas, related
to meaningfull speach production and spech comprehension. The areas are named
after Broca and Wernicke, who in the late 1800 reported lesions in
that region for aphasic patients~\cite{Johns}. Another example is the human homunculus,
a representation of motor and sensory functions, which Penfied [cite] mapped
trough experimenting with electrical stimulation of different brain areas of
patients undergoing open brain surgery.

As stated in the previous section (sec. \ref{sec:cyto_maps}, Cythoarchitectonic
parcellations can be seen as functional ones, since they show close relation 
to function. However, they do not cover the whole brain, and we can only relate
them after many patients show specific impediments and have lesions in those
regions. The advenment of functional MRI (fMRI) allowed to measure the level of
oxygen in blood. Knowing that the neurons need oxygen as fuel to fire, we can
find wich region of the brain are being activated for specific tasks. This
allowed to refine and validate in vivo the functional specialization of 
cytoarchitectural parcels and to better define anatomical ones as the human
humunculus \cite{Lashkari2010, Michel2011}. Even more interesting is the fact
that there are some regions that activate while in a resting state. From the
no-task state (resting state), is possible to derive parcellations by seing
which systems activate simultaneusly. The most common way is to computed the
Pearson's correlation between the fMRI time series at each spatial location.
While we have no idea what this means, people uses it. 

PNAS of 2004 \cite{Johansen-Berg2004}


%\cite{Power2012} use Graph analysis and subgraph detection techniques in order
%to characterize functional subnetworks in the connectivity graph. 
Yeo et al. [CITE] 
Data from 1,000 subjects were registered using surface-based alignment.
The connectivity fingerprints were modeled with a von Mises-Fisher distribution,
then, the data points were randomly assigned to different groups and then
iteratively reassigning the group memberships of points to maximize the
agreement of connectivity profiles among points of the same group. The results
yield a 7 parcels and 17 parcels network.

The Cortical Area Parcellation from Resting-State Correlations dataset consists
of 333 cortical patches segmented using resting-state fMRI (Gordon 2014).

By applying different clustering algorithms to this matrix, we
obtain different parcellations. We can use ward clustering [cites from Thirion];
k-means clustering [cites from thirion]; spectral clustering [cites from thirion],
and principal component analysis (PCA) [cites from thirion]. There are also
local gradient approaches, that detect abrupt transitions in functional
connectivity patterns \cite{Wig2014, Schaefer2017}. 
Based on the work by Thomas Yeo et al. [CITE], Schaefer et al.  \cite{Schaefer2017}
add further refinement by subparcellating the global networks based on a local
gradient approach. Parcellations come in several version, breaking down the cortex
into up to 1000 regions based on rs-fMRI.

Presents a multigraph K-way clustering method and applies it to obtain 
parcellations of 50, 100, 150... maybe put in the same bag as the rest \cite{Shen2013}.


A fine-grained parcellations ranging form 200 to 1000 parcellations,
using spectral clustering on rs-fMRI connectivity matrix \cite{Craddock2011}.

Thomas Blumensath et al. makes a thousand small functionally homogeneous regions,
and then apply hierarchical clustering to obtain the rest of the parcellations \cite{Blumensath2013}.

Huth et al. atlas \cite{Huth2016} maps semantic selectivity across the cortex
by using voxel-wise modelling of functional MRI (fMRI) data collected while 
subjects listened to hours of narrative stories. This creates a map where each
region is ligated to a specific systematically map semantic domain. In this
maps, a region of the brain is ligated to a specific semantic domain, i.e.
violent, temporal, proffesional \cite{Huth2016}.

While we dont do EEG, you can review it here \cite{Shen2013}.

add \cite{Ryali2013} https://www.ncbi.nlm.nih.gov/pubmed/23041530

%Good review in:
%https://www.biorxiv.org/content/biorxiv/early/2017/06/06/135632.full.pdf
%https://biomedia.doc.ic.ac.uk/brain-parcellation-survey/


\section{Structural Connectivity}
In a structural parcellation, regions have a homogeneous pattern of connectivity
with the rest of the brain. This is, a region is defined as: 'all of the points
inside are connected to some other ROIs in a similar way'. The granularity
of the ROIs can go from coarse brain regions, defined by other atlas, to voxels.
The first connectivity divisions came form tracers in monkey PANDIAS. Diffusion
MRI (dMRI) enables the in vivo exploration of extrinsic connectivity on the
human brain. As with fMRI, the most common way to generate a parcellation is
to compute a connectivity matrix, and then parcellate it using some clustering
technique. Given the difficulty of handling huge connectivity matrices, initial
techniques used to divide only portions of the brain.

Behrens et al.\cite{Behrens2003} thalamus Hard segmentation was performed by classifying the seed voxel as connecting to the cortical mask with the highest con- nection probability.
Alfred Anwander performs k-means\cite{Anwander2006} in the connectivity matrix of Broca's area.
Thiebaut de Schotten et al. divide the occipital cortex\cite{ThiebautdeSchotten2014}, frontal lobe \cite{ThiebautdeSchotten2016}

Braintome\cite{Fan2016} starting from Desikan, they subdivide each region based on tractography.
Moreno-Dominguez et al. use a constrained hierarchical clustering to parcellate a structural matrix \cite{Moreno-Dominguez2014}
Gallardo buils on top of this.
They use spectral reordering to study the organization of the temporal lobe \cite{Bajada2017}
(O'Muircheartaigh and Jbabdi, 2017) is entirely data-driven (based on independent component analysis) and allows one to obtain sets of white matter components with their associated gray matter networks \cite{Muircheartaigh2018}


Previous single-subject structural parcelling techniques work by refining other
parcellations~\cite{Clarkson2010};parcelling only part of the
cortex~\cite{Lefranc2016, Roca2009, ThiebautdeSchotten2014, ThiebautdeSchotten2016},
perform a groupwise parcellation after constructing an average connectivity profile~\cite{Clarkson2010, Roca2010},

Parisot uses ideas from cosegmentation and spectral decomposition to create consistent groupwise parcellations \cite{Paristot2015}.


\section{Multi-modal}
So far, all the presented methodologies were solely on one technique. In multi-modal parcellations, the goal is to mix two or more.
The multi-modal parcellations are, by construction, maximizing the consistency between modalities. For example,
Diez et al. \cite{Diez2014} proceed by computing common structure-function modules (SFMs). They start by computing both a resting state and a structural connectivity matrix. Then, they find the subdivision that maximizes a cross-modularity index, that, when high, indicates that, using a given common partition, both matrices are highly modular and, at the same time, the moduli are internally wired in a similar way.
Parisot et al. \cite{Parisot2017} compute a set of parcellations from different modalities and fuse them based on their local reliabilities. The fused parcellation is used to initialise the next iteration, forcing the parcellations to converge towards a set of mutually informed modality specific parcellations.
Glasser et al. \cite{Glasser2016}. Architecture was measured using T1w/T2w myelin content maps plus cortical thickness maps with surface curvature regressed out5,9,10 (Supplementary Methods 1.5). Function was measured using task-fMRI responses to 7 tasks in 86 task contrasts (47 unique; 39 were sign-reversed contrasts). Effect size maps (beta maps) after correction for the receive field were used instead of Z statistic maps because we were interested in regional differences in the magnitude of the BOLD (blood oxygen level dependent) signal change induced by the tasks, rather than differences in the significance of the BOLD signal change. Neuroanatomist defined initial border areas in a semi-automated way using group average data for all of these modalities, then an automated algorithm optimized the border placement, based on areal features of each subject.

\section{Discussion}
The brain atlas concordance problem: quantitative comparison of anatomical parcellations \cite{Bohland2009}.


Here we name them, some people profile them:

Thirion did it for functionali \cite{Thirion2014}
Arslan did it for functional \cite{Arslan2018}
Gorgolewski KJ, Tambini A, Durnez J et al. Evaluation of full brain parcellation schemes using the NeuroVault database of statistical maps [version 1; not peer reviewed]. F1000Research 2017, 6:1986 (poster) (doi: 10.7490/f1000research.1115065.1) 

Some other reviews: Jbabdi \cite{Jbabdi2013}, \cite{Arslan2018}, 

Gatherings: http://www.lead-dbs.org/helpsupport/knowledge-base/atlasesresources/cortical-atlas-parcellations-mni-space
            https://fsl.fmrib.ox.ac.uk/fsl/fslwiki/Atlases


There's not such a thing as a unique brain parcellation. While we cannot say
that it doesn't exist. We know that all of our techniques have a limitation,
either in resolution, SNR. At the same time, we cannot study things at the
neuron level, because we have billions of neurons in the brain. It makes a
lot of sense that each parcellation is different, because they're based
in different criteria. Even when trying to parcellate data coming from a
same machine (fmri, dmri, etc), changing the hypothesis will change the 
resulting parcellation. For example, in structural gradient vs structural
parcel. The important thing is to be able to select the right parcellation
for the right study.

The important thing is to validate it correctly, some people do, like \cite{Gallardo}, \cite{Auzias2016}, \cite{ThiebautdeSchotten2014, ThiebautdeSchotten2016}

\section{Resume of all of the parcellations presented}
Here I put a beautiful table resuming everything.

\section{Conclusion}
In order to reduce the complexity of the brain many studies start by subdividing
the brain. Depending on the hypothesis driving the study, different type
of parcellations can be used. In this chapter we presented parcellations based
on different criteria: cytoarchitecture; anatomy, function, semantics and
structural. We also presented some parcellation which are driven by a mixture
of suche criteria. Each parcellation possess its own advantages, and should be
used in the right context. In the following chapter we will introduce the
first contribution of this thesis: a technique to parcellate the cortex based
on its structural connectivity. Our technique allows to create parcellations
at the single-subject and group level, while having a good correlation with
known divisions of the brain.


\chapterbib
