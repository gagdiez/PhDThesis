\chapter{Mapping the Brain: A review of the brain divisions}

\section{Overview}
Neuroscientist have long thought of the brain as a mosaic of interconnected
regions. This simplification allows to abstract the complex neuronal network
that underlies cognition. Until today, it's not clear that an unique and true
division of the human brain exists. Different modalities have been used to
study the brain, deriving in different parcellations. In this chapter, we make
a review of the different criteria to divide the brain; the hypothesis in
which they are based, and present the most notable parcellations of each.


\section{Introduction}
Composed by billions of interconnected neurons, the brain is the most complex
biological machine that we know. Untangling this network is of special importance
to understand the process underlying concious life, a task that has proven to be
arduous. In fact, with nowadays technology, is simply not feasible
to study the whole human neural network at cellular level. Extant techniques to
study brain microstructure are highly invasive, and can only be used in
postmortem brain~\cite{Swanson1982, Schmahmann2006, Amunts2007, Ding2016}.
While useful to quantify the cellular composition of the tissue, these techniques
do not allow to study neuronal communication (synapses) nor long range axonal
connectivity. Even when whole brain dissections are made at super high resolution\cite{Mohlberg2012, Amunts2013, Ding2016},
the size of the acquired data, and the damage induced in the tissue while dissecting it,
makes hard to coregister and study continuity between slices. These problems
highlight the need to compress information from the brain in a coarser network.


% Say in the next paragraph that we want to:
% Characterize groups of neurons inside of a subject, to study their properties
% Characterize same regions across subjects, to infer properties about the human brain
Studies in cytoarchitecture\cite{Meynert1872, Brodmann1909, VonEconomo1925}, brain
function\cite{Penfield1954, VonderMalsburg1994}, and tracers~\cite{Schmahmann2006, Stephan2013}
show that neurons tend to organize and activate in spatially coherent groups.
This provides us with a biological basis for dividing the brain into spatially coherent
parcels. By doing so, it's not only possible to reduce the dimensionality of the
cellular network, but also to derive common properties for population of neurons.
Furthermore, having consistent divisions across subjects allows to infer
properties about the human brain in general. Then the problem is, how to
subdivide the brain. Until today, there's not a gold standard division of the
human brain. Moreover, how to divide the brain depends heavily on the modality
being use to study it.

\begin{figure}[h!]
    \includegraphics[width=0.49\textwidth,height=550px]{missing}
    \caption{Anatomical Parcellation of the Brain}
    \label{fig:brain_function}
\end{figure}

\begin{figure*}[t!]
    \includegraphics[width=\textwidth,height=150px]{missing}
    \caption{Cytoarchitectural Parcellation of the Brain}
    \label{fig:brain_function}
\end{figure*}

Historically, the brain has been divided following criteria such as its anatomy,
its activation during cognitive tasks, the cellular composition, and its connectivity.
On an anatomical parcellation, a region is characterize by its shape or its
relative position in the brain. For example, in the Desikan atlas,
the pars opercularis region is defined as 'the first gyrus from the precentral
gyrus'~\cite{Desikan2006}. Anatomical parcellations were most probably the first way to subdivide
the brain in regions, since references to anatomical landmarks can be found
in ancient texts~\cite{Elsberg1945, Collice2008}. Functional parcellations
map cognitive function to regions in the brain. In this type of parcellation,
each parcel is characterized by a cognitive function. A classical example is that
of Broca and Wernicke areas, related to meaningful speech production and speech
comprehension. While the first functional maps were infered by studying cognitive
changes following brain injuries, modern ones are derived from neuroimaging.
Cytoarchitectural divisions of the brain are
based solely in the cellular composition of the cortex. In these atlases, a region
is characterize by its thickness and cellular organization. For example,
Brodmann Area 4 has 'an unusually thick cortex; possess giant pyramidal cells,
and lacks both its internal and external granular layer'~\cite{Brodmann1909}.
Currently, all extant cytoarchitectonic maps are derived from destructive
post-mortem brain analysis.
Finally, structural parcellations are based on patters of axonal connectivity. Parcels have an
homogeneous pattern of connectivity respect to a predefined set of regions of
interest. For example, area MFG-5 of the Brainnetome atlas has 'connections 
with the major frontal subregions, the limbic area, the parietal subregions
and the subcortical connections with the thalamus and basal ganglia subregions'\cite{Fan2016}.

In recent years we witness a rapidly growing in the field
of brain mapping, driven by advances in Magnetic Resonance Imaging (MRI) and
computational power. In this chapter, we make a review of the most important
parcellation of which we know about for each modality, giving a brief introduction
to the their methodologies. We limit ourselves to present the parcellations without
benchmarking them, for more information on benchmarking please refer to the
works of Thirion et al.~\cite{Thirion2014}, and Arslan et al.~\cite{Arslan2018}.
This review is by no means extensive, and more information can be found in
Amunts et al.\cite{Amunts2007}, Triarhou (2007)\cite{Triarhou2007} 
Jbabdi et al.~\cite{Jbabdi2013}, Arslan et al. \cite{Arslan2018},
de Reus et al.~\cite{DeReus2013}, and Eichhoff et al.~\cite{Eickhoff2015}.
 
\begin{figure*}[t!]
    \includegraphics[width=\textwidth,height=150px]{missing}
    \caption{Functional Parcellation of the Brain}
    \label{fig:brain_function}
\end{figure*}

\section{Anatomical Parcellations}
\label{sec:anatomical}
The first anatomical atlases of the brain where solely based on the study of
post-mortem brains. Most probably, the first anatomical division of the brain
is that of gyrus and sulci in the 17th century~\cite{Collice2008}. Post-mortem
segmentation are still in use today, the most notable exampe being the one
of Talairach~\cite{Talairach1988}. This atlas is based on the dissection of one
human brain, and defines a
standardized system of coordinates for neurosurgery.

In moderns times, the advent of MRI permitted to acquire massive amount
of brain images and create brain templates. A brain template is an image
which is a representative of the different brains on a population. Two
examples of this are the Colin27 atlas~\cite{Collins1998}, based on the average
of 27 healthy subjects images, and the Montreal Neurological Institute (MNI) brain~\cite{Holmes1998},
which latest version (ICBM152) is based on the average of 152 healthy subjects images.
The parcellation of a brain templates is, by definition, anatomically
consistent across subjects, and therefore is considered an atlases. The
most used anatomical atlas, AAL~\cite{Landeau2002}, is based on the manual
parcellation of the MNI brain template. It divides it in 45 anatomical
volumes of interest.

When working only on the cortical surface, the most used atlas is that of
Desikan~\cite{Desikan2006}. Desikan atlas is based on the segmentation of a surface
template, generated from the average cortical folding of 40 healthy subjects
projected on a sphere. The template was manually segmented into 34 coarse 
structures per hemisphere. In order to label a new subject, they cortical 
folding is projected into the sphere and aligned to the template, then 
the labels are mapped from the atlas to the cortical surface. The automatic
labeling method developed by Desikan shows a great accuracy when labeling
new subjects~\cite{Desikan2006}. Another parcellation made by 
Destrieux et al.~\cite{Destrieux2010} presents a finer division of the cortex in
74 labels per hemisphere. In the case of Destrieux, not only the cortical
folding is taken into account in order to label a region. Indeed, their
underlying probabilistic model takes into account information as: curvature
and average convexity of the cortical surface, prior labeling probability for
that vertex, as well as the labels of vertices in a local neighborhood.

A more modern atlas, the MarsAtlas~\cite{Auzias2016}, uses the superior temporal
and inferior frontal sulcus as orthogonal axis to defines a grill over the cortex.
The rest of the sulcus in the brain are aligned to this grill, and used to divide
the cortical surface in 41 regions. As shown in their paper, the resulting parcels
have good correspondence with some specific functional activations~\cite{Auzias2016}.

\begin{figure*}[t]
    \includegraphics[width=\textwidth,height=150px]{missing}
    \caption{Structural Parcellation of the Brain}
    \label{fig:brain_function}
\end{figure*}

\section{Cytoarchitecture}
\label{sec:cytoarchitecture}

Meynert~\cite{Meynert1872} is considered the first neuroanatomist who noted
regional differences in cellular composition among different areas of the
cerebral cortex. 30 years after, Campbell~\cite{Campbell1905} presented a
cortical parcellation composed of 14 areas, followed shortly after by Elliot
Smith\cite{Smith1907} with a parcellation in 50 areas.

The most known cytoarchitectonic division of the brain is that of
Broadmann~\cite{Brodmann1909}. Broadmann defines 52 cortical regions, based
on the inspection trough microscope of cortical sections on post-mortem
primates brains. In 1925 von Economo and Koskinas publish
the 'Atlas of Cytoarchitectonics of the Adult Human Cerebral Cortex'\cite{VonEconomo1925}.
von Economo and Koskinas recognized 54 fundamental cytoarchitectonic areas with
76 variants and 107 modifications~\cite{Triarhou2007}. Their atlas is based in the
examination of mentally healthy subjects in the range of 30–40 years of age
through improved dissection and acquisition methodologies. von Economo and 
Koskinas atlas is considered one of the most detailed and reproducible cytoarchitectonic
atlas available~\cite{Peden1947}.

More recently, Schleicher and Zilles~\cite{Schleicher1990} introduced a microstructural
metric, the gray level index, to create observer-independent parcellation methods.
Also, new divisions of anatomical defined regions have been defined, as the one by
Eickhoff et al.~\cite{Eickhoff2008}, who studied neurotransmitter receptors to
map divisions in the visual cortex, or Rosenke et al.~\cite{Rosenke2018} with
their atlas of the human ventral visual stream based on the analysis of 11
post-mortem brains. Finally, new atlases have been created, as the 
Jubrain~\cite{Mohlberg2012}, a cytoarchitectonic probabilistic map base of the
histological sections of ten post-mortem human brains; or the one by
Ding et al.~\cite{Ding2016}, based on the manually dissection and parcellation
of a 34 years old brain.

\section{Functional}
\label{sec:functional}
The first functional maps were derived from lesions. The language regions for
example, are named after Broca and Wernicke, who in the late 1800 reported
lesions in that region for aphasic patients~\cite{Johns}. Another example is
that of the human homunculus, a representation of motor and sensory functions,
which Penfied~\cite{Schleicher1990} mapped trough experimenting with electrical
stimulation of different brain areas of patients undergoing open brain surgery.

The advent of functional MRI (fMRI) allowed to measure the blood-oxygen level
dependant signal. Knowing that the neurons need oxygen as fuel to activate,
fMRI can help to characterize which region of the brain active during specific
cognitive tasks, or during rest (resting state fMRI, rs-fmri). Many functional parceling
techniques rely on what is known as a functional connectivity matrix, a two
dimensional matrix that quantifies the connection between a set of regions
in the brain. Regions can be as big as pre-defined brain region or as small as
a single voxel. The simplest way to quantify connection between two regions is
by means of the Pearson's correlation between the fMRI time series at each region.

Most parceling techniques work by applying clustering algorithms to the
functional connectivity matrix. The most popular techniques use 
mixture models~\cite{Lashkari2010, Ryali2012}; ward clustering~\cite{Blumensath2013}cites from Thirion];
k-means clustering~\cite{Yeo2011, Shen2013, Kahnt2012}; hierarchical clustering~\cite{Eickhoff2011, Michel2011};
spectral clustering~\cite{Thirion2006, Craddock2011, Schaefer2017}, and
boundary detection~\cite{Gordon2016, Wig2014, Schaefer2017}.

Some widely used whole-brain atlases are those of Yeo et al.~\cite{Yeo2011},
Power et al.~\cite{Power2011}, Craddock~\cite{Craddock2011}. 
Yeo et al.~\cite{Yeo2011} propose to use k-means clustering on the average
rs-fmri connectivity from 1000 subjects. The results yield two parcellations,
with 7 parcels and 17 parcels respectively, that show to be reproducible across
groups of subjects. Power et al.\cite{Power2011} use graph analysis and subgraph
detection techniques in order to characterize functional subnetworks in the
connectivity graph, creating a 264 regions atlas. Finally, Craddock et al.~\cite{Craddock2011}
use spectral clustering on rs-fmri connectivity to create fine-grained parcellations
ranging form 200 to 1000 parcellations.

Other papers worth mentioning are those of Deen et al.~\cite{Deen2011} and Hurt
et al.~\cite{Huth2016}. Deen et al.\cite{Deen2011} present a functional division
of the insula, that has been proved to be highly reproducible across subjects
and studies. Meanwhile, Huth et al. present an innovative atlas which maps semantic
domains (i.e. violent, temporal, professional) across the cortex~\cite{Huth2016}.
They use using voxel-wise modelling of functional MRI (fMRI) data collected while 
subjects listened to hours of narrative stories.


\section{Structural Connectivity}
\label{sec:structural}
Based on Diffusion MRI (dMRI), tractography enables the in vivo exploration of
axonal connectivity on the human brain. As with functional data, the most
common way to generate a parcellation is by first computing a connectivity matrix
between regions (in this case, a structural connectivity matrix), and then parcellate it using
some clustering technique. When computing the matrix, the simplest way to estimate
connectivity between two regions is by computing the proportion of streamlines
that pass through them, respect to the total number of streamlines created.

Given the difficulty of handling huge connectivity matrices, initial techniques
used to divide only portions of the brain\cite{Behrens2003, Jbabdi2009, Anwander2006, ThiebautdeSchotten2014, ThiebautdeSchotten2016, Bajada2017}.
Advances in computational power and clustering methodology allowed to create
full brain parcellations\cite{Roca2009, Clarkson2010, Lefranc2016, Fan2016, Moreno-Dominguez2014, Muircheartaigh2018, Lefranc2016, Paristot2015, Roca2009, Gallardo2017a, Muircheartaigh2018}. 
Notables work as those of Behrens et al.\cite{Behrens2003}, Anwander et al.~\cite{Anwander2006},
Thiebaut et al.~\cite{ThiebautdeSchotten2016}, Moreno-Dominguez et al.~\cite{Moreno-Dominguez2014},
Bajada et al~\cite{Bajada2017} and Fan et al.~\cite{Fan2016}.
Behrens et al.~\cite{Behrens2003} segments the thalamus by classifying the seed voxel as connecting to the cortical
mask with the highest connection probability. Anwander et al.~\cite{Anwander2006}
uses k-means clustering over the connectivity matrix of Broca's area, obtaining
a division in 3 regions, consistent with cytoarchitectural divisions.
Thiebaut de Schotten et al.~\cite{ThiebautdeSchotten2016}
parcellates the frontal lobe in 12 regions by means of principal components analysis,
and shows its reproducibility across subjects and datasets.
Moreno-Dominguez creates a hierarchical parcellation of the brain by using
ward clustering~\cite{Moreno-Dominguez2014}, allowing to obtain a parcellation
of the brain with different granularities. Fan et al.\cite{Fan2016} use
spectral clustering over connectivity data to subdivide regions of the Desikan
atlas~\cite{Desikan2006}, obtaining a parcellation with 210 cortical areas and
36 subcortical regions. Finally, Bajada et al.~\cite{Bajada2017} take a different
approach and use spectral reordering to create a soft parcellation over the
temporal lobe. This allows to have parcels that diffuse into each other, instead
of sharp boundaries dividing them.

The parcellation of Gallardo et al.\cite{Gallardo2017a} will be presented in
full detail in the following chapter.

\begin{figure*}[t]
    \includegraphics[width=\textwidth,height=150px]{missing}
    \caption{Multi-modal Parcellation of the Brain}
    \label{fig:brain_function}
\end{figure*}

\section{Multi-modal}
\label{sec:multimodal}
So far, all the presented methodologies were solely based on one criteria.
Multi-modal parcellations are relatively new in the field of brain mapping.
By combining information from different neuroimaging methodologies, multi-modal
techniques create regions which boundaries are consistent with multiple
independent neurobiological properties.

Examples of multi-modal parcellations are those of Diez et al. \cite{Diez2014},
Parisot et al. \cite{Parisot2017}. 
Diez et al. \cite{Diez2014} merges structural connectivity and functional
connectivity in order to compute common structure-function modules (SFMs).
Their methodology defines a cross-modularity index, that indicates how
modular a parcellation is respect to both matrices, and how similar is the
internal connectivity in both modalities. By searching for the parcellation
that maximizes this index, they obtain a cortical parcellation with 20 structure-function
modules. Parisot et al. \cite{Parisot2017} starts by computing a set of
parcellations from fMRI, rs-fMRI, estimations of myelin maps, and tractography.
These parcellations are then fuse based on their local reliabilities by means of
mixture models. The fused parcellation is iteratively refined, forcing 
the parcellations to converge towards a set of mutually informed modality specific
parcellations.

The most notable example of multi-modal parcellations is the
Glasser-van Essen atlas.~\cite{Glasser2016}. Glasser et al.~\cite{Glasser2016}
divide the cortex in 180 regions which borders are consistent with: 
myelin content maps, cortical thickness maps, and task-fMRI activations. Their
approach combines a semi-automated prior segmentation with a machine learning
algorithm to optimize the border placement.

\section{Discussion}

%The brain atlas concordance problem: quantitative comparison of anatomical parcellations \cite{Bohland2009}.
The brain is a highly complex cellular network, in order to study it, some
way of dimensionality reduction is needed. Given that neurons tend to organize
and activate in a spatial coherent way, modeling the brain as a set of interacting
regions seems like a valid approach. As seen in sections \ref{sec:anatomical}-\ref{sec:multimodal},
different brain parcellations exist based on criteria such as anatomy, function,
cytoarchitecture or extrinsic connectivity. Each criteria sees the brain in
a different way, and relies on different acquisition methods. We find important
to discuss their advantages and shortcomings.

Cytoarchitectural parcellation denote the cellular composition of the brain,
therefore, they are the perfect candidate to abstract populations of neurons.
However, when using these parcellations is important to acknowledge their limitations.
First, extant cytoarchitectonic maps do not cover the whole brain, in fact, only 
approximately 40\% of the cortical surface has been mapped up to
now~\cite{Amunts2007}. Second, all cytoarchitectonic atlases are based on just
a few post-mortem brains, making them hard to represent a population. This is
product of the complex process of dissecting a brain, delimiting its regions,
and registering the results to a common space. The whole process requires in most
cases one person year work per region. Finally, and most importantly, given the
amount of variability~\cite{Zilles2013} in different subjects, registering a 
cytoarchitectonic atlas to a new brain based on anatomical features does not
guarantee the correct localization of the areas. Until now, post-mortem dissection
remains as the only way to correctly locate cytoarchitectural areas.

Anatomical parcellations are based on fairly common brain landmarks, making them
highly reproducible across subjects. This incurs in the trade-off of having coarse
brain regions, which most of the time do not account correctly for cellular
composition, or brain function~\cite{Amunts2007}.

Functional parcellations map cognitive functions to brain locations, and are
a key element to understand how the brain works. Most techniques are based
on the modular paradigm, which states that one brain region is specialized
for one function. Therefore, regions are generally inferred based on relative
differences of some sort, this is, a region is defined as functionally specialized
because: many subjects with a deficiency in the cognitive function showed a
lesion in that region; or, the region it shows more activation while the subject
realizes the function. This has at least two heavy limitations. First, lesions
in the brain almost never affect only one cognitive function~\cite{Fuster2000},
making hard to decide which function to map. Second, is hard to create tasks
that isolate a single function, in order to correlate it with in-vivo brain
activation. Also, limitations in the time resolution of fMRI do not allow to
infer causality interactions between regions. Finally, accumulating evidence
suggests that the modular paradigm has serious limitations and might in fact
be misleading~\cite{Bressler2010}. Bressler and Menon~\cite{Bressler2010}
propose to start defining functional networks, and relate cognitive functions
to the interactions between such networks.

Structural parcellations define regions with homogeneous axonal connectivity.
While many studies show structural parcels are functionally
relevant~\cite{Gallardo2017a, Anwander2006, ThiebautdeSchotten2016},
more studies on reproducibility are needed\cite{Zilles2013}. Another
important downside, is that the underlying technique, tractography, is still
not mature enough. In fact, recent studies show that state-of-the-art
tractography algorithms create four times more false positives than true
positives.

Multi-modal parcellations combine information from different neuroimaging methodologies,
in order to create regions consistent with multiple neurobiological properties.
Their main limitation is that sometimes, regions tend to overrepresent one
modality. A clear example is the subdivision of the precentral and poscentral
gyrus in the atlas of Glasser et al.~\cite{Glasser2016}. Even when motor tasks
are used during the construction of the atlas, the resulting subdivisions of
both motor and sensory cortex appear to be driven by myelination. This
result is inconsistent with the well known functional subdivision of both motor
and sensory cortex.

If a true and unique parcellation of the brain exists, it has not been found yet.
Meanwhile, different atlases and techniques to divide the brain coexist. In this
review, we discussed the most common criteria to parcellate the brain, alongside
their underlying biological hyphotesis, and present different parcellations 
for each one. We hope our discussion helped to highlight that, which parcellation
to use in practice will heavily depend on the hypothesis and the goal of the
study to be done.

\section{Conclusion}
In order to reduce the complexity of the brain many studies start by subdividing
the brain. Depending on the hypothesis driving the study, different type
of parcellations can be used. In this chapter we presented parcellations based
on different criteria: cytoarchitecture; anatomy; function, and axonal connectivity.
We also presented some parcellation which are driven by a mixture of such
criteria. In the following chapter we will introduce the first contribution of
this thesis: a technique to parcellate the cortex based on its structural
connectivity. Our technique allows to create parcellations at the single-subject
and group level, while having a good correlation with known divisions of the brain.

\chapterbib
