\chapter{Mapping the Brain: A review of the brain divisions}

\section{Overview}
The brain is composed of billions of neurons interacting at the same time
between them. Given the complexity of this network, a dimensionality reduction
is needed in order to study its properties. Since its beginning, neuroscientist
have divided the brain based on different criteria. This allows to study the
brain as a set of interacting regions, allowing to abstract the underlying
neuronal complexity. However, it's not clear that a unique and truth division
of the human brain exists. In this chapter, we make a review of the different
type of parcellations that exist; explain their advantages, and the best
scenarios where to use each.

\section{Introduction}
Subdividing the brain in regions homogeneous respect to some criteria is a
necessary first step of every study in neuroscience. By doing this, it's not
only possible to reduce dimensionality, but also to derive rules more general
than when using the whole brain. How to divide the brain will heavily depend
on the task desired to achieve. Using the wrong parcellation can introduce
a bias in the results. Historically, the brain has been divided following
different criteria. Furthermore, with every new technological advance, new
parcellations arise. Perhaps the first parcellation to exist was the anatomical
one. Still something from Demians thesis here. Based solely in the brain
morphology, gross divisions of the brain can be made, as lobes and etc.



\section{All}
where I published mine:
https://www.sciencedirect.com/science/article/pii/S1053811917310091

Surface-Based and Probabilistic Atlases of Primate Cerebral Cortex
David, in Mendeley.

\section{Cythoarquitecture}
Cythoarquitectonic divisions of the brain are based solely in the cellular
composition of the cortex. In this atlases, a brain region possess the same
cells inside of it. The first and most known division is that of Brodmann.
Brodmann divided the brain in XX regions. He showed stuff for humans and
animals. He showed relationship with function.
Von Economo extended this work, and did more stuff. Someone showed that 
in this parcellation, the regions with similar cellular composition tend
to be connected between them.
in my mendeley \cite{Glover2011}.

\section{Anatomical}
Anatomical divisions of the brain are based on the cortical morphology/shape.. check the name.
The most know one is Desikan. Desikan is based on detecting a set of particular
gyris, and then divide the brain based on them. It has XX divisions per
hemisphere, is simmetrical and can be applied in monkeys also, I think.. check it.
This map was further refined by Destrieux. Destrieux works in a similar way,
but makes further subdivisions of giry and stuff. AAL is older than them?
Anyway, no one recomends AAL.

https://biomedia.doc.ic.ac.uk/brain-parcellation-survey/

\section{Functional}

Functional subdivisions of the brain create maps of regions functionally 
specialized. This is, each region is in charge of a specific motor/cognitive
function, or is part of a greater system with a specific functional goal.

\subsection{Anatomical + Cythoarchitectural}
The first functional maps where derived from lessions. Lessions in here
determinated that function. Lessions in there determinated another one.
Some activations while torturing cats also showed interesting stuff.
Penfield + Knowledge of anatomy + broadman + broca + wernicke + etc
There's a lot of information on diseases in the book Clinical Neuroscience,
sounds like a good place to start

\subsection{EEG}
No idea... i have to read it.
https://www.sciencedirect.com/science/article/pii/S1053811917307474

\subsection{fMRI}
With the invention of fMRI, it was possible to measure the level of oxygen
in blood. Knowing that the neurons need oxygen as fuel to fire, we can
find wich region of the brain are being activated for specific tasks.
Even more interesting is the fact that there are some regions that activate
while in a resting state. Here we talk about many of them.

Good review in:
https://www.biorxiv.org/content/biorxiv/early/2017/06/06/135632.full.pdf

https://biomedia.doc.ic.ac.uk/brain-parcellation-survey/

\section{Brain Networks}
Vinod, brain as a system
The lobes and subcortical structures do not function in isolation, in fact they are heavily connected through the fibre bundles which compose the white matter.

[VINOD]
In his view, the human brain contains at least five major core functional networks: (i) a spatial attention network anchored in posterior parietal cortex and frontal eye fields;
(ii) a language network anchored in Wernicke’s and Broca’s areas; (iii) an explicit memory network anchored in the hippocampal–entorhinal complex and inferior parietal cor- tex; (iv) a face-object recognition network anchored in
midtemporal and temporopolar cortices; and (v) a working memory-executive function network anchored in prefron- tal and inferior parietal cortices. 


Gael (which fmri clustering... in mendeley)
https://www.frontiersin.org/articles/10.3389/fnins.2014.00167/full

\section{Semantic}
Semantic parcellations try to map semantic processing to the cortex.
In this maps, a region of the brain is ligated to a specific 'definition?/semantic'.

https://www.nature.com/articles/nature17637
% Mendeley \cite{Glover2011}
Where Is the Semantic System? A Critical Review and Meta-Analysis of 120 Functional Neuroimaging Studies
% In mendeley also

\section{Structural Connectivity}
In a structural parcellation, regions have a homogeneous pattern of connectivity
with the rest of the brain. The first maps come from tracers in monkey. The new
ones come from tractography. Yeah, we know that tractography is not perfect,
but we cannot torture any more monkeys. We name all of the principal players.
Gallardo.
Moreno-Dominguez.
Alfred Awander PCA.
Michel Thiebaut.
Parisot.

\section{Multimodal}
Finally, more and more people is doing this. Maybe a precursor of this is
anatomy+cytoarchitecture. But here we talk about really different modalities,
as T1+T2+fMRI+goodknowswhat. We present at least two. I'm sure there are more.
Parisot.
Van Essen.
...?

\section{Discussion}
There's not such a thing as a unique brain parcellation. While we cannot say
that it doesn't exist. We know that all of our techniques have a limitation,
either in resolution, SNR. At the same time, we cannot study things at the
neuron level, because we have billions of neurons in the brain. It makes a
lot of sense that each parcellation is different, because they're based
in different criteria. Even when trying to parcellate data coming from a
same machine (fmri, dmri, etc), changing the hypothesis will change the 
resulting parcellation. For example, in structural gradient vs structural
parcel. The important thing is to be able to select the right parcellation
for the right study.

\section{Resume of all of the parcellations presented}
Here I put a beautiful table resuming everything.

\section{Conclusion}
In order to reduce the complexity of the brain many studies start by subdividing
the brain. Depending on the hypothesis driving the study, different type
of parcellations can be used. In this chapter we presented parcellations based
on different criteria: cythoarquitecture; anatomy, function, semantics and
structural. We also presented some parcellation which are driven by a mixture
of suche criteria. Each parcellation possess its own advantages, and should be
used in the right context. In the following chapter we will introduce the
first contribution of this thesis: a technique to parcellate the cortex based
on its structural connectivity. Our technique allows to create parcellations
at the single-subject and group level, while having a good correlation with
known divisions of the brain.


\chapterbib
