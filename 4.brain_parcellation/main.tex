\chapter{Mapping the Brain: A review of the brain divisions}

\section{Overview}
The brain is composed of billions of neurons interacting at the same time
between them. Given the complexity of this network, a dimensionality reduction
is needed in order to study its properties. Since its beginning, neuroscientist
have divided the brain based on different criteria. This allows to study the
brain as a set of interacting regions, allowing to abstract the underlying
neuronal complexity. However, it's not clear that a unique and truth division
of the human brain exists. In this chapter, we make a review of the different
type of parcellations that exist; explain their advantages, and the best
scenarios where to use each.

\section{Introduction}
Composed by billions of interconnected neurons, the brain is the most complex
biological machine that we know. Untangling this network is of special importance
to understand the underlying process of concious life. However, studying the 
brain at the cellular level, alongside their interactions is simply not feasible.
Existent techniques to study brain microstructure are highly invasive, and can
only be used in postmortem tissue [cite]. Furthermore, while this techniques
succefully help to characterize the cellular composition of the tissue, they
do not allow to account for the interaction between cells. Specially for whole
brain data, the obtained dataset are simply too huge to be handled [cite].
Because of all of this, some way to reduce dimentionallity is needed.

Thanks to the study of cythoarquitectonics, brain function, and tracers we know
that neighboring neurons tend to work together [many cites]. In particular,
the brain behaves as a small world graph. While which populations
of neurons, and how they interact between is highly subject dependant, this
studies give us a solid support to subdivide the brain before studying it.
Nowadays, subdividing the brain in regions homogeneous respect to some criteria
is a necessary first step of every study in neuroscience. By doing so, it's not
only possible to reduce the dimensionality of the network, but also to derive
rules more general than in the cellular case. Furthermore, having consistent
divisions across subjects allows to infer properties from a population. The 
problem recides in the fact that there's not a gold standard/unique division of
the human brain. How to divide the brain will heavily depend on the task desired
to achieve, since using the wrong parcellation can introduce a bias in the
results. 

Historically, the brain has been divided following different criteria. Furthermore,
with every new technological advance, new parcellations arise. Perhaps the first
parcellation to exist was the anatomical one. Still something from Demians thesis
here. Based solely in the brain morphology, gross divisions of the brain can be
made, as lobes and etc. Then, through the study of brain lesions, brain surgery
and EEG allowed to study brain function, from which gross functional divisions
arise. A couple of years later, Brodmann creates his map based on the
cythoarchitecture. Enter MRI to improve anatomical divisions, PET/fMRI for functional
and more recently dMRI for structural. Then, semantical, and finally, the
multischeme ones.

In this chapter, we make a review of the most important techniques of which
we know about for each modality, giving a brief introduction to how they work
and their main strenghts. All the reviewed parcellations are comprised in the
table X. The chapter is divided in the following: Introduction, Parcellations
and Discussion.

Speciall issue neuroimage \cite{The2018}1

\section{Anatomical}
The regions on an Anatomical divisions of the brain are defined by their shape
or relative position in the brain. For example, in the Desikan atlas, the
pars opercularis region is defined as 'the first gyrus from the precentral gyrus'.
The division of the brain in lobes is a coarse anatomical parcellation, finer
divisions can be created based on the sulci. AAL\cite{Landeau2002} is the most used anatomical
atlases. It divides the brain in 45 anatomical volumes of interest. It was
created by manual parcellation of the spatially normalized single-subject 
high-resolution T1 volume provided by the Montreal Neurological Institute (MNI) \cite{Collins1998}.
One of the most used anatomical atlases is that of Desikan~\cite{Desikan2006}.
Another anatomical atlas is the Harvard-Oxford cortical/subcortical atlases,
distributed with the software FSL [CITE].
The desikan atlas divides each hemisphere in 34 coarse structures (Fig X), and each
area is delimited mostly by sulcus. Desikan atlas is built on the cortical surface by: projecting the cortical folding
patters into a spherical surface, aligning them with an atlas computed from the
average of 40 subjects, and then transfering the labels from the atlas to the
cortical surface. Another one, Destrieux, divides the brain in 74 labels per
hemisphere. The labeling is done following a probabilistic model that takes
into account informationas: curvature and average convexity of the cortica
surface, prior labeling probability for that vertex, as well as the labels
of vertices in a local neighborhood.

add MarsAtlas

https://biomedia.doc.ic.ac.uk/brain-parcellation-survey/


\section{Cythoarquitecture}
\label{sec:cyto_maps}
Only the borders of a few architectonically defined areas show a sufficiently precise association with sulci~\cite{Amunts2007}
As shown with modern techniques, borders between cytoarchitectonic areas are functionally relevant, to the point that many functional regions are refered using cythoarquitectonic labels.

Cythoarquitectonic divisions of the brain are based solely in the cellular
composition of the cortex. In this atlases, a brain region possess a consistent
thickness and cellular organization. For example, Brodmann Area 4 has an unusually
thick cortex; possess giant pyramidal cells, and lacks an internal and external
granular layer. The pioneers in the area are Meynert [CITE4-Thiarhou],
Campbell [CITE5] who divided the cortex into 14 areas, and Elliot Smith [CITE6]
into 50.

The most known cythoarchitectonic division of the brain is that
of Broadmann~\cite{Brodmann1909}. Broadmann defines 52 cortical regions, based
on the inspection trough microscope of cortical sections on post-mortem
primates brains. In 1925 von Economo and Koskinas publish the 'Atlas of 
Cytoarchitectonics of the Adult Human Cerebral Cortex' [CITE]. Economo and
Koskinas [cite] recognized 54 fundamental cytoarchitectonic areas with 76
variants and 107 modifications~\cite{Triarhou2007}. Their atlas is based in the
examination of mentally healthy subjects in the range of 30–40 years of age
through improved microscopes. We also have the Jubrain~\cite{Mohlberg2012}, a
cytoarchitectonic probabilistic maps viever, made by Juelich and Duesseldorf
institutes by analizing the histological sections of ten human postmortem brains.
More recent work includes the study of
neurotransmitter receptors to map divisions in the visual cortex~\cite{Eickhoff2008}, 
and the introduction of microstructural metrics, as the gray level index
[CITE](Schleicher and Zilles, 1990), to create observer-independent parcellation
methods, 

Creates an atlas of the human ventral visual stream from the postmortem brain
of 11 subjects\cite{Rosenke2018}.
    
Extant cytoarchitectonic maps do not cover the complete cerebral cortex. 
Only approximately 40\% of the cortical surface has been mapped up to now. 
This is caused by the difficulties of: studying sections; 3D reconstructing the
histological sections, and registration of post-mortem data to a reference space.
This is a very time-consuming process, requiring as minimum as 1 person year for
an area.


\section{Functional}
Functional parcellations of the brain denotes how different human functions 
are localized in the brain. This is, each region is in charge of a specific
function (i.e. movement, speach, vision), or is part of a greater system with
a specific functional or cognitive goal. The first functional maps where derived
from lessions. A classical example is that of Broca and Wenicke areas, related
to meaningfull speach production and spech comprehension. The areas are named
after Broca [CITE] and Wernicke [CITE], who in the late 1800 reported lesions in
that region for aphasic patients. Another example is the human homunculus,
a representation of motor and sensory functions, which Penfied [cite] mapped
trough experimenting with electrical stimulation of different brain areas of
patients undergoing open brain surgery.

As stated in the previous section (sec. \ref{sec:cytho_maps}, Cythoarchitectonic
parcellations can be seen as functional ones, since they show close relation 
to function. However, they do not cover the whole brain, and we can only relate
them after many patients show specific impediments and have lesions in those
regions. The advenment of functional MRI (fMRI) allowed to measure the level of
oxygen in blood. Knowing that the neurons need oxygen as fuel to fire, we can
find wich region of the brain are being activated for specific tasks. This
allowed to refine and validate in vivo the functional specialization of 
cythoarchitectural parcels and to better define anatomical ones as the human
humunculus \cite{Lashkari2010, Michel2011}. Even more interesting is the fact
that there are some regions that activate while in a resting state. From the
no-task state (resting state), is possible to derive parcellations by seing
which systems activate simultaneusly. The most common way is to computed the
Pearson's correlation between the fMRI time series at each spatial location.
While we have no idea what this means, people uses it. 

PNAS of 2004 \cite{Johansen-Berg2004}


%\cite{Power2012} use Graph analysis and subgraph detection techniques in order
%to characterize functional subnetworks in the connectivity graph. 
Yeo et al. [CITE] 
Data from 1,000 subjects were registered using surface-based alignment.
The connectivity fingerprints were modeled with a von Mises-Fisher distribution,
then, the data points were randomly assigned to different groups and then
iteratively reassigning the group memberships of points to maximize the
agreement of connectivity profiles among points of the same group. The results
yield a 7 parcels and 17 parcels network.

The Cortical Area Parcellation from Resting-State Correlations dataset consists
of 333 cortical patches segmented using resting-state fMRI (Gordon 2014).

By applying different clustering algorithms to this matrix, we
obtain different parcellations. We can use ward clustering [cites from Thirion];
k-means clustering [cites from thirion]; spectral clustering [cites from thirion],
and principal component analysis (PCA) [cites from thirion]. There are also
local gradient approaches, that detect abrupt transitions in functional
connectivity patterns \cite{Wig2014, Schaefer2017}. 
Based on the work by Thomas Yeo et al. [CITE], Schaefer et al.  \cite{Schaefer2017}
add further refinement by subparcellating the global networks based on a local
gradient approach. Parcellations come in several version, breaking down the cortex
into up to 1000 regions based on rs-fMRI.

Presents a multigraph K-way clustering method and applies it to obtain 
parcellations of 50, 100, 150... maybe put in the same bag as the rest \cite{Shen2013}.


A fine-grained parcellations ranging form 200 to 1000 parcellations,
using spectral clustering on rs-fMRI connectivity matrix \cite{Craddock2011}.

Thomas Blumensath et al. makes a thousand small functionally homogeneous regions,
and then apply hierarchical clustering to obtain the rest of the parcellations \cite{Blumensath2013}.

Huth et al. atlas \cite{Huth2016} maps semantic selectivity across the cortex
by using voxel-wise modelling of functional MRI (fMRI) data collected while 
subjects listened to hours of narrative stories. This creates a map where each
region is ligated to a specific systematically map semantic domain. In this
maps, a region of the brain is ligated to a specific semantic domain, i.e.
violent, temporal, proffesional \cite{Huth2016}.

While we dont do EEG, you can review it here \cite{Shen2013}.

add \cite{Ryali2013} https://www.ncbi.nlm.nih.gov/pubmed/23041530

%Good review in:
%https://www.biorxiv.org/content/biorxiv/early/2017/06/06/135632.full.pdf
%https://biomedia.doc.ic.ac.uk/brain-parcellation-survey/


\section{Structural Connectivity}
In a structural parcellation, regions have a homogeneous pattern of connectivity
with the rest of the brain. This is, a region is defined as: 'all of the points
inside are connected to some other ROIs in a similar way'. The granularity
of the ROIs can go from coarse brain regions, defined by other atlas, to voxels.
The first connectivity divisions came form tracers in monkey PANDIAS. Diffusion
MRI (dMRI) enables the in vivo exploration of extrinsic connectivity on the
human brain. As with fMRI, the most common way to generate a parcellation is
to compute a connectivity matrix, and then parcellate it using some clustering
technique. Given the difficulty of handling huge connectivity matrices, initial
techniques used to divide only portions of the brain.

Behrens et al.\cite{Behrens2003} thalamus Hard segmentation was performed by classifying the seed voxel as connecting to the cortical mask with the highest con- nection probability.
Alfred Anwander performs k-means\cite{Anwander2006} in the connectivity matrix of Broca's area.
Thiebaut de Schotten et al. divide the occipital cortex\cite{ThiebautdeSchotten2014}, frontal lobe \cite{ThiebautdeSchotten2016}
Moreno-Dominguez et al. use a constrained hierarchical clustering to parcellate a structural matrix \cite{Moreno-Dominguez2014}
Gallardo buils on top of this.
They use spectral reordering to study the organization of the temporal lobe \cite{Bajada2017}
(O'Muircheartaigh and Jbabdi, 2017) is entirely data-driven (based on independent component analysis) and allows one to obtain sets of white matter components with their associated gray matter networks \cite{Muircheartaigh2018}


Previous single-subject structural parcelling techniques work by refining other
parcellations~\cite{Clarkson2010};parcelling only part of the
cortex~\cite{Lefranc2016, Roca2009, ThiebautdeSchotten2014, ThiebautdeSchotten2016},
perform a groupwise parcellation after constructing an average connectivity profile~\cite{Clarkson2010, Roca2010},

Parisot uses ideas from cosegmentation and spectral decomposition to create consistent groupwise parcellations \cite{Paristot2015}.


\section{Multi-modal}
So far, all the presented methodologies were solely on one technique. In
multi-modal parcellations, the goal is to mix two or more 
Parisot \cite{Parisot2017}.
Based on function, Van Essen \cite{Glasser2016}.
Braintome\cite{Fan2016}
Structural+Functional correlation and then hierarchical clustering\cite{Diez2014}


\section{Discussion}
The brain atlas concordance problem: quantitative comparison of anatomical parcellations.

Here we name them, some people profile them:

Thirion did it for functionali \cite{Thirion2014}
Arslan did it for functional \cite{Arslan2018}
Gorgolewski KJ, Tambini A, Durnez J et al. Evaluation of full brain parcellation schemes using the NeuroVault database of statistical maps [version 1; not peer reviewed]. F1000Research 2017, 6:1986 (poster) (doi: 10.7490/f1000research.1115065.1) 

Some other reviews: Jbabdi \cite{Jbabdi2013}, \cite{Arslan2018}, 

Gatherings: http://www.lead-dbs.org/helpsupport/knowledge-base/atlasesresources/cortical-atlas-parcellations-mni-space


There's not such a thing as a unique brain parcellation. While we cannot say
that it doesn't exist. We know that all of our techniques have a limitation,
either in resolution, SNR. At the same time, we cannot study things at the
neuron level, because we have billions of neurons in the brain. It makes a
lot of sense that each parcellation is different, because they're based
in different criteria. Even when trying to parcellate data coming from a
same machine (fmri, dmri, etc), changing the hypothesis will change the 
resulting parcellation. For example, in structural gradient vs structural
parcel. The important thing is to be able to select the right parcellation
for the right study.

The important thing is to validate it correctly, some people do, like \cite{Gallardo}, \cite{Auzias2016}, \cite{ThiebautdeSchotten2014, ThiebautdeSchotten2016}

\section{Resume of all of the parcellations presented}
Here I put a beautiful table resuming everything.

\section{Conclusion}
In order to reduce the complexity of the brain many studies start by subdividing
the brain. Depending on the hypothesis driving the study, different type
of parcellations can be used. In this chapter we presented parcellations based
on different criteria: cythoarquitecture; anatomy, function, semantics and
structural. We also presented some parcellation which are driven by a mixture
of suche criteria. Each parcellation possess its own advantages, and should be
used in the right context. In the following chapter we will introduce the
first contribution of this thesis: a technique to parcellate the cortex based
on its structural connectivity. Our technique allows to create parcellations
at the single-subject and group level, while having a good correlation with
known divisions of the brain.


\chapterbib
