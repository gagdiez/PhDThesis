\section{Tractograf\'ia}

Dipy es una librer\'ia para python que contiene, entre otras herramientas,
varios algor\'itmos para generar \textit{streamlines}. Un \textit{streamline} es
uno de los posibles caminos que puede realizar una particula, comenzando desde una
semilla en la materia blanca, siguiendo un mapa probabil\'istico de direcciones.
El repetir este experimento es una forma posible de crear tractogramas. Para este
trabajo eleg\'imos utilizar la implementaci\'on de \textit{LocalTracking} (LT de
aqu\'i en mas) que se encuentra en el paquete \textit{dipy.tracking.local}; y 
una implementaci\'on  pr\'opia (MSL de aqu\'i en mas) que utiliza la clase \textit{ProbabilisticDirectionGetter} del paquete \textit{dipy.direction} y la funci\'on
\textit{markov\_streamlines} de \textit{dipy.tracking.markov}.

Es importante destacar que como la interfaz del \textit{DirectionGetter} (DG) 
no es compatible con la funci\'on \textit{markov\_streamlines} (ver documentaci\'on
de dipy) fue necesario crear una clase que hiciera de puente entre ambas. Dicha 
clase se encuentra en el paquete scripts.proba, y lo que hace es seleccionar 
una direcci\'on inicial al azar entre las propuestas por DG, y la direcci\'on
propuesta por DG luego de cada paso hasta que se sale de la mascara de materia
blanca. 

Para utilizar cada uno de estos algoritmos es necesario encuadrar la informaci\'on
de dMRI dentro de un modelo. En este caso se utiliz\'o el modelo 
\textit{ Constrained Spherical Deconvolve Model} para ambos.  
Por cuestiones de \textit{performance} se gener\'o un script para computar en
paralelo los \textit{streamlines} de cada semilla por cada algoritmo. Dicho script 
fue ejecutado en el cluster del INRIA.

\textbf{notebook: Tractograf\'ia probabil\'istica.}

\subsection{Estabilidad Algoritmos}

Para determinar si el tractograma medio converg\'ia se utiliz\'o un n\'umero 
peque\~no de semillas y la t\'ecnica  estad\'istica de \textit{bootstrap}.
En principio se comput\'o un total de 15.000 \textit{streamlines} por semilla.
Luego por cada semilla se calcul\'o el tractograma medio y la varianza de 
cada voxel utilizando mil submuestras aleator\'ias del mismo tama\~o. Esto se
repitio\'o con varios tama\~nos de submuestra para ver as\'i que tan variable
era la muestra a medida que crec\'ia.
