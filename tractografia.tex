Dado que las MRI convencionales basan su contraste en el tiempo de relajaci\'on
de los tejidos, la materia blanca se ve homogenea tanto en las im\'agenes de tipo
T1 como en las de tipo T2. 

En cambio, las Im\'agenes de Difusi\'on por
Resonancia Magn\'etica (dMRI) se basan en la idea de que las particulas 
de agua tienden a moverse en la direcci\'on de las fibras donde se encuentran.
Por ello se cree que estimando las direcciones de difusi\'on es posible estimar
la estructura de los axones en el cerebro.

\newpage
