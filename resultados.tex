\chapter{Resultados}

En el capitulo {\ref{ch:metodos} presentamos la t\'ecnica de 
\textit{bootstraping}; el m\'etodo de Moreno-Dominguez para agrupar
tractogramas junto con algunas de sus falencias te\'oricas y como
solucionarlas usando la funci\'on \textit{logit}. En la secci\'on 
\ref{ch:nuestro}, presentamos nuestro m\'etodo para parcelar la corteza en
su totalidad. En las siguientes secciones mostramos primero los resultados
obtenidos al estudiar la estabilidad de los tractogramas; luego parcelamos
el \'area de Broca con ambos m\'etodos y finalmente parcelamos el
hemisferio derecho usando tambi\'en ambos m\'etodos. Todos los estudios se
realizaron sobre una misma mujer diestra de entre 23 y 26 a\~nos. Sus datos
fueron descargados de la base de datos \textit{Human Connectome Project}
\cite{VanEssen2012}.


\section{Estabilidad tractogramas}

Las Figuras \ref{fig:m1}, \ref{fig:m2} y \ref{fig:m3} muestran, para tres semillas
distintas, cinco cortes axiales del tractograma que se consigue al utilizar quince
mil part\'iculas.\\

Las Figuras \ref{fig:s1}, \ref{fig:s2} y \ref{fig:s3} muestran la varianza de 
cada voxel dentro de un mismo corte axial. La varianza se calcul\'o generando
mil tractogramas desde distinto n\'umero de streamlines. \\

La Figura \ref{fig:mv} muestra la media y varianza de los voxels $A$, $B$ y $C$
marcados en las Figuras \ref{fig:s1}, \ref{fig:s2} y \ref{fig:s3}. Estos voxels
fueron los que mayor varianza presentaron al generar tractogramas con dos mil
\textit{streamlines}.


\begin{figure}[h!]
   \centering
    \includegraphics[width=\textwidth]{img/m1.png}
    \caption{Tractograma para la semilla $S1$ utilizando toda la muestra.}
    \label{fig:m1}
\end{figure}

\begin{figure}[h!]
   \centering
    \includegraphics[width=\textwidth]{img/s1.png}
    \caption{Desviaci\'on Estandar respecto a la semilla $S1$. Mismo corte axial
             variando el tama\~no de las submuestras.}
    \label{fig:s1}
\end{figure}

\begin{figure}[h!]
   \centering
    \includegraphics[width=\textwidth]{img/m2.png}
    \caption{Tractograma para la semilla $S2$ utilizando toda la muestra.}
    \label{fig:m2}
\end{figure}

\begin{figure}[h!]
   \centering
    \includegraphics[width=\textwidth]{img/s2.png}
    \caption{Desviaci\'on Estandar respecto a la semilla $S2$. Mismo corte axial
             variando el tama\~no de las submuestras.}
    \label{fig:s2}
\end{figure}

\begin{figure}[h!]
   \centering
    \includegraphics[width=\textwidth]{img/m3.png}
    \caption{Tractograma para la semilla $S3$ utilizando toda la muestra}
    \label{fig:m3}
\end{figure}

\begin{figure}[h!]
   \centering
    \includegraphics[width=\textwidth]{img/s3.png}
    \caption{Desviaci\'on Estandar respecto a la semilla $S3$. Mismo corte axial
             variando el tama\~no de las submuestras.}
    \label{fig:s3}
\end{figure}

\begin{figure}[h!]
   \centering
    \includegraphics[width=\textwidth]{img/med_var_all.png}
    \caption{Media y desviaci\'on estandar de los voxels con mayor varianza.}
    \label{fig:mv}
\end{figure}


\section{Parcelamiento del \'area de Broca}

Estos son los resultados de parcelar el \'area de Broca con un total de 
aproximadamente $760$ semillas. 

\subsection{M\'etodo Moreno-Dominguez}

Las figuras \ref{fig:moreno0} y \ref{fig:moreno1} muestran las parcelas obtenidas 
usando $k=0$ a distintos niveles de profundidad del dendrogama. Las figuras 
\ref{fig:moreno2} y \ref{fig:moreno3} muestran los resultados usando $k=400$ y $k=700$
respectivamente.

\begin{figure}[h!]
    \includegraphics[width=\textwidth]{img/broca/moreno_0.png}
    \caption{M\'etodo Moreno sin restricciones}
    \label{fig:moreno0}
\end{figure}
                                                                                                                       
\begin{figure}[h!]
    \includegraphics[width=\textwidth]{img/broca/moreno_0_deep.png}
    \caption{M\'etodo Moreno sin restricciones, mayor profundidad en el 
            dendrograma}
    \label{fig:moreno1}
\end{figure}

\begin{figure}[h!]
    \includegraphics[width=\textwidth]{img/broca/moreno_400.png}
    \caption{M\'etodo Moreno, primeras cuatrocientas uniones entre vecinos}
    \label{fig:moreno2}
\end{figure}

\begin{figure}[h!]
    \includegraphics[width=\textwidth]{img/broca/moreno_750.png}
    \caption{M\'etodo Moreno, primeras setecientas uniones entre vecinos}
    \label{fig:moreno3}
\end{figure}

\clearpage

\subsection{Parcelamiento en el espacio eucl\'ideo}

Las figuras \ref{fig:nuestro0} y \ref{fig:nuestro1} muestran las parcelas obtenidas 
usando $k=0$ a distintos niveles de profundidad del dendrogama. Las figuras 
\ref{fig:nuestro2} y \ref{fig:nuestro3} muestran los resultados usando $k=400$ y
$k=700$ respectivamente.

\begin{figure}[h!]
    \includegraphics[width=\textwidth]{img/broca/logit_0.png}
    \caption{Nuestro m\'etodo sin restricciones}
    \label{fig:nuestro0}
\end{figure}
                                                                                                                        
\begin{figure}[h!]
    \includegraphics[width=\textwidth]{img/broca/logit_0_deep.png}
    \caption{Nuestro m\'etodo sin restricciones, mayor profundidad en el 
            dendrograma}
    \label{fig:nuestro1}
\end{figure}

\begin{figure}[h!]
    \includegraphics[width=\textwidth]{img/broca/logit_400.png}
    \caption{Nuestro m\'etodo sin preprocesamiento, cuatrocientos pasos de preprocesamiento}
    \label{fig:nuestro2}
\end{figure}

\begin{figure}[h!]
    \includegraphics[width=\textwidth]{img/broca/logit_750.png}
    \caption{Nuestro m\'etodo sin preprocesamiento, setecientos pasos de preprocesamiento}
    \label{fig:nuestro3}
\end{figure}

\clearpage

\subsection{Acercamiento a ambos m\'etodos}
\label{sec:acercamiento}

Las figuras \ref{fig:ambos0}; \ref{fig:ambos1} y \ref{fig:ambos2} presentan los
resultados de ambos m\'etodos en simultaneo para $k=0$, $k=400$ y $k=700$ respectivamente.

\begin{figure}[h!]
    \centering
    \includegraphics[width=0.9\textwidth]{img/broca/vs_0.png}
    \caption{M\'etodo Moreno-Dominguez (izquierda) y nuestro (derecha) sin preprocesamiento}
    \label{fig:ambos0}
\end{figure}

\begin{figure}[h!]
    \centering
    \includegraphics[width=0.9\textwidth]{img/broca/vs_400.png}
    \caption{M\'etodo Moreno-Dominguez (izquierda) y nuestro (derecha). Cuatrocientos pasos de preprocesamiento}
    \label{fig:ambos1}    
\end{figure}
    
\begin{figure}[h!]
    \centering
    \includegraphics[width=0.9\textwidth]{img/broca/vs_700.png}
    \caption{M\'etodo Moreno-Dominguez (izquierda) y nuestro (derecha). Setecientos pasos de preprocesamiento}
    \label{fig:ambos2}
\end{figure}

\clearpage


\section{Parcelamiento del hemisferio izquierdo}

Estos son los resultados de parcelar el hemisferio izquierdo con un total de
aproximadamente $24000$ semillas. 

\subsection{M\'etodo Moreno-Dominguez}

Las figuras \ref{fig:moreno_corteza0}; \ref{fig:moreno_corteza1} y \ref{fig:moreno_corteza2}
muestran las parcelas obtenidas usando $k=10000$ a distintos niveles de profundidad
del dendrogama.

\begin{figure}[h!]
    \includegraphics[width=\textwidth]{img/all_brain/moreno_10000.png}
    \caption{M\'etodo Moreno-Dominguez, primeras 10000 uniones entre vecinos}
    \label{fig:moreno_corteza0}
\end{figure}

\begin{figure}[h!]
    \includegraphics[width=\textwidth]{img/all_brain/moreno_10000_deep0.png}
    \caption{M\'etodo Moreno-Dominguez, primeras 10000 uniones entre vecinos, mayor profundidad}
    \label{fig:moreno_corteza1}
\end{figure}

\begin{figure}[h!]
    \includegraphics[width=\textwidth]{img/all_brain/moreno_10000_deep1.png}
    \caption{M\'etodo Moreno-Dominguez, primeras 10000 uniones entre vecinos, mayor profundidad}
    \label{fig:moreno_corteza2}             
\end{figure}

\clearpage

\subsection{Utilizando nuestro m\'etodo}

Las figuras \ref{fig:nosotros_corteza0} y \ref{fig:nosotros_corteza1} muestran 
las parcelas obtenidas usando $k=0$ a distintos niveles de profundidad del dendrogama.
Las figuras \ref{fig:nosotros_corteza2}; \ref{fig:nosotros_corteza3} y
\ref{fig:nosotros_corteza4} muestran los resultados usando $k=20000$  a distintos
niveles de profundidad del dendrograma.

\begin{figure}[h!]
    \includegraphics[width=\textwidth]{img/all_brain/logit_0.png}
    \caption{Nuestro m\'etodo sin restricciones}
    \label{fig:nosotros_corteza0}
\end{figure}
                                                                                                                        
\begin{figure}[h!]
    \includegraphics[width=\textwidth]{img/all_brain/logit_0_deep.png}
    \caption{Nuestro m\'etodo sin restricciones, mayor profundidad en el 
            dendrograma}
    \label{fig:nosotros_corteza1}
\end{figure}

\begin{figure}[h!]
    \includegraphics[width=\textwidth]{img/all_brain/logit_20000.png}
    \caption{Nuestro m\'etodo, primeras 20000 uniones entre vecinos}
    \label{fig:nosotros_corteza2}
\end{figure}

\begin{figure}[h!]
    \includegraphics[width=\textwidth]{img/all_brain/logit_20000_deep0.png}
    \caption{Nuestro m\'etodo, primeras 20000 uniones entre vecinos, mayor profundidad}
    \label{fig:nosotros_corteza3}
\end{figure}

\begin{figure}[h!]
    \includegraphics[width=\textwidth]{img/all_brain/logit_20000_deep1.png}
    \caption{Nuestro m\'etodo, primeras 20000 uniones entre vecinos, mayor profundidad}
    \label{fig:nosotros_corteza4}
\end{figure}

\subsection{Acercamiento a ambos m\'etodos}
\label{sec:acercamiento_corteza}

La figura \ref{fig:vs_moreno} presenta la corteza parcelada usando el m\'etodo 
Moreno-Dominguez con $k=10000$. La figura \ref{fig:vs_nosotros} muestra el 
resultado de usar nuestro m\'etodo con $k=20000$.

\begin{figure}[h!]
    \centering
    \includegraphics[width=0.62\textwidth]{img/all_brain/vs_moreno.png}
    \caption{M\'etodo Moreno-Dominguez, primeras 10000 uniones entre vecinos}
    \label{fig:vs_moreno}
\end{figure}

\begin{figure}[h!]
    \centering
    \includegraphics[width=0.62\textwidth]{img/all_brain/vs_nuestro.png}
    \caption{Nuestro m\'etodo, primeras 20000 uniones entre vecinos}
    \label{fig:vs_nosotros}
\end{figure}



