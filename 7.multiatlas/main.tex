\chapter{Inferring the Localization of White-Matter Tracts using Diffusion Driven Label Fusion}

\section{Overview}
White matter pathologies such as tumors or traumatic brain injury disrupt the
structure of white matter. These disruptions hamper the inference of affected
pathways using tractography. A way to overcome this is to use a label fusion
technique. Label fusion aims to infer the localization of the brain structure
of a subject from its localization in a group of control subjects. The most
common technique is known as the voting rule \cite{voting}, where a structure
is said to be present in a voxel if it's present in the majority of the voting
subjects. Furthermore, this can be improved by weighting each vote by the
similarity between the T1 of each voting subject and the subject to be inferred.
However, these techniques only relay in the spatial localization of the structures.
In this work, we introduce a way to weight the vote of each subject based on
how the voted pathway is supported by the test subject's diffusion data. This
is, if the diffusion data of the test subject is consistent with the direction
of the voted pathway, the vote has a higher weight. We show that adding dMRI to
the label fusion process achieves a similar number of true positives than the
voting technique, with a 60\% less of false positives. However, this incurs in
a trade-off of a 40\% false negatives.

\section{Introduction}

White matter pathologies such as tumors or traumatic brain injury disrupt the
structure of white matter. These disruptions hamper the inference of affected
pathways using tractography. Furthermore, for some pathologies as tumors or
edemas, estimating brain pathways using tractography becomes impossible [cite].
For such pathologies, it's not possible to use Diffusion Magnetic Resonance
Imaging (dMRI) to tract within or around the lesion. Given the nature of the
lesions, the tracking results in interrupted or erroneous tracts. This situation
makes hard to infer which pathways are directly affect by the pathology. One
possible way to overcome this issue, is by aggregating spatial information
from other subjects in order to infer the affected tracts. Assuming we know the
location of a set of tracts in the brains of a group of healthy subjects, we
could register them to the brain of our patient and use a label fusion technique.
Label fusion is a family of techniques that aims to infer the localization of the
 brain structure of a subject from its localization in a group of control subjects.
One well known technique within this family is Majority Voting [cite]. Given a
specific location, each subject is said to "vote" for one label. The inferred
label for that location will be the one with more votes. For example, when
labeling a volume, each voxel receives the label voted by the majority of the
control subjects.

Majority Voting is simple to implement and has been demonstrated to yield
accurate segmentations [reference]. However, this technique is blind both to
registration problems and anatomical variability between subjects. To overcome
this, it has been propose to weight the vote of each subject by some measure of
 similarity between the patient and the subject [cite]. The underlying
intuition is the choosing of labels should be driven by those subjects who
resemble the most to the one being labeled. The practical advantages of various
strategies based on this idea have recently been demonstrated [6-sabuncu].

The techniques described so far relay only on anatomical information, not taking 
into account the structure of white matter. In the case of white matter pathways,
the presence of a path constrains the brownian motion in the brain, which can be
measured by dMRI. In this work, we introduce a new label fusion technique that,
taking advantage of dMRI, weights the vote of each subject based on how the voted
pathway is supported by the test subject's diffusion data. This is, if the
diffusion data of the test subject is consistent with the direction of the voted
pathway, the vote has a higher weight. Our technique also allows to work with
crossing tracts, by modeling multiple labels per voxel.

We validate our technique in 13 subjects of the Human Connectome Project (HCP).
For each subject, we infer the location of 18 left hemisphere tracts using
whole-brain tractography and an implementation of the white matter query language
(WMQL). We use this results as ground truth to compare against inferring the
tracts from using the other 12 subjects with our proposed technique and Majority
Voting. Our results show that adding dMRI to the label fusion process achieves a
similar number of true positives than Majority Voting, with a 60\% less of
false positives, incurring in a trade-off of a 40\% false negatives.

\section{Methods}

Our label fusion technique takes advantage of dMRI to weight the vote of each
subject. The vote for a specific label gets a higher weight if it's supported
by the diffusion data of the subject being labeled. In this section, we first
start explaining the concept of Orientation Density Function of diffusion in
dMRI and tract directionality. Then, we present how to use these concepts to
compute the weights of each vote.

\subsection{Estimating an Orientation Density Function from dMRI Data}
A dMRI image is composed by many volumes, each one linked to an acquisition
direction. Each value in these volumes represents the intensity of diffusion
in that point of the brain for the direction linked to the volume. By fitting
the diffusion information of a voxel in a Constrained Spherical Deconvolution
model, it's possible to estimate a orientation density function (ODF) over a
sphere. Needs more explanation.

\subsection{Estimating an Orientation Density Function from Streamlines}
Given a tract expressed as set of streamlines, assuming they don't have sharp
turns we can estimate their main directionality on a voxel by looking at its
entry and exit points. For each pair of entry and exit points we can compute a
direction, and from this set of directions, we can compute a ODF over a sphere,
expressing the distributions of directions for that voxel. Explain ACGD.

% Since we also want to estimate directionality from tracts, we introduce the concept of acgd
\subsection{Label fusion}
In the previous section we presented how to compute and ODF from dMRI data and
how to estimate the ODF from a tract in a voxel. Now, we introduce the Majority
Voting technique and our improvement using dMRI.

\subsubsection{Majority Voting}
Let $labels = \{l_i\}, \forall_i l_i \in N$ be the set of labels representing
tracts and grey matter structures in one hemisphere. Let ${L_s}, s \in S$
represent the labeling of a set of subjects $S$, where each 
$L_s \in labels^{v\times v}$ is a 3D volume represeting the labeling of a
specific subject. Majority Voting [Rohlfing et al. (2004)] infers the label of
each voxel $x$ in the test subject ($L(x)$) by computing:

\begin{equation}
\label{eq:mvoting}
\begin{aligned}
    \hat L(x) = \argmax_{l \in labels} \sum_{s\in S} p(L(x) = l | L_s(x)),\\
    \text{where} \\
    p(L(x) = l | L_s(x)) =
    \begin{cases}
        1,& \text{if } L_s(x) = l \\
        0,& \text{otherwise}
    \end{cases}
\end{aligned}
\end{equation}

In this case, it's said that each train subject votes for a label per voxel,
and the label with the most amount of votes is assigned to the test subject's
voxel.

\subsubsection{Diffusion based label fusion}
We want to introduce a weight to these votes, taking into account how well the
tract can explain the underlying Diffusion data of our test subject. In
particular, we want to the Orientation Distribution Function (ODF) of our
test subject's diffusion. This is, we want to profit of the fact that we know
how the water particles are moving in its brain, following the underlying tracts.
As explained in section X, in a given voxel, we can compute the main
directionality of the tract $l$ of the train subject $s$ and the ODF of our
test subject's diffusion. To see if the tract are aligned with the
directionality of the voted tract we proposed the following model:

\begin{equation}
\label{eq:argmax}
\hat L(x) = \argmax_{l \in labels} \sum_{s\in S} p(L(x) = l ; L_s(x)) p(D(x) ; D_{sl}(x)),
\end{equation}
where
\begin{equation}
\label{eq:peaks}
p(P(x) ; D_{sl}(x)) = <ODF(x), ODF_sl(x)>.
\end{equation}

In our model, the first term remains the same as in the voting scheme.
Our second term, $p(D(x) | D_{sl}(x))$ express the probability
of seeying the ODF that we found in our test subject, given that the real
tract passing by is the one voted by the test subject. We compute this
probability as in [REFERENCE], by computing the inner product between the
two ODFs. The ODFs need to be first normalized in order for <ODF, ODF> = 1.

In the case of multi-label, we can combine the info of directions in a same acgd.
This allows us to better compute the real label when there are fibers crossing
inside of a voxel.
For each voxel in the white-matter of our test subject, we select the label
with the highest summation of weighted votes. 

\section{Experiments and Results}

In the previous section we presented how to add Diffusion weighted information
to the process of Mayority Voting in order to improve the multi-atlas technique.
Now, we present experiments, both in synthetic and real data, showing that
our technique achieves better results than Mayority Voting.

\subsection{Data and Preprocessing}
We randomly selected 13 subjects from the HCP500 dataset from the Human
Connectome Project. For each subject, we computed whole-brain tractography
using each voxel in the white-matter as a seed and simulating 8 particles per
seed [REF]. We extracted the main tracts from the left hemisphere tractogram
(18 tracts in total) using the implementation of the white-matter query
language (WMQL),(Wasserman et al. 2016). For each pair of subjects, we
registered they tracts to each others brain.

\subsection{Assessing goodness of the technique in synthetic data}
The dwi of TEST has one tract. We have two types of subjects, ST who votes for
a tract, and S0 who votes for no tracts. ST starts with a completely aligned
tract, and we go rotation it until 90 deg. Then, we compute the weight of
each vote. The proportion between them, tells us how many subjects we need to
get either one or the other label. We show that, when we have subjects voting
for the right tract, their weight go up, allowing to only 30\% of them to win.
But when the tract is not aligned, then they lose, except if they're the 70\%.
In between.. stuff.
But, when the GT is NO-TRACT, we always need more than 60\% of the people.
Something similar happens with crossing, I hope.
This is better than the 50\% of voting.

\subsection{Assessing goodness of the techniques in real data}
To assess the "goodness" of our technique we made a leave-one-out
cross-validation. At each step, we selected one of the subjects as test and 
used the rest as train subjects. Using the registered tracts of the train
subjects, we computed parcellations using both the voting rule and our technique.
Since we also have the tracts of each test subject, we compute a 'ground truth'
parcellation of they white matter. Finally, we computed the confusion matrix of
both the Majority Voting and our technique. A confusion matrix is a matrix of 
size labels by labels where the entry (i,j) is the number of times the label in
the ground truth was i and the technique labeled j. Table 1 shows that our
proposed achieves a similar number of false negatives while obtaining a 64\%
less of false positives in average. This incurs on a trade-off of having 39\%
more false negatives and 18\% less true positives in average, underlying that
our technique a more conservative.\\

\subsection{Simulating Lesions in the White-Matter}
To test how our technique would behave in a injured brain, we simulate a tumor in the white matter of one of our subjects. We place the tumor to interrupt the ILF. We do so by selecting a set of neighboring voxels where the ILF passes by, and mix their signal with diffusion signal from random voxels in the ventricles. This is, for each selected voxel $x$ in the brain, we chose a random voxel $v$ in the ventricle and mix their signals:
$$S(x) = S(x)(1-\alpha) + S(v)\alpha, \alpha \in [0,1]$$.
Since the ventricles are regions filled with cerebrospinal fluid (CSF), they diffusion is approximately isotropic. We compute for which values of $\alpha$ WMQL stops being able to detect the ILF, and for which values our technique stops being able to detecting it. Since the Major Voting does not rely on diffusion data, this experiment does not affect the results obtained in the previous experiment for Major Voting.
WMQL stops detecting the ILF at... and our technique stops detecting it in the voxels at .... 

\subsection{Discussion}
We want to infer the position of tracts in the white matter.
Since we do not have many subjects, we cannot relay on Deep Learning techniques.
We decide to use Label Fusion techniques.
Some people showed that Mayority Voting works well for this things.
The problem is that majority voting relies only on the spatial location.
Problems of registration.
We introduce a new way to take into account the diffusion information.
This makes a lot of sense, since we are trying to infer tracts, and the 
diffusion data is related to the underlying tracts.
In synthetic data, we show that our technique is much better than
majority voting, specially when the tracts align correctly with the
underlying diffusion. It also works like charm, when the tracts are completely
missaligned with the diffusion. In other cases, is as good as majority voting.
In real data, this is reflected, by showing that diffusion voting is more
conservative. WMM lessions.

\subsection{Conclusions}
In this work we presented a labeling fusion technique that relies on dMRI data
to infer the localization of white-matter tracts. The results show that our
technique is more conservative than the voting rule, which is desired when
studying pathologies, at the cost of having more false negatives.

\subsubsection*{Table}

\begin{table}[]
\centering
\caption{Confusion matrix for both techniques and the ratio between them}
\label{my-label}
\begin{tabular}{|l||l|l||l|l||l|l|}
\hline
      & \multicolumn{2}{c||}{Diffusion} & \multicolumn{2}{c||}{Voting} &  \multicolumn{2}{|c|}{Diffusion / Voting} \\ 
      \hline
            & Background  & Tract & Background  & Tract     & Background  & Tract \\
      \hline
Background  & 3658350     & 11849 & 3568123     & 24619     & 1.00        & 0.48   \\
      \hline
Tract       & 27932       & 2592  & 17050       & 3163      & 1.64        & 0.82   \\
\hline
\end{tabular}
\end{table}

Figure 1. Outline of our technique. We extract the main white-matter tracts using WMQL, register them to the 'test' subject and then compute a voting rule weighted by diffusion information. For each voxel $x$ in the 'test' subject, we select the label $l$ that maximizes equation \ref{eq:argmax}, where S is the set of 'train' subjects, $t$ is the 'test' subject; $L_i(x)$ is the label of voxel $x$ for the subject $i$; $P$ are the principal directions of diffusion in the 'test' subject and $D_{sl}(x)$ are the directions of tract $l$ in the voxel $x$ of the 'train' subject $s$.
