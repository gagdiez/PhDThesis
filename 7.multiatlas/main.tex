\chapter{Inferring the Localization of White-Matter Tracts using Diffusion Driven Label Fusion}

\section{Overview}
In the previous chapters we studied the structural organization of the brain,
and saw that it is highly related to function. Some white matter pathologies,
such as tumors or traumatic brain injury, affect the white matter to the point
were it's impossible to use tractography. Therefore, we cannot tell which
tracts are affected. Given the relationship between brain and function, it's
important to determinate which tracts were affected in order to prevent
which function will be affected.
In this chapter, we introduce a way to infer the location of pathways, even
when it's not possible to use tractography to locate them. Our technique is
based on a methodology named label fusion. In particular, we show how to add
dMRI information to the label fusion in order to better estimate the location
of white matter pathways.

This work was presented as part of OHBM 2018~\cite{Guillermo2018}.

\section{Introduction}
%

%%To reduce the impact of erroneous mapping of voxels, and consequent
%misparcellation  of  target  structures,  multi-atlas approaches have
%been postulated. Suppose that the hippocampus is defined in N different
%atlases and each atlas is warped to a  to-be-parcellated subject image,
%then the N definitions (labels) of the hippocampus are casted to the
%subject space, which can be fused  (a.k.a ‘‘label fusion’’) based upon
%pre-defined algorithms such as those proposed by [7,36–42]. If the mis-registration of each atlas causes random errors, the errors should be reduced by integrating N definitions.  

Pathologies such as brain tumors or edemas disrupt the structure of white
matter, resulting in cognitive deficits. Depending on the type and severity
of the pathology, fiber bundles can be displaced, infiltrated or directly
interrupted. Inferring which pathways are affected, or surround the tumor
is key for both pre and post-surgical planning. With this knowledge, neurosurgeons
attempt to remove as much damaged tissue as possible while minimizing damage to
functional white matter networks~\cite{McGirt2009}. Diffusion MRI allows to
non-invasively reconstruct white matter tracts by means of tractography. In
healthy brains, tractography is able to recover the major fiber bundles in the
brain~\cite{Catani2008}. However, in the presence of pathologies, tracking
through the white-matter becomes challenging.

Pictorial et al.\cite{Pictorial2004} identify four types of patterns found
in major bundles and diffusion information affected by the 
presence of tumors. The first pattern consists of normal Fractional Anisotropy
(FA), a measure derived from dMRI, with tract displacement. In this case, a
bundle is displaced by the tumor, without interrupting it. The second pattern
is substantially decreased FA with no tract displacement, this represents an
edema, and sometimes it is possible to track through it~\cite{Deslauriers-Gauthier2018}.
The third pattern is substantially decreased FA with some lost of directionality,
for which tracking becomes hard. Finally, the fourth pattern consists of
isotropic diffusion within the tumor, which causes a disruption of the tracts.
In this last case, it's possible to track until, and after the tumor, but not
within it.

The third and fourth patterns presented by Pictorial et al.\cite{Pictorial2004}
denote situations in which the tracking results in interrupted or erroneous
tracts. The incorrect shape of the resulting tracts, or its lack of continuity
makes hard to infer which pathways are directly affect by the pathology. In these
cases were it's not possible to use tractography, aggregating spatial information
from other subjects in order to infer the affected tracts could be a solution.
Assuming we defined major bundles in a group of healthy subjects, we could
register them to the brain of our patient and combine them using a label fusion
technique. Label fusion is a family of techniques which aim is to infer the
localization of a structure in a target subject, based on its
localization in a group of control subjects.

One well known label-fusion technique is Majority Voting~\cite{Xu1992}. Given a
voxel on a brain image, each subject is said to "vote" for a label. The resulting
voxel label will be that with the most votes. Majority Voting is simple to
implement and has been demonstrated to yield accurate segmentations~\cite{Asman2013}
in healthy subjects. However, this technique is blind both to registration problems
and anatomical variability between subjects. To overcome this, it has been propose
to weight the vote of each subject by how similar the subject looks to the
target~\cite{Sabuncu2010}. The underlying intuition is that the choosing of labels
should be driven by those subjects who resemble the most to the one being labeled.
The practical advantages of various strategies based on this idea have recently
been demonstrated by Artaechevarria et al.~\cite{Artaechevarria2009}.

Current label fusion techniques rely only on anatomical information, not taking 
into account the structure of white matter. In the case of white matter pathways,
the presence of a path constrains the diffusion of water particles, which can be
measured by dMRI. Therefore, adding diffusion information to the fusion algorithm
could help to better delineate fiber bundles. In this work, we introduce a new 
label fusion technique that, taking advantage of dMRI, weights the vote of each
subject based on how the voted pathway is supported by the test subject's diffusion
data. This is, if the diffusion data of the test subject is consistent with the
direction of the voted pathway, the vote has a higher weight. Our technique also
allows to work with crossing tracts, by modeling multiple labels per voxel.

We validate our technique in 13 subjects of the Human Connectome Project (HCP).
For each subject, we infer the location of 18 left hemisphere tracts using
whole-brain tractography and an implementation of the white matter query language
(WMQL). We use this results as ground truth to compare against inferring the
tracts from using the other 12 subjects with our proposed technique and Majority
Voting. Our results show that adding dMRI to the label fusion process achieves a
similar number of true positives than Majority Voting, with a 60\% less of
false positives, incurring in a trade-off of a 40\% false negatives.

This work is organized as follows: In the Methods section we make an introduction
to label fusion techniques and present how to extend them using diffusion
information. In the Experiments and Results section we present our results 
both on synthetic and on HCP data. We then discuss our results and position
ourselves with respect to the state of the art in the Discussion section. 
Finally, in the last section we provide our conclusions.

\section{Methods}
\label{sec:methods}

\subsection{Majority Voting}
Let $labels = \{l_i\}, \forall_i l_i \in N$ be the set of labels representing
tracts and grey matter structures in one hemisphere. Let ${L_s}, s \in S$
represent the labeling of a set of subjects $S$, where each 
$L_s \in labels^{v\times v}$ is a 3D volume representing the labeling of a
specific subject. Majority Voting~\cite{Rohlfing2004} infers the label of
each voxel $x$ in the test subject ($L(x)$) by computing:

\begin{equation}
\label{eq:mvoting}
\begin{aligned}
    \hat L(x) = \argmax_{l \in labels} \sum_{s\in S} p(L(x) = l | L_s(x)),\\
    \text{where} \\
    p(L(x) = l | L_s(x)) =
    \begin{cases}
        1,& \text{if } L_s(x) = l \\
        0,& \text{otherwise}
    \end{cases}
\end{aligned}
\end{equation}

In this case, it's said that each subject in the training set votes for a label
per voxel, and the label with the most amount of votes is assigned to the target
voxel.

\subsection{Diffusion Based Voting}

Our label fusion technique takes advantage of dMRI to weight the vote of each
subject. The vote for a specific label gets a higher weight if it's supported
by the diffusion data of the subject being labeled. In this section, we first
start explaining the concept of Orientation Density Function of diffusion in
dMRI and tract directionality. Then, we present how to use these concepts to
compute the weights of each vote.

\subsubsection{Fiber Orientation Density Function from dMRI Data.}
By fitting the diffusion information into a Constrained Spherical Deconvolution (CSD)
model, it's possible to estimate a fiber orientation density function\cite{Tournier2004} (fODF). 
The fiber ODF $F_x(\theta, \phi)$ represents the estimated fraction of fibers
within the voxel $x$ that are aligned along the direction $(\theta, \phi)$,
expressed in spherical coordinates.

\begin{figure*}[t]
    \includegraphics[width=\textwidth,height=150px]{missing}
    \caption{A. Estimation of tract directionality within a voxel. B. etc}
    \label{fig:weighted_diffusion}
\end{figure*}


\subsubsection{Fiber Orientation Density Function from Tractography.}
A tract can be described as a set of streamlines, were a streamline is a
discretrized 3-dimensional curve. Assuming that a streamline doesn't have sharp
turns, we can estimate its directionality within a voxel by looking at its
entry and exit points (Fig. \ref{fig:weighted_diffusion} A). Repeating this for each streamline
on a tract, we obtain a set of directional vectors, representing the directionality
of the tract within the voxel. A fiber ODF can be estimated from this set of
vectors by means of directional statistics. In this work, we use an
Angular Central Gaussian Distribution~\cite{Mardia1999} (ACGD). The ACGD models
antipodal symmetric directional data. It has a close form to estimate its
parameters, making it easier to use than other directional distributions on the
sphere~\cite{Mardia1999}.
    
% Since we also want to estimate directionality from tracts, we introduce the concept of acgd
\subsubsection{Label Fusion Weighted by Diffusion}
Majority Voting (Eq. \ref{eq:mvoting}) decides the label of a voxel based on
how many subjects 'vote' for it. Given that we are interested in inferring
white-matter paths, we introduce a weight to each vote driven by diffusion
information:

%We want to introduce a weight to these votes, taking into account how well the
%tract can explain the underlying Diffusion data of our test subject. In
%particular, we want to the Orientation Distribution Function (ODF) of our
%test subject's diffusion. This is, we want to profit of the fact that we know
%how the water particles are moving in its brain, following the underlying tracts.
%As explained in section X, in a given voxel, we can compute the main
%directionality of the tract $l$ of the train subject $s$ and the ODF of our
%test subject's diffusion. To see if the tract are aligned with the
%directionality of the voted tract we proposed the following model:

\begin{equation}
\label{eq:mvoting_weighted}
\hat L(x) = \argmax_{l \in labels} \sum_{s\in S} p(L(x) = l ; L_s(x)) p(D(x) ; D_{sl}(x)).
\end{equation}

In our segmentation scheme, the term $p(L(x) = l ; L_s(x))$ is modeled in the same
way as in the voting scheme (eq. \ref{eq:mvoting}). Our second term,
$p(D(x) | D_{sl}(x))$ express the probability of seeing the diffusion of our
target subject ($D(x)$) on voxel $x$, given that the tract $l$ is passing through
it with a shape as in subject $s$ ($D_{sl}(x)$). Intuitively, if subject $s$ votes
for the tract $l$, the weight represents how much they tract $l$ in voxel $x$
assembles the diffusion of the target subject in the same voxel. We model this
second term as:

\begin{equation}
\label{eq:inner_odf}
\begin{aligned}
    p(D(x) ; D_{sl}(x)) = 
    \begin{cases}
        <F(x), F_{sl}(x)>,& \text{if } L_s(x) = l,\\
                        & \text{and } l \neq 0 \\
        <F(x), U>,& \text{if } L_s(x) = 0 \\
        0,& \text{otherwise}
    \end{cases} \\
    \text{where} \\
    <F(x),F(x)> = 1,\\ < F_{sl}(x), F_{sl}(x)>=1.
\end{aligned}
\end{equation}

In our model, $F(x)$ is the fiber ODF on voxel $x$ estimated from the diffusion
of the target by means of CSD. It's normalized in such a way that the inner
product with itself results in 1. $F_{sl}(x)$ is the fiber ODF of the tract $l$
registered from subject $s$, estimated by means of ACGD. $U$ is a uniformly
distributed fiber ODF, this is the diffusion assumed for either the label 
no-tract, representing the background or a gray matter structure.
By computing the inner product between ODFs, we can estimate how much they look
alike. This allow us to weight the votes of each subject, while accounting for
the white-matter structure of both the voting subjects and the target.

\section{Experiments and Results}

In section \ref{sec:methods} we presented how to add diffusion information to
Majority Voting (Eq. \ref{eq:mvoting_weighted}). This allow us to weight each
vote by how much the diffusion of the target supports it. Now, we present
experiments both in phantoms and subjects from the Human Connectome 
Project (HCP). We start by assessing that the computed weights correctly reflect
the diffusion information in dwi phantoms. Then, we proceed to infer the localization
of white-matter pathways in subjects of the HCP, and compare them with their
truth shape. Finally, we simulate lesions in the white-matter and test the
performance of our method.

\subsection{Data and Preprocessing}
We created three types of diffusion weighted image phantoms using Phantomas~\cite{Caruyer2014}.
The first phantom possess only one tract, traveling from one side to the other of the
image horizontally (Fig. \ref{fig:pha_exp_1} A). The second possess two crossing tracts,
forming a 90 degrees angle between them (Fig. \ref{fig:pha_exp_1} B). The last, has no
fibers, and represents isotropic diffusion (Fig. \ref{fig:pha_exp_1} C). From
each one of them, we generated 31 DWIs. All the DWIs were generated using a
Signal to Noise Ratio of 20, and a resolution of $1mm$ per voxel. The final
images are 3-dimensional matrices, with 10 voxels in each dimension. Having
such small images, allows us to test how our label fusion technique behaves
on a controlled environment.

To test our technique in more realistic scenarios, we randomly selected 9 subjects
from the HCP500 dataset from the Human Connectome Project. For each subject,
we computed whole-brain tractography using each voxel in the white-matter as
a seed and simulating 8 particles per seed~\cite{Garyfallidis2014}. We extracted
the main tracts from the left hemisphere tractogram (13 tracts in total) using
the implementation of the white-matter query language (WMQL)~\cite{Wassermann2016}.
For each subject we computed registrations to the rest using as reference their
T1w images~\cite{Jenkinson2012}. Using the resulting warp transformations, we
registered the tracts between every pair of subjects.

For both the phantom DWIs and the HCP subjects, we estimated their fiber ODFs
using the implementation of CSD in Dipy~\cite{Garyfallidis2014}. The ODFs were
discretize on a sphere with $n=100$ vertices. A uniform distribution over the sphere
was created by assigning to each vertex the value $\sqrt(n)/n$, making $<U, U> = 1$.

\begin{figure*}[h]
    \includegraphics[width=\textwidth, height=150px]{missing.png}
    \caption{Phantoms created to test how our technique weights the votes.
             (A) Only one tract in the white matter (B) No tracts, free diffusion.}
    \label{fig:pha_exp_1}
\end{figure*}

\subsection{Assessing the Correctness of Voting Weights in Synthetic Data}
In order to study how the tract's directionality influences its vote weight,
we started by recreating the tracts present in the first phantom (Fig. \ref{fig:pha_exp_1} A).
For this, we took one of the 31 DWIs, and computed 1000 streamlines by means of 
probabilistic tracking from the voxels in which the tract passes. In the DWIs
generated from the first phantom, this tract is the one that should generate the
highest weight. At the same time, any change in its directionality, should decrease
the weight. To test this, we computed the weights obtained by the estimated tract
and planar rotations of it around the central voxel in the remaining 30 DWIs.
Figure \ref{fig:pha_exp_1} A shows the obtained weights on the first phantom.
Effectively, the weight starts to rapidly decrease as the angle increments and 
the directionality of the tract moves away from that of the diffusion.
Figure \ref{fig:pha_exp_1} B shows the weights obtained when repeating the 
experiment using 30 DWIs of the second phantom (Fig. \ref{fig:pha_exp_1} B),
which have a crossing of fibers in the central voxel. In this case, the weight
is higher when the tract is aligned to one of the crossing fibers (at 0 degrees
or 90 degress), while rapidly decaying in between them. Finally,
(Fig. \ref{fig:pha_exp_1} C) shows the weights when using 30 DWIs with
isotropic diffusion (Fig. \ref{fig:pha_exp_1} C). In this case, the weight
is always low, driven by the discrepancy between the directionality of the tract
and the free diffusion present in the DWIs.

To assess that the proposed model is not overweighting tracts, we also computed
the weight that a 'non-tract' label would receive in each of the phantoms. As
explained in section \ref{sec:methods}, equation \ref{eq:inner_odf}, when a
subject is voting for a non-tract label, a uniform fiber ODF is compared against
the diffusion ODF. The three subfigures composing figure \ref{fig:pha_exp_1}
also show the weight obtained in the central voxel when a subject is voting for
the label 'no-tract'. In figure \ref{fig:pha_exp_1} A, we can see that the
weight of 'no-tract' is low, specially when compared with the high weight
of the correctly aligned tracts (low angle rotations). This is driven by the
highly directional underlying diffusion data of the phantom. 
Figure \ref{fig:pha_exp_1} B shows that the weight of a 'no-tract' vote is
similar to that of an aligned tract. In this case, the underlying crossing
diffusion matches better with the uniform ODF. Finally, figure \ref{fig:pha_exp_1} C
shows always a higher weight for the 'non-tract' than for any tract, consistent
with the isotropic diffusion of our third phantom.


%Since our method is based on a weighted majority voting, the final result of
%the parcellation will depend not only on the weights, but also on the amount of
%subjects voting for each structure.
%Having the ability to compute the weights received by a structure allow us to
%examine what would happen in particular scenarios. Lets assume that only two types
%of subjects are voting, one type of subjects votes for a specific tract, and
%the other type of subjects votes for 'no tract'. The dotted lines in figure
%\ref{fig:pha_exp_1} show which proportion of subjects are needed to vote
%for the 'tract' in order for it to appear in the final segmentation. For
%example, in the phantom with one tract, the weight for the 'tract' with 0 degrees
%is such that only a 30\% of the subjects are needed to vote for it. Meanwhile,
%for tracts at 90 degrees, 70\% of the subjects are needed to vote for it. 
%In the phantom case when there's only isotropic DWI, we always need more than
%50\% of the subjects. These results show a good behaviour of our technique. 

\subsubsection{Assessing goodness of the techniques in real data}
To assess the "goodness" of our technique we made a leave-one-out
cross-validation. At each step, we selected one of the subjects as test and 
used the rest as train subjects. Using the registered tracts of the train
subjects, we computed parcellations using both the voting rule and our technique.
Since we also have the tracts of each test subject, we compute a 'ground truth'
parcellation of they white matter. Finally, we computed the confusion matrix of
both the Majority Voting and our technique. A confusion matrix is a matrix of 
size labels by labels where the entry (i,j) is the number of times the label in
the ground truth was i and the technique labeled j. Table 1 shows that our
proposed achieves a similar number of false negatives while obtaining a 64\%
less of false positives in average. This incurs on a trade-off of having 39\%
more false negatives and 18\% less true positives in average, underlying that
our technique a more conservative.\\

%\subsection{Simulating Lesions in the White-Matter}
%To test how our technique would behave in a injured brain, we simulate a tumor in the white matter of one of our subjects. We place the tumor to interrupt the ILF. We do so by selecting a set of neighboring voxels where the ILF passes by, and mix their signal with diffusion signal from random voxels in the ventricles. This is, for each selected voxel $x$ in the brain, we chose a random voxel $v$ in the ventricle and mix their signals:
%$$S(x) = S(x)(1-\alpha) + S(v)\alpha, \alpha \in [0,1]$$.
%Since the ventricles are regions filled with cerebrospinal fluid (CSF), they diffusion is approximately isotropic. We compute for which values of $\alpha$ WMQL stops being able to detect the ILF, and for which values our technique stops being able to detecting it. Since the Major Voting does not rely on diffusion data, this experiment does not affect the results obtained in the previous experiment for Major Voting.
%WMQL stops detecting the ILF at... and our technique stops detecting it in the voxels at .... 

\subsection{Discussion}
We want to infer the position of tracts in the white matter.
Since we do not have many subjects, we cannot relay on Deep Learning techniques.
We decide to use Label Fusion techniques.
Some people showed that Mayority Voting works well for this things.
The problem is that majority voting relies only on the spatial location.
Problems of registration.
We introduce a new way to take into account the diffusion information.
This makes a lot of sense, since we are trying to infer tracts, and the 
diffusion data is related to the underlying tracts.
We avoid registering diffusion, but we still have the white matter info.
In synthetic data, we show that our technique is much better than
majority voting, specially when the tracts align correctly with the
underlying diffusion. It also works like charm, when the tracts are completely
missaligned with the diffusion. In other cases, is as good as majority voting.
In real data, this is reflected, by showing that diffusion voting is more
conservative. WMM lessions.
Does this actually works? Registration is difficult around tumors.


The election of an uniform distribution comes from the fact that, in order to
be labeled, you should look closet to them than the free diffusion case.


\subsubsection{Our Technique Creates Weights Consistent With the Underlying
               Diffusion Data.}
In order to study how the tract's directionality influences its vote weight,
we created three different phantoms, and DWIs from them. Taking profit of 
these DWIs, we computed the weight of tracts with different directions. 
The weights obtained in figure \ref{fig:pha_exp_1} A show that, tracts
with a directionality similar to the underlying diffusion get higher votes.
But, the more we rotate the tract respect to the central voxel, the lower the
weight it obtains. In particular, after reaching an angle of 20 degrees, the
weight starts to drop rapidly, falling bellow the weight of the 'non-tract' label.
The Figure \ref{fig:pha_exp_1} B shows that the tracts aligned with one of the 
crossing fibers get weights similar to 'non-label'. In this case, if the tract
is roughly aligned with the underlying diffusion, then it will compete with
equal weights against the background. Finally, Figure \ref{fig:pha_exp_1} C
shows that, when there's no underlying white-matter structure in the DWI image,
then the label 'no-trac' is the one that receives the highest weight. These
results show that our technique is able to correctly weight each label based
on white-matter features, as diffusion directionality.


\subsection{Conclusions}
In this chapter we presented a labeling fusion technique that relies on dMRI
data to infer the localization of white-matter tracts. The results show that
our technique is more conservative than the voting rule, which is desired when
studying pathologies, at the cost of having more false negatives.

\subsubsection*{Table}

\begin{table}[]
\centering
\caption{Confusion matrix for both techniques and the ratio between them}
\label{my-label}
\begin{tabular}{|l||l|l||l|l||l|l|}
\hline
      & \multicolumn{2}{c||}{Diffusion} & \multicolumn{2}{c||}{Voting} &  \multicolumn{2}{|c|}{Diffusion / Voting} \\ 
      \hline
            & Background  & Tract & Background  & Tract     & Background  & Tract \\
      \hline
Background  & 3658350     & 11849 & 3568123     & 24619     & 1.00        & 0.48   \\
      \hline
Tract       & 27932       & 2592  & 17050       & 3163      & 1.64        & 0.82   \\
\hline
\end{tabular}
\end{table}

Figure 1. Outline of our technique. We extract the main white-matter tracts using WMQL, register them to the 'test' subject and then compute a voting rule weighted by diffusion information. For each voxel $x$ in the 'test' subject, we select the label $l$ that maximizes equation \ref{eq:argmax}, where S is the set of 'train' subjects, $t$ is the 'test' subject; $L_i(x)$ is the label of voxel $x$ for the subject $i$; $P$ are the principal directions of diffusion in the 'test' subject and $D_{sl}(x)$ are the directions of tract $l$ in the voxel $x$ of the 'train' subject $s$.

\chapterbib
