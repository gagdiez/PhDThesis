\chapter{Inferring the Localization of White-Matter Tracts using Diffusion Driven Label Fusion}
\label{ch:multiatlas}

\section{Overview}
In the previous chapters we studied the structural organization of the brain,
and saw that it is highly related to function. White-matter pathologies
disrupt the white-matter organization rebounding in brain function. When
treating such pathologies, it is of great importance to infer which pathways are
affected. However, sometimes the white-matter lesions hamper the use of
tractography to infer fiber bundles. 
In this chapter, we introduce a way to infer the location of white-matter pathways, even
when it is not possible to use tractography to locate them. Our technique is
based on a methodology named label fusion. Particularly, we show how to add
dMRI information to the label fusion in order to better estimate the location
of white matter pathways.

This work was presented as part of OHBM 2018~\cite{Guillermo2018}.

\section{Introduction}
Pathologies such as traumatic brain injury, brain tumors; or edemas disrupt
the structure of white matter, resulting in cognitive deficits. Depending on
the type and severity of the pathology, fiber bundles can be displaced,
infiltrated or directly interrupted~\cite{Schonberg2006, Huisman2009, Won2016}.
Inferring which pathways are closely located to the lesion or being directly
affected by it is key for palliative care, and for both pre and post-treatment planning.
With this knowledge, neurologists and neurosurgeons can decide if a lesion should be
treated more aggressively or  conservatively~\cite{Huisman2009, McGirt2009}.
In healthy brains, tractography allows to non-invasively reconstruct the major
white matter pathways in the brain~\cite{Catani2008}. However, in the presence of 
pathologies, tracking through the white-matter becomes challenging~\cite{Pictorial2004}.

Four pattern types associated to brain pathologies can be identified in major
tracts and Diffusion Weighted Images (DWIs) ~\cite{Pictorial2004}. The first
pattern consists of unchanged Fractional Anisotropy (FA)~\cite{Basser1996}, and
tract displacement~\cite{Pictorial2004}. In this case, a
bundle is displaced by the tumor, without damaging it. The second pattern
is substantially decreased FA with no tract displacement, this could be
an indicator of edema or small tumor\cite{Schonberg2006, Huisman2009}.
The third pattern is substantially decreased FA with some loss of directionality,
which makes tracking hard~\cite{Schonberg2006, Pictorial2004}. Finally,
the fourth pattern consists of isotropic diffusion within the area affected by
the pathology, which causes a disruption of the tracts~\cite{Pictorial2004}.
The last two patterns denote situations in which the tracking results in
interrupted or distorted tracts. This hampers the inference of which pathways
are directly affected by pathology.

When tracking is not possible, aggregating anatomical information
from other subjects could help to infer tracts affected by pathology.
Assuming we identified major bundles in a group of healthy subjects, we could
register them to our patient's brain, and combine them using a label fusion
technique. Label fusion is a family of techniques which aim is to infer the
localization of a structure in a target subject, based on its characteristics
in a group of control subjects\cite{Asman2013}.

\begin{figure*}[t]
    \includegraphics[width=\textwidth]{7.multiatlas/img/diagram.png}
    \caption{We use a label fusion technique to infer the location of white
             matter bundles by aggregating information of healthy subjects.
             After registering the tracts, each subject "votes" for either
             a tract or a non-tract structure (gray matter or background).
             Our technique adds diffusion based weights to each one of these votes.
             The weights are computed based on the similarity between the fiber
             Orientation Distribution Function of the structure being voted
             and the DWI of the target subject.}
    \label{fig:weighted_diffusion}
\end{figure*}

One well known label-fusion technique is Majority Voting~\cite{Xu1992}. Given a
voxel on a brain image, each template subject "votes" for a label. The resulting
label for the voxel will be that with the most votes. Majority Voting is simple to
implement and has been shown to generate accurate segmentations~\cite{Asman2013},
even when using few template subjects to perform the inference. However,
this technique is blind both to registration problems and anatomical variability
between subjects. To overcome this, it has been proposed
to weigh the vote of each subject based on a similarity measure between the
template subject and the target~\cite{Sabuncu2010}. The underlying intuition is
that the label choice should be driven by those subjects who resemble the most
to the one being labeled. The practical advantages of various strategies based
on this idea have been demonstrated by Artaechevarria et al.~\cite{Artaechevarria2009}.

Current label fusion techniques rely only on anatomical information, not taking 
into account the structure of white matter. In the case of white matter pathways,
the presence of a path constrains the diffusion of water particles, which can be
measured by dMRI. Therefore, adding diffusion information to the fusion algorithm
could help to better delineate fiber bundles~\cite{Girard2017}.

In this work, we introduce a label fusion technique that takes advantage of dMRI
and weights the vote of each subject based on how the voted pathway is supported
by the target subject's diffusion data (Fig. \ref{fig:weighted_diffusion}. Hence,
if the diffusion data of the target subject is consistent with the direction of
the voted pathway, the vote has a higher weight.

We first validate our technique with synthetic DWIs. Using Phantomas~\cite{Caruyer2014},
we generate synthetic DWIs (phantoms) with different white-matter structures. On each phantom
we simulate the votes of different subjects, and assess that they receive an
appropriate weight. Hence, our votes for structures consistent with the
phantom diffusion should obtain a higher weight. 

Then, we randomly select 10 subjects from the Human Connectome Project (HCP).
On each subject, we infer the shape and location of 4 left hemisphere tracts
using whole-brain tractography and an implementation of
the white matter query language (WMQL). We use these results as ground truth to
benchmark the inferences made by our technique and Majority Voting on a leave-one-out
cross-validation experiment. Our results show that our proposed technique has a lower
sensitivity than Majority Voting, but a higher precision. This is, our technique
is able to give a more trustable labeling, incurring in the trade-off of labeling
less voxels.

Finally, we simulate lesions with different degrees of severity in the white
matter of one HCP subject. Particularly, we target a spherical region where
the Superior Longitudinal Fasciculus passes by. We do so by iteratively increasing
the FA of each voxel in the spherical region, until obtaining isotropic diffusion.
We show that our technique labels less voxels as the FA decreases, but is still
able to label voxels around the pathology. 

\section{Methods}
\label{sec:methods}

\subsection{Majority Voting.}
Let $labels = \{l_i\}$ be the set of labels representing tracts and grey matter
structures the brain. Let ${L_s}, s\in S$ represent the labeling of a set of
template subjects $S$, where each $L_s \in labels^{v_x\times v_y \times v_z}$ is
a 3D volume with dimension $(v_x,v_y,v_z)$ representing the labeling of template
subject s. Majority Voting~\cite{Rohlfing2004} infers the label of
each voxel $x$ in a target subject by computing:

\begin{equation}
\label{eq:mvoting}
\begin{aligned}
    L^*(x) = \argmax_{l \in labels} \sum_{s\in S} p(L(x) = l | L_s(x)),\\
    \text{where} \\
    p(L(x) = l | L_s(x)) =
    \begin{cases}
        1,& \text{if } L_s(x) = l \\
        0,& \text{otherwise.}
    \end{cases}
\end{aligned}
\end{equation}

In this case, it's said that each subject votes for a label, and the label with
the most amount of votes is assigned to the target voxel. It is important to
notice that no information from the target subject is being used to infer the
label $L^*(x)$.

\subsection{Diffusion Based Voting}
Majority Voting (Eq. \ref{eq:mvoting}) decides the label of a target voxel based
on the 'votes' of template subjects, without using any information from the
target subject. The aim of this work is to infer white matter pathways. Knowing
that water particles diffuse along pathways, we can profit of diffusion information
to weigh the voting process. In particular, votes for tracts aligned with the
diffusion of the target subject should get higher weights.

One way to characterize the directionality on a diffusion weighted image (DWI) is
by means of fiber orientation distribution functions (fODFs). Now, we will first
explain how to estimate a fODF from both DWI and tracts. Then, we will present
how to compare them, in order to compute weights for each vote.

\subsubsection{Fiber Orientation Density Function from dMRI Data.}
By fitting the diffusion information into a Constrained Spherical Deconvolution (CSD)
model, it's possible to estimate a fiber orientation density function\cite{Tournier2004} (fODF). 
The fiber ODF $F_x(\theta, \phi)$ represents the estimated fraction of fibers
within the voxel $x$ that are aligned along the direction $(\theta, \phi)$,
expressed in spherical coordinates.

\begin{figure*}[t!]
    \includegraphics[height=150px]{7.multiatlas/img/phantoms.png}
    \caption{In order to test our technique, we created three types of synthetic DWIs, known as phantoms.
             (A) A phantom with only one tract in the white matter. (B) A phantom with two fibers crossing.
             (C) A phantom with no tracts, representing isotropic diffusion.}
    \label{fig:pha_exp_1}
\end{figure*}

\subsubsection{Fiber Orientation Density Function from Tractography.}
A tract can be described as a set of streamlines, where a streamline is a
discretrized 3-dimensional curve. Assuming that a streamline doesn't have sharp
turns, we can estimate its directionality within a voxel by looking at its
entry and exit points (Fig. \ref{fig:weighted_diffusion} A). Repeating this for each streamline
on a tract, we obtain a set of directional vectors, representing the directionality
of the tract within the voxel. A fiber ODF can be estimated from this set of
vectors by means of directional statistics. In this work, we use the
Angular Central Gaussian Distribution~\cite{Mardia1999} (ACGD). The ACGD models
antipodal symmetric directional data, and has close forms to estimate its
parameters, making it both suitable and easy to estimate the directionality of
a tract in a voxel~\cite{Mardia1999}.

   
% Since we also want to estimate directionality from tracts, we introduce the concept of acgd
\subsubsection{Label Fusion Weighted by Diffusion}
Majority Voting (Eq. \ref{eq:mvoting}) decides the label of a voxel based on
how many subjects ``vote'' for it. Given that we are inferring brain pathways,
we want to introduce a weight that denotes how much the voted tract resembles
the target's diffusion data: 

\begin{equation}
\label{eq:mvoting_weighted}
L^*(x) = \argmax_{l \in labels} \sum_{s\in S} p(L(x) = l | L_s(x)) p(D(x) | D_{sl}(x)).
\end{equation}

In our segmentation scheme, the term $p(L(x) = l | L_s(x))$ is modeled as in
the voting scheme (eq. \ref{eq:mvoting}). Our second term, $p(D(x) | D_{sl}(x))$
express the probability of seeing the diffusion of our target subject, $D(x)$,
on voxel $x$, given the diffusion of subject $s$ generated by tract $l$ on
the same voxel, $D_{sl}(x)$. Since registering DWIs is a highly time consuming
task~\cite{ODonnell2017}, we want to avoid it. Instead, we can register tracts,
for which efficient algorithms exist, and use them as an estimator of the diffusion
of each template subject. Knowing that water particles in the brain diffuse along
tracts, we can estimate $D_{sl}(x)$ by computing the fODF of the registered tract
$l$. Simultaneously, we can characterize $D(x)$ with the fODF computed from
the DWI of the target subject. In order to reflect how much the fODF of
our target subject's diffusion resembles the fODF of the voted tract on a voxel,
we model $p(D(x) | D_{sl}(x))$ as:

\begin{equation}
\label{eq:inner_odf}
\begin{aligned}
    p(D(x) | D_{sl}(x)) = 
    \begin{cases}
        \langle F(x), F_{sl}(x) \rangle,& \text{if } L_s(x) = l,\\
                        & \text{and } l \neq 0 \\
        \langle F(x), U \rangle,& \text{if } L_s(x) = 0 \\
        0,& \text{otherwise}
    \end{cases} \\
\end{aligned}
\end{equation}

where $F(x)$ is the fiber ODF on voxel $x$ estimated from the target's DWI by
means o CSD, and is normalized such that $\langle F(x),F(x) \rangle = 1$.
$F_{sl}(x)$ is the fiber ODF of the tract $l$ registered from subject $s$,
estimated by means of ACGD, and normalized as $F(x)$. $U$ is a uniformly
distributed fiber ODF, this is the diffusion assumed for either the label 
no-tract, representing the background or a gray matter structure.
By computing the inner product between normalized ODFs, we can estimate how much
they look alike. In this way, we weight the vote of each subject accounting for
the white-matter structure of both the voting and target subjects.

\section{Experiments and Results}
In section \ref{sec:methods} we presented how to add diffusion information to
Majority Voting (Eq. \ref{eq:mvoting_weighted}). This allows us to weigh the
vote for a tract by how much the diffusion of the target supports it. Now, we present
experiments both in synthetic data and subjects from the Human Connectome 
Project (HCP). We start by assessing that the computed weights correctly reflect
the diffusion information in DWI phantoms. Then, we proceed to infer the location
of white-matter pathways in subjects of the HCP, and compare them with their
true shape. Finally, we simulate lesions in the white matter and test how our
method behaves in their presence.

\subsection{Data and Preprocessing}
We created three types of diffusion weighted image phantoms using Phantomas~\cite{Caruyer2014}.
The first phantom possess only one tract, traveling from one side to the other of the
image horizontally (Fig. \ref{fig:pha_exp_1} A). The second possess two crossing tracts,
forming a 90 degrees angle between them (Fig. \ref{fig:pha_exp_1} B). The last, has no
fibers, and represents isotropic diffusion (Fig. \ref{fig:pha_exp_1} C). From
each one of them, we generated 31 DWIs. All the DWIs were generated using a
Signal to Noise Ratio of 20, and a resolution of $1mm$ per voxel. The final
images are 3-dimensional matrices, with 10 voxels in each dimension. Having
such small images, allows us to test how our label fusion technique behaves
on a controlled environment.

To test our technique in more realistic scenarios, we randomly selected 10 subjects
from the HCP500 dataset from the Human Connectome Project. For each subject,
we computed whole-brain tractography using each voxel in the white-matter as
a seed and simulating 8 particles per seed~\cite{Garyfallidis2014}. We extracted
the main tracts 4 main tracts from the left hemisphere using
the implementation of the white-matter query language (WMQL)~\cite{Wassermann2016}.
For each subject we computed non linear registrations to the rest using as
reference their T1w images~\cite{Jenkinson2012}. Using the resulting warp
transformations, we registered the tracts between every pair of subjects.

We estimated fiber ODFs in each voxels of the phantom DWIs, and in the DWI data
of the HCP subjects by means of CSD using Dipy~\cite{Garyfallidis2014}. The fODFs were
discretrized on a sphere with $n=100$ vertices. A uniform fODF, $U$, was created
by assigning to each vertex of the sphere the value $\sqrt(n)/n$, making $<U, U> = 1$.

\begin{figure*}[t]
    \includegraphics[width=\textwidth]{7.multiatlas/img/weights.png}
    \caption{In order to study how a tract's directionality influences its vote weight,
             we estimated the fiber bundle by means of tractography in the Phantom A.
             Then, we computed the weights obtained by the tract and planar rotations of
             it in: (A) Phantom A. (B) Phantom B. (C) Phantom C. The figures show that
             our technique gives the highest weights to structures that are aligned
             with the underlying diffusion.}
    \label{fig:weights}
\end{figure*} 

\subsection{Assessing the Correctness of Voting Weights in Synthetic Data}
In order to study how the tract's directionality influences its vote weight,
we started by reconstructing the tract present in the first phantom (Fig. \ref{fig:pha_exp_1} A).
For this, we took one of the 31 generated DWIs, and computed 1000 streamlines by
means of probabilistic tracking from the voxels in which the tract passes.

In the DWIs generated from the first phantom, the reconstructed tract is the one
that should generate the highest weight. At the same time, any change in its
directionality, should decrease the received weight. To assess this was happening,
we computed weights in 30 DWIs for the reconstructed tract, and for planar
rotations of it around the central voxel. Figure \ref{fig:weights} A shows the
obtained weights on the first phantom. Effectively, the weight starts to rapidly
decrease as the angle increments and the directionality of the tract moves away
from that of the diffusion. Figure \ref{fig:weights} B shows the weights obtained
when computing weights of the reconstructed tract in 30 DWIs derived from the
second phantom (Fig. \ref{fig:pha_exp_1} B), which have a crossing of fibers in
the central voxel. In this case, the weight is higher when the reconstructed tract
aligns with one of the crossing fibers (at 0 degrees or 90 degrees), while rapidly
decaying in between them. Finally, figure \ref{fig:weights} C shows the weights
when using 30 DWIs with isotropic diffusion (Fig. \ref{fig:pha_exp_1} C). In this
case, the weight is always low, driven by the discrepancy between the
directionality of the tract and the free diffusion present in the DWIs.

To assess that the proposed model is not overweighting tracts, we also computed
the weight that a ``non-tract'' label would receive in each of the phantoms. As
explained in section \ref{sec:methods}, equation \ref{eq:inner_odf}, when a
subject is voting for a non-tract label, a uniform fiber ODF is compared against
the diffusion fODF. Figure \ref{fig:pha_exp_1} A, B, and C show the weight
obtained in the central voxel when a subject is voting for the label 'no-tract'.
In figure \ref{fig:pha_exp_1} A, we can see that the
weight of ``non-tract'' is low, specially when compared with the high weight
of the correctly aligned tracts (low angle rotations). This is driven by the
highly directional underlying diffusion data of the phantom. 
Figure \ref{fig:pha_exp_1} B shows that the weight of a 'no-tract' vote is
similar to that of an aligned tract. Finally, figure \ref{fig:pha_exp_1} C
shows always a higher weight for the 'non-tract' than for any tract, consistent
with the isotropic diffusion of our third phantom.

\subsubsection{Inferring Tracts in Human Connectome Project Subjects}
To validate our technique in a more realistic but yet controlled scenario, we
inferred single tracts in the HCP subjects. We selected the following tracts to
work with: inferior part of the Superior Longitudinal Fasciculus (SLF1),
Inferior Longitudinal Fasciculus (ILF), middle part of the
Corpus Callosum (CC2), and External Capsule (EC). These four tracts provide
a fair diversity of directionality, shape, and position in the brain.

For each tract we performed a leave-one-out cross-validation. At each step, we
inferred the tract of one subject from the registered tracts of the others
using both Majority Voting and our technique. Using as 'ground truth' the
target's bundle computed by means of tractography and WMQL, we quantified the
performance of both techniques. In particular, we computed their confusion matrix.
A confusion matrix is a matrix $M \in \R^{2\times2}$, where each entry $M_{ij}$
represents the number of times the label in the ground truth was i and the
technique labeled j. Finally, we computed the sensitivity, and precision on
each confusion matrix~\cite{Kuhn2013}. Sensitivity measures the proportion of voxels
in the ground-truth tract that were 'discovered'. Precision measures the proportion
of voxels that were correctly labeled, over all the labeled voxels. Table 1 shows
the results obtained for each technique and tract. In all of the tracts, our
technique achieves a lower sensitivity than Majority Voting. This means that
we label a smaller portion of the ground-truth bundle. On the other hand,
our diffusion weighted label-fusion always achieve a higher precision. This
is, if we only look at the labels created by the techniques, our technique
has the highest proportion of correct ones. Another way to phrase it is, our
technique has a lower number of false positives.
Therefore, our technique is discovering less voxels, but those which are labeled
can be trusted more.

\begin{figure*}[t]
    \includegraphics[width=\textwidth]{7.multiatlas/img/pathology.png}
    \caption{We tested how our technique behaves in the presence of simulated lesions.
             Lesions were simulated in a specific region (red circle) by following
             eq. \ref{eq:mixing} to lower the FA of the region. In such region,
             the SLF passes by. The figures show the result of applying our technique 
             in order to infer the SLF at different values of the parameter $\alpha$ in
             eq. \ref{eq:mixing}: (A) $\alpha=0.2$, (B) $\alpha=0.5$, (C) $\alpha=0.75$, and (D) $\alpha=1$.
             Results show that while the value of $\alpha$ increases, the amount of voxels
             labeled within the affected region decreases.}
    \label{fig:labeling}
\end{figure*}

\subsubsection{Inferring Tracts in the Presence of Simulated Lesions}
To test how our technique behaves on an injured brain, we simulated
lesions at different degrees of severity in the white matter of one of our subjects.
Given that some brain lesions directly affect FA~\cite{Schonberg2006, Huisman2009},
we simulated lesions by adding isotropic signal to a set of voxels, therefore
lowering their FA. We targeted the SLF bundle, in order to compare how the labeling
changes with lesion of different degrees. We did so by selecting a spherical region
of $4mm$ where the SIF passes by, and mixing the diffusion signal there with signal
from the ventricles. Since the ventricles are regions filled with cerebrospinal
fluid (CSF), their diffusion is approximately isotropic. In particular, for each
voxel $x$ in the lesioned region, we chose a voxel $v$ in the ventricle and mix
their signals, $S(\cdot)$, as follows:

\begin{equation}
    \label{eq:mixing}
S(x) = S(x)(1-\alpha) + S(v)\alpha, \alpha \in [0,1],
\end{equation}

where $\alpha$ manages the severity of the lesion. In this case, $\alpha=0$
represents healthy tissue, and $alpha=1$ represents a total disruption of
the white-matter, resulting in pure isotropic diffusion. Figure \ref{fig:labeling}
shows that, at higher alpha (lower FA), less voxels are labeled within the lesion.
This is a good behaviour, since by lowering the FA we make the diffusion more
isotropic, loosing the underlying tract. In particular, for $\alpha=1$, the
diffusion is completely isotropic, meaning that there's no tract, therefore,
it's correct to not label it. Since our technique still labels the surroundings
of the pathology, it allows to correctly identify the affected tract.

\section{Discussion}
In this work we presented a label-fusion techniques to infer white-matter
structures in the brain. An advantage of label-fusion techniques is that
they can achieve accurate segmentations even when the inference is
made from few subjects~\cite{Asman2013}. In particular, our technique allows to
infer white-matter pathways without the need of tracking. This is of specially
importance in the presence of brain pathologies that show no deformations. In the case of having
deformations, then the registration between subjects, in which label-fusion
techniques heavily rely, becomes hard. The main contribution of this work
is to add diffusion information in the process of label-fusion. Given that fiber
bundles constrains the diffusion of
water particles in the brain, our technique uses diffusion information to improve
Majority Voting~\cite{Xu1992}. More specifically, we weight each vote based on how
the voted pathway is supported by the target's diffusion data. In this way, voted
pathways that better resemble the white matter of the target subject obtain a 
higher weight. The weights come from comparing how much the diffusion fODF of our
target subject's resembles the fODF of the voted tract on a voxel. In this way, we
can compare the white-matter structure of our target subject with that of the voting
subject, without having to register DWIs. Adding diffusion weights to Majority Voting,
allowed us to profit of its robustness while improving the labeling of white-matter
bundles, as shown by our results in synthetic data and subjects from the Human
Connectome Project.

\subsection{Our Technique Creates Weights Consistent With the Underlying
               Diffusion Data.}
In order to study how the tract's directionality influences its vote weight,
we created three different phantoms, and derived DWIs from them. This allowed us to
create a controlled environment, in which we could study how the directionality
of a tract affects its vote weight. Since weight variability can
come from changes on the directionality of the tract, we generated many DWIs
for each phantom. At the same time, we computed weights for tracts in different
directions. Figure \ref{fig:pha_exp_1} shows the weights obtained on a phantom
with a single bundle. The weights show that tracts with a directionality
similar to the underlying diffusion get higher votes. However, when the alignment
start to decrease, also does the weight. Particularly, when the directions
differ by more than 20 degrees, the weight starts to drop rapidly, falling bellow the weight of the
'non-tract' label.
The Figure \ref{fig:pha_exp_1} B shows that, in the presence of crossing fibers
in the DWI, tracts aligned with a crossing fibers get the highest weights.
It is important to notice that the highest weights are similar to those of the
'non-label' votes. This means, that tracts roughly aligned with one of the
crossing fibers will compete with equal weights against the 'non-label' structures.
Finally, Figure \ref{fig:pha_exp_1} C shows that, when there's no underlying
white-matter structure in the DWI image,
then the label 'no-tract' is the one that receives the highest weight. These
results show that our technique is able to correctly weight each label based
on diffusion directionality.

\begin{table*}[t]
    \small
\label{table:sensitivity}
\centering
    \caption{Sensitivity and precision of our proposed
             method (Weighted) and Majority Voting (Majority) when inferring single
             bundles in 9 subjects. The inferred bundles are: Superior Longitudinal
             Fasciculus (SLF), Inferior Longitudinal Fasciculus (ILF), Corpus
             Callosum (CC), and the External Capsule (EC).}
\label{my-label}
\begin{tabular}{|l||c|c||c|c||c|c||c|c|}
\hline
 & \multicolumn{2}{c|}{SLF} & \multicolumn{2}{c||}{ILF} & \multicolumn{2}{|c||}{CC} & \multicolumn{2}{|c|}{EC}\\ 
 \hline
            &  Weighted & Majority & Weighted & Majority & Weighted & Majority & Weighted & Majority \\
  \hline
Sensitivity & 0.19\rpm0.04 & \bf{0.30}\rpm0.05 & 0.33\rpm0.02 & \bf{0.47}\rpm0.07 & 0.53\rpm0.04 & \bf{0.64}\rpm0.09 & 0.06\rpm0.02 & \bf{0.27}\rpm0.20 \\
  \hline                                                                                                                               
Precision   & \bf{0.70}\rpm0.11 & 0.58\rpm0.10 & \bf{0.74}\rpm0.05 & 0.41\rpm0.20 & \bf{0.91}\rpm0.15 & 0.82\rpm0.17 & \bf{0.42}\rpm0.20 & 0.31\rpm0.13 \\
\hline
\end{tabular}
\end{table*}

\subsection{Our Technique Shows Lower Sensitivity but Higher Precision than Majority Voting.}
To test our technique in realistic data, we registered tracts between different
subjects of the HCP. Using these registered tracts, we inferred the position
of individual tracts in each subject. Table 1 shows that in each inferred tract,
our technique achieved a lower sensitivity but a higher
precision. In fact, the precision, except for the External Capsule, is always
higher than 0.7. This means that our technique was able to discover less voxels
belonging to the tracts, but at least 70\% of those labeled are correct. Another
way to phrase this is that our techniques presents less false positives at the
cost of more false negatives, making it more conservative than Majority Voting.

\subsection{Our Technique Infers Tracts Before and After, But Not Within 
            Voxel Affected by Pathologies}
We further characterized the behaviour of our technique by simulating lesions
in the white-matter of a HCP subject. Particularly, we defined a spherical region
on the path of the Superior Longitudinal Fasciculus, and increased its
Fractional Anisotropy until achieving isotropic diffusion. In doing so, we
simulated lesions in the brain of a subject for two pathology types:
edema, and tumor with no tract displacement. As explained in the Introduction,
these pathologies are characterized by a decreased FA inside
the lesion, but do not deform the white-matter around it. Figure \ref{fig:labeling}
shows that our technique labels less voxels as the FA becomes higher on the
region. This is a correct behaviour, since the more isotropic is the region,
the less it is related to a white-matter tract. However, since our technique
is still able to label the surroundings of the lesion, we can still identify
the affected tract. It is important to notice that the simulated lesions do not
account for mass-effect, this is, white matter deformations. Such pathologies
are difficult to simulate and study, since the process of registration from
healthy subjects to the patient has to be fined tuned manually . 

\section{Conclusions}
In this chapter we presented a label fusion technique that relies on dMRI
data to infer the localization of white-matter tracts. The results show that
our technique is more conservative than the voting rule, which is desired when
studying pathologies, at the cost of having more false negatives.


\chapterbib
