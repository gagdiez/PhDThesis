\chapter{Abstract}
Understanding how brain connectivity is organized, and how
this constrains brain functionality is a key question of neuroscience. The advent
of Diffusion Magnetic Resonance Imaging (dMRI) permitted the in vivo estimation
of brain axonal connectivity. In this thesis, we leverage these advances in order
to: study how the brain connectivity is organized; study the relationship between
brain connectivity, anatomy, and function; find correspondences between 
structurally-defined regions of different subjects, and infer connectivity in
the presence of a brain’s pathology. We present three major contributions. Our
first contribution is a model for the long-range axonal connectivity, and an
efficient technique to divide the brain in regions with homogeneous connectivity.
Our parceling technique can create both single-subject and groupwise structural
parcellations of the brain. The resulting parcels are in agreement with anatomical,
structural and functional parcellations extant in the literature. Our second
contribution is a method to find correspondence between structural parcellations
of different subjects. Based on Optimal Transport, it performs significantly better
than the state-of-the-art ones. Our third contribution is a multi-atlas technique
to infer the location of white-matter bundles in patients with a brain pathology.
As existent techniques, our approach aggregates spatial information from healthy subjects,
our novelty is to weight such information with the diffusion image of the patient.
We show that our technique achieves better results than the non-weighted methods

\chapter{Résumé}
Comprendre l'organisation de la connectivité
structurelle du cerveau ainsi que comment celle-ci contraint sa fonctionnalité
est une question fondamentale en neuroscience. L'avènement de l'Imagerie par
Résonance Magnétique de diffusion (IRMd) a permit l'estimation de la connectivité
des neurones in vivo. Dans cette thèse, nous profitons de ces avancées pour:
étudier l'organisation structurelle du cerveau; étudier la relation entre la
connectivité, l'anatomie et la fonction cérébrale; identifier les régions
corticales correspondantes d'un sujet à un autre; et inférer la connectivité
en présence de pathologie. Cette thèse contient trois contributions majeures.
La première est un modèle pour la connectivité axonale et une technique efficace
pour diviser le cerveau en régions de connectivité homogène. Cette technique de
parcellisation permet de diviser le cerveau tant pour un seul sujet que pour une
population. Les parcelles résultantes sont en accord avec les parcellations
anatomiques, structurelles et fonctionnelles existant dans la littérature. 
La seconde contribution de cette thèse est une technique qui permet
d'identifier les régions correspondantes d'un sujet à un autre. Cette technique,
basée sur le transport optimal, offre une meilleure performance que les techniques
courantes. La troisième contribution est une technique de segmentation, dite
multi-atlas, pour identifier les faisceaux d'axones de la matière blanche de
patients atteints d'une pathologie cérébrale. Comme les techniques existantes,
notre approche utilise l'information spatiale provenant d'atlas de sujets sains,
mais pondère celle-ci avec l'information d'IRMd du patient. Nous montrons que
notre technique obtient de meilleurs résultats que les méthodes non pondérées.
