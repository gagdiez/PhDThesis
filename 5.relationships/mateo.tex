Introduction:
Due to its ill-posed nature, tractography generates a significant number of false positive connections between brain
regions (Maier-Hein et al. 2017). To reduce the number of false positives, Daducci et al. (2015) proposed the COMMIT
framework, which has the goal of re-establishing the link between tractography and tissue microstructure (Girard et al.
2017). In this framework, the diffusion MRI signal is modeled as a linear combination of local models associated with
streamlines were the weights are identified by solving a convex optimization problem. Streamlines with a weight of zero do
not contribute to the diffusion MRI data and are assumed to be false positives. Removing these false positives yields a
subset of streamlines supporting the anatomical data. However, COMMIT does not make use of the link between structure
and function and thus weights all bundles equally. In this work, we propose a new strategy that enhances the COMMIT
framework by injecting the functional information provided by functional MRI. The result is an enhanced tractogram
filtering strategy that considers both functional and structural data.

Methods:
We randomly selected 3 subjects from the HCP500 dataset, each processed with the HCP minimal pipeline (Van Essen et
al. 2012). For each subject, we performed probabilistic tractography using the vertices of the white matter mesh as seed
locations. We then parcellated the cortex into 74 regions based on their extrinsic anatomical connectivity using the
algorithm proposed by Gallardo et al. (2017). Using the resting state functional MRI of each subject, we computed the
functional connectivity between cortical regions. For this, we first high pass-filtered the time series and computed the
average functional MRI signal within each region. Finally, we divided the signals into windows of 100 seconds and
computed the maximum sliding window correlation between every pair of regions. These correlation coefficients were
used to define a regularization coefficient for each bundle of the tractogram. The regularization coefficients were designed
to favour streamlines associated with a high correlation while penalizing but not excluding those with a low correlation.
The rationale is that streamlines connecting regions that are functionally correlated should be favoured to explain the
diffusion MRI data. As in the non functional COMMIT, we solved the optimization problem and removed streamlines with a
weight of zero. For comparison, we also filtered the tractograms using equal weights for all bundles, i.e. without functional
prior.

Results:
To measure the performance of the proposed algorithm we evaluated the quantity of candidate false positive streamlines
that we were able to detect and the diffusion MRI signal fitting error. A filtering strategy is considered superior if it is able
to remove more streamlines while still explaining the diffusion MRI signal. Figure 1 illustrates that including functional
information allows us to increase the number of candidate false positives detected without affecting the data fitting error,
therefore indicating enhanced filtering. Interestingly, the addition of functional information modified the spatial location of
the removed bundles in a non-uniform manner. The streamlines removed by both filtering strategies and those removed
by one or the other are illustrated in Figure 2.

