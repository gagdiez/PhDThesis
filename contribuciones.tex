\chapter{Contribuciones Metodol\'ogicas}
\label{ch:contribuciones}

En el cap\'itulo anterior presentamos como generar los tractogramas y el 
estado del arte para agruparlos. En este cap\'itulo presentamos las
contribuciones de nuestro trabajo. Comenzamos con una contribuci\'on menor
que es estudiar la estabilidad del algoritmo de tractograf\'ia que
utilizamos. Luego hacemos un an\'alisis te\'orico del estado del arte y
mostramos algunas de las problem\'aticas que posee. Finalmente utilizamos
una funci\'on para transformar los tractogramas a un espacio vectorial y
mostramos como esto permite mejorar tanto la complejidad espacial como
temporal del algoritmo de \textit{clustering}. \\

\section{Estudio de la Convergencia de los Tractogramas}

Algunas preguntas interesantes a realizar sobre los tractogramas son: ¿Al repetir
el experimento, podremos obtener el mismo tractograma?; ¿Cu\'antas part\'iculas son
necesarias para ello? y ¿Qu\'e tanto difieren los resultados entre los distintos 
algoritmos de tractograf\'ia?. \\

Para responder estas preguntas tomamos una implementaci\'on de tractograf\'ia ya 
existente llamada \textit{LocalTracking} (LT de aqu\'i en mas) y un algoritmo
propio (MSL de aqu\'i en m\'as). Ambos algoritmos poseen una estructura 
similar: Encuadran la imagen de difusi\'on en un modelo; en base a ese modelo 
crean un mapa de transiciones probabil\'isticas entre voxels y lo recorren 
de manera aleatoria hasta cumplir un criterio de parada. Para detalles sobre la 
implementaci\'on, por favor referirse al Anexo. \\

Para determinar si los algoritmos se estabilizaban y el n\'umero de part\'iculas
necesario para que eso suceda utilizamos la t\'ecnica estad\'istica de
\textit{bootstrap} \cite{Efron1982}. Bootstrap es una forma de aproximar la
distribuci\'on del muestreo de un estad\'istico en base a calcular el mismo
utilizando sucesivos remuestreos de los datos con repeticiones. Esto es
especialmente \'util cuando el n\'umero de muestras que se posee de la poblaci\'on
no es significativamente alto. En nuestro caso situamos mas de setecientas
semillas en el \'Area de Broca y luego generamos quince mil \textit{streamlines}
por cada una. Luego calculamos el tractograma medio y la varianza de cada voxel 
utilizando mil submuestras aleatorias del mismo tama\~no. Esto se repiti\'o con 
varios tama\~nos de submuestra para estudiar as\'i la variabilidad a medida que 
la cantidad de part\'iculas crec\'ia.\\


\section{An\'alisis del m\'etodo Moreno-Dominguez}
\label{sec:analisis_moreno}

El cuarto paso del diagrama en la figura \ref{fig:diagrama} es agrupar los
tractogramas mediante un algoritmo de $clustering$. Moreno-Dominguez et
al. \cite{Moreno-Dominguez2014} implementan el algoritmo
\textit{Agglomerative Hierarchical Clustering} para agrupar los
tractogramas. En particular utilizan como medida de similitud la distancia
coseno (ecuaci\'on \ref{eq:cosine}) y como criterio de \textit{linkage} el
centroide (ecuaci\'on \ref{eq:centroide}). Utilizar \textit{Hierarchical
Agglomerative Clustering} con la distancia coseno y el \textit{linkage}
centroide presenta algunas desventajas. Primero, la distancia coseno no
distingue distancias entre clusters colineales. Segundo, el promedio de
probabilidades no necesariamente representa una probabilidad 
\cite{Pohl2007}. Esto quiere decir que el centroide de un grupo de
tractogramas no necesariamente representa el promedio de estos
tractogramas. Finalmente la distancia coseno obliga a tener que comparar
expl\'icitamente cada centroide con los clusters existentes. Esto aumenta
la complejidad temporal y espacial ya que es necesario mantener los
tractogramas en memoria. A continuaci\'on detallamos cada una de estas
desventajas. \\

\subsection{Clustering de vectores colineales}

La distancia coseno (ecuaci\'on \ref{eq:cosine}) es una forma de medir 
correlaci\'on entre vectores. Supongamos tenemos
vectores aleatorios colineales donde cada componente es independiente y 
proviene de una distribuci\'on de Bernulli. Agruparlos utilizando como
medida de similitud la distancia coseno lleva a resultados err\'oneos. 
Por ejemplo, agrupar los puntos de la figura \ref{fig:3clusters} usando
el m\'etodo de Moreno-Dominguez da como resultado la figura 
\ref{fig:3moreno}. Podemos ver que si bien ten\'iamos tres clusters bien
definidos en el espacio el m\'etodo no logr\'o separarlos. \\

\begin{figure}[h!]
        \centering
        \includegraphics[width=0.5\textwidth]{img/3pop.png}
        \caption{Tres $clusters$ colineales. Cada punto representa un
                 vector aleatorio con componentes provenientes de una
                 distribucion Bernulli. Queremos agruparlos utilizando
                 \textit{Agglomerative Hierarchical Clustering}.}
        \label{fig:3clusters}
\end{figure}

Tambi\'en existe otro problema. Por la forma que posee el algoritmo 
\ref{alg:itract}, los tractogramas creados poseen muchos voxels lejanos
con valores peque\~nos. El algoritmo simula el recorrido de part\'iculas de
agua sobre mapas de transiciones probabil\'isticas. Cuanto mas lejano un 
voxel mayor cantidad de transiciones se necesitan para llegar. Por lo 
tanto, cuanto mas lejos se encuentra un voxel, menor probabilidad tiene de
ser visitado. Para visualizar esto mejor situamos semillas en el \'area
de Broca de un sujeto y generamos sus tractogramas. La figura 
\ref{fig:hist_tract} muestra el histograma de los valores en los mapas de
visitas. Cada voxel dentro de un mapa de visitas indica que cantidad de 
part\'iculas pasaron por \'el.
En este caso el $70\%$ de los voxels visitados fueron alcanzados por solo
una part\'icula. Esta gran cantidad de voxel con valores tan bajos podr\'ia
generar ruido en las correlaciones. \\

\begin{figure}[h!]

\centering                                                                                                          
\begin{minipage}[b]{0.85\textwidth}
    \includegraphics[width=\textwidth]{img/3pop_moreno.png}
    \caption{$Clustering$ resultante de utilizar el m\'etodo de 
             Moreno-Dominguez para agrupar los vectores. El algoritmo
             agrupa la mayor\'ia de los vectores en una \'unica 
             categor\'ia.}
    \label{fig:3moreno}
\end{minipage} ~

\end{figure}  

\begin{figure}[h!]
              \centering                                                                                                          
\begin{minipage}[b]{0.8\textwidth}
    \includegraphics[width=\textwidth]{img/hist_tract.png}
    \caption{Histograma normalizado de los valores en los mapas de visitas
             del \'area de Broca. No se muestran los valores mayores a 12.
             La mayor parte de los voxels visitados tuvieron una \'unica
             visita.}
    \label{fig:hist_tract}
\end{minipage} ~

\end{figure}  


\subsection{Relaci\'on m\'etrica-\textit{linkage}}

Moreno-Dominguez et al. utilizan como medida de similitud la distancia
coseno (ecuaci\'on \ref{eq:cosine}) y como criterio de \textit{linkage} el
centroide (ecuaci\'on \ref{eq:centroide}). La medida de similitud 
compara el \'angulo de los vectores, pero el $linkage$ crea un 
representante minimizando la distancia eucl\'idea a ambos clusters. No
parecieran ser compatibles. Veamos un ejemplo. La figura \ref{fig:cos_cen}
muestra 4 vectores aleatorios $p_i = (X_i, Y_i)$. Las variables $X_i$ e
$Y_i$ son independientes y provienen de una distribuci\'on Bernulli. Las
posiciones en coordenadas polares de cada vector son: 
$ p_1 = (0.4, 45^\circ)$; $p_2 = (0.3, 25^\circ)$; $p_3 = (0.4, 66^\circ)$
y $p_4 = (0.4, 4.5^\circ)$. \\

\begin{figure}[h!] 
    \centering
    \includegraphics[width=\textwidth]{img/cosine_centroid.png}
    \caption{El centroide ($p_c$) de $p_2$ y $p_3$ no representa el punto
             medio ($p_m$) respecto al \'angulo entre ellos.}
    \label{fig:cos_cen}
\end{figure} 

Podemos apreciar que al principio $d(p_2,p_3) < d(p_3,p_4) < d(p_1,p_2)$,
siendo $d(x,y)$ la distancia coseno (ecuaci\'on \ref{eq:cosine}). Sin
embargo, luego de utilizar el \textit{linkage centroid} 
(ecuaci\'on \ref{eq:centroide}) sucede que $d(p_1,p_c) < d(p_4,p_c)$. $p_4$
es ahora el punto que mas lejos est\'a del centroide. Creando un
representante $p_m$ usando el \'angulo medio entre $p_2$ y $p_3$ esto no
sucede. Este fen\'omeno se da porque la distancia coseno tiene en cuenta
el \'angulo pero el centroide no. Por lo tanto el centroide no caracteriza
al punto medio respecto a la distancia coseno. \\


\subsection{Complejidad algor\'itmica del clustering}
\label{sec:complejidad_moreno}

El diagrama de la figura \ref{fig:diagrama} divide el proceso de 
parcelaci\'on de la corteza en varios pasos: seleccionar los voxels que 
ser\'an utilizados como semilla de cada tractograma; luego generar cada
tractograma y finalmente agruparlos usando alg\'un algoritmo de clustering.
En todo este proceso el paso mas caro en t\'erminos computacionales es el
\textit{clustering}. Veamos la  complejidad del m\'etodo propuesto por
Moreno-Dominguez. En cada iteraci\'on del algoritmo es necesario calcular
un centroide, almacenarlo y compararlo expl\'icitamente con el resto de
los clusters. Por cada iteraci\'on es necesario hacer 
$O(c^2 m)$ operaciones para recalcular todas las distancias, donde $c$ es
la cantidad de clusters y $m$ es la longitud de los mismos. Dadas $s$
semillas iniciales, la cantidad de iteraciones a realizar son $s-1$. La
complejidad temporal de este m\'etodo es $O(s^3 m)$. Respecto a la 
complejidad espacial, el mantener todos los clusters en memoria requiere
$O(s m)$ espacio. Tambi\'en es necesario mantener la matriz de distancias
entre $clusters$, la cual requiere $O(s^2)$. La complejidad espacial total
es $O(s m + s^2)$. Podemos notar que ambas complejidades dependen de $m$.
\\ 

Resumiendo, utilizar \textit{Hierarchical Agglomerative Clustering} 
con la distancia coseno y el \textit{linkage} centroide presenta algunas
desventajas: La medida de similitud y el $linkage$ no son compatibles;
no permite agrupar de manera correcta clusters colineales y 
necesita mantener todos los $clusters$ en memoria as\'i como tambi\'en
compararlos expl\'icitamente. \\


\section{Clustering en el Espacio Eucl\'ideo}

En la secci\'on anterior presentamos algunos de los inconvenientes te\'oricos 
del m\'etodo propuesto por Moreno-Dominguez. A continuaci\'on mostramos como 
solucionar todos ellos haciendo uso de la tranformaci\'on \textit{logit}. \\

\subsection{Transformaci\'on Logit}

Sea $P_M$ el espacio de una distribuci\'on discreta para $M$ etiquetas: 

$$P_M = \left\{  p | p = (p_1,\dots p_n) \in (0,1)^M , \sum{p_i} = 1 \right\}$$

La funci\'on \textit{logit}:$P_M \rightarrow R^{M-1}$ define una transformaci\'on
entre el espacio $P_M$ y el espacio eucl\'ideo $R^{M-1}$. Dados los vectores $Q \in P^M$ y
$S \in R^{M-1}$:

$$S_i = logit(Q_i) = log\left(\frac{Q_i}{Q_M}\right)$$

Para el caso de $M=2$ permite transformar la distribuci\'on Bernoulli
discreta al espacio eucl\'idio. Podemos ver en la ecuaci\'on \ref{eq:logit} su
expresi\'on anal\'itica y en la Figura \ref{fig:dominio} representaci\'on 
gr\'afica.

\begin{figure}[h!]

\begin{minipage}[b]{0.45\textwidth}

    \begin{equation}
    \vspace{2.85cm}
        \label{eq:logit}
    logit(p) = log\left(\frac{p}{1-p}\right)
    \end{equation}
\end{minipage} ~
\hfill
\begin{minipage}[b]{0.45\textwidth}
    \includegraphics[width=\textwidth]{img/logit.png}
    \caption{Representaci\'on gr\'afica de la funci\'on logit}
    \label{fig:dominio}
\end{minipage} ~

\end{figure}  

Trabajar en el espacio eucl\'ideo nos asegura que la suma y la multiplicaci\'on
por escalares est\'an contenidos en el mismo espacio. Pohl et al. \cite{Pohl2007} 
utilizan esta propiedad para realizar operaciones lineales sobre mapas
probabil\'isticos. A su vez, demuestran que la suma y multiplicaci\'on
por escalares en el espacio eucl\'ideo poseen un significado en el espacio de la
distribuci\'on. \\

\subsection{Modificando el Algoritmo de Clustering}

Asumiendo que cada voxel $v$ de un tractograma proviene de la variable 
aleatoria binaria:

 $$X_v= \textrm{``La semilla est\'a conectada con el voxel v''}$$
 
Es posible transformar el tractograma aplicando la funci\'on \textit{logit} en
cada uno de sus voxels. El resultado es un vector donde cada coordenada se 
encuentra en el espacio eucl\'ideo.  M\'as a\'un, las operaciones lineales entre
los mismos voxels en distintos tractogramas est\'an definidas. Esto nos permite
usar la m\'etrica eucl\'ideana como funci\'on de similitud en \textit{Agglomerative
Hierarchical Clustering}. \\

Transformar de espacio los tractogramas y cambiar la funci\'on de similitud 
posee varias ventajas. Para empezar permite agrupar correctamente vectores
colineales. La Figura \ref{fig:3logit} muestra el resultado de aplicar este
m\'etodo a los vectores de la Figura \ref{fig:3clusters}.  Podemos apreciar 
que las tres poblaciones se encuentran correctamente separadas y bien definidas.\\

\begin{figure}[h!]

\centering
\begin{minipage}[b]{0.85\textwidth}
    \includegraphics[width=\textwidth]{img/3pop_logit.png}
    \caption{Clustering resultado de utilizar el m\'etodo logit}
    \label{fig:3logit}
\end{minipage} ~

\end{figure}  

Otra ventaja es la buena relaci\'on m\'etrica-\textit{linkage}. Por definici\'on
el centroide es el centro de masa de los clusters. Esto quiere decir que es el 
punto que minimiza la distancia euclidiana entre los clusters que lo componen. 
Por ende, el centroide caracteriza bien el punto medio de los vectores en el
espacio eucl\'ideo. \\

Finalmente, nuestro m\'etodo tambi\'en permite mejorar la complejidad algor\'itmica.
Como ya explicamos, por cada iteraci\'on del algoritmo 
\textit{Agglomerative Herarchical Clustering} es necesario calcular un representante
de la uni\'on y luego computar su distancia al resto. Sin embargo, al usar la
m\'etrica euclideana junto con el \textit{linkage} centroide es posible simplificar
este paso. La formula de Lance y Williams permite computar las nuevas distancias
sin comparar expl\'icitamente los clusters. Esto baja significativamente la
complejidad. Cada iteraci\'on pasa a costar $O(c^2)$ en vez de $O(c^2 m)$, siendo
$c$ la cantidad de clusters y $m$ la longitud de los mismos. Dadas $n$ semillas 
iniciales, la complejidad temporal total del \textit{clustering} es $O(n^3)$. 
Con la distancia coseno era $O(n^3 m)$, siendo $m$ la longitud de los tractogramas.
Recordemos que en el contexto que estamos utilizando este algoritmo $m>>n$. Por
lo tanto, este resultado implica una gran mejora en la eficiencia del algoritmo.  


\subsection{Almacenando tractogramas: Matrices ralas}
\label{sec:ralas}

Los m\'etodos para parcelar la corteza utilizados en este trabajo est\'an
basados en el $clustering$ de tractogramas. Podemos ver en la figura 
\ref{fig:diagrama} que primero generamos los tractogramas y luego se los
damos a un algoritmo para que los agrupe. Entre estos pasos es necesario
almacenar o mantener en memoria los tractogramas. En esta secci\'on
veremos como aprovechar la propiedad rala de los tractogramas para 
almacenarlos eficientemente. Tambi\'en mostramos como recuperar esta
propiedad luego de transformarlos al espacio $LogOdds$. \\ 
 
A modo de ejemplo situamos $762$ semillas en el \'area de Broca de un 
sujeto y generamos sus tractogramas. Aplanamos todos los tractogramas
a una sola dimensi\'on y los usamos como filas de una matriz. La matriz 
resultante tiene dimensiones $762\times3587328$. Asumiendo que cada valor
se representa usando $8$ Bytes ({\it double precision}), esta matriz ocupa
un total de
aproximadamente $20$ Gigabytes. Sin embargo, solo un $1\%$
de los datos almacenados son no nulos. Esto implica que casi todo el
espacio utilizado es desperdiciado.\\

\begin{figure}[h!]
   \centering
    \includegraphics[width=0.9\textwidth]{img/densa_broca.png}
    \caption{Matriz con los tractogramas generados en el \'area de Broca
             de un sujeto. Las columnas cuyos valores eran todos nulos
             fueron removidas. }
    \label{fig:densa}
\end{figure}

Una forma de reducir el espacio requerido para almacenar la matriz es 
eliminar las columnas con solo elementos nulos. La figura \ref{fig:densa}
muestra la matriz que resulta de eliminar las columnas vac\'ias de la matriz
del \'area de Broca. Manteniendo la representaci\'on de $8$ Bytes este
proceso reduce el espacio necesario a aproximadamente $700$ Megabytes.
Si bien esto es una gran mejora, la nueva matriz posee solo un $27\%$ de
valores no nulos. Seguimos desperdiciando mucho espacio. \\

En el caso de crear los tractogramas de todo el hemisferio derecho usando
$21657$ semillas el eliminar columnas no es suficiente. La matriz
resultante posee dimensiones $21657\times3587328$. En esta, solo un $1\%$
de los valores son no nulos. Para almacenar dicha matriz es necesario
utilizar $587$ Gigabytes. Por esto es necesario utilizar estructuras mas
eficiente, que aprovechen lo ralo de las matrices. Ejemplos de estas
estructuras son: \textit{Dictionary of Keys}, o una matriz 
\textit{Compressed Sparse Row} (CSR). \\

%eliminando columnas in\'utiles nos queda 21657x145574, son 23.5 Gn

Al momento querer aplicar esto en nuestro m\'etodo nos encontramos con un
inconveniente. Nosotros transformamos los tractogramas al espacio $LogOdds$
antes de agruparlos. Por la forma que tiene la funci\'on $logit$ 
(ecuaci\'on \ref{eq:logit}) los tractogramas transformados ya no son ralos. 
Podemos ver en la figura \ref{fig:dominio} que 
\textit{logit}$(0) = -\infty$. Esto implica que la transformaci\'on de una
matriz rala no es rala en t\'erminos de elementos nulos. Sin embargo
podemos aprovechar ciertas propiedades para recuperar las matrices ralas.
La distancia euclidiana entre vectores es invariante a traslaciones lineales
del sistema. Lo mismo sucede con las posiciones relativas de los
$centroides$. Asignemos una representaci\'on finita $c$ al valor $-\infty$.
Un buen candidato para $c$ es el $log(\epsilon)$, donde $\epsilon$ es el
\textit{epsilon de la maquina}. Transformar todos los vectores y luego
trasladarlos sumando $c$ en cada componente dar\'a como resultado una
representaci\'on rala. Gracias a esto podemos utilizar DOK, CSR o cualquier
estructura para reducir el espacio necesario para almacenar los
tractogramas. \\

Dadas dos matrices ralas existen maneras eficientes de multiplicarlas entre
si \cite{Bank1993}. Si calculamos la distancia euclidiana entre dos vectores
ralos usando la siguiente relaci\'on: 

\begin{equation}
\label{eq:simileuc}
simil_{euc}(X,Y) = \sqrt{X \cdot X^t - 2 X \cdot Y^t + Y \cdot Y^t} 
\end{equation}

Podemos generar la matriz de distancias usando directamente las estructuras
de matrices ralas. Es importante destacar que este m\'etodo generara una
matriz que no necesariamente es sim\'etrica. Algunos valores que deber\'ian
ser iguales pueden presentar un peque\~no error num\'erico. Esto nos
permite utilizar matrices ralas durante todo el proceso de parcelamiento de
la corteza. \\




\input{contribuciones/nuestro.tex}

\section{Correlato con otras parcelaciones de la literatura}
\label{sec:cont_correlato}

Nuestro m\'etodo divide la corteza cerebral bas\'andose en el 
agrupamiento de tractogramas. El criterio usado es puramente estructural.
Solo usamos informaci\'on sobre la estructura f\'isica de la materia
blanca. En la literatura actual existen parcelaciones basadas en otros
criterios. Desikan et al. \cite{Desikan2006} dividen la corteza usando
como referencia las circunvoluciones del cerebro. Peinfield 
\cite{Penfield1954} parcela la corteza en base a las respuestas a 
est\'imulos el\'ectricos sobre la misma. La parcelaci\'on de Desikan es
anat\'omica, mientras que la de Peinfield es funcional. Una pregunta
interesante es si nuestra parcelaci\'on posee correlaci\'on con otras
anat\'omicas o funcionales. Estudiaremos esto contrastando el resultado
de nuestro m\'etodo contra dos parcelaciones. Una funcional basada en 
Bach et al. \cite{Barch2013} y una anat\'omica basada en Desikan
\cite{Desikan2006}.\\

Para comparar nuestro resultado contra el atlas anat\'omico 
proyectaremos sus parcelas sobre las nuestras. Luego estudiaremos cuantas
de nuestras regiones quedan bien delimitadas dentro de las proyecciones.
Esto es, que proporci\'on de las \'areas en una proyecci\'on se
encuentran efectivamente dentro de la proyecci\'on. \\

Para estudiar la relaci\'on funcional usaremos un estudio hecho por
Bach et al. \cite{Barch2013}.
En el mismo utilizan Resonancia Magn\'etica Funcional
(fMRI) para estudiar la respuesta sobre la corteza a est\'imulos
particulares. La Resonancia Magn\'etica Funcional es un tipo especial de
resonancia magn\'etica que mide el nivel de oxigeno en sangre 
\cite{Ogawa1990}. Nosotros nos enfocamos en las respuestas a los siguientes
est\'imulos: mover la mano; mover el pie; mover la lengua; reconocer formas
y caras vistas con anterioridad; clasificar un cuento breve dentro de dos
categor\'ias y resolver problemas aritm\'eticos simples. Cada uno de estas
tareas se encuentra detallada en el trabajo de Bach et al. \cite{Barch2013}.
Por cada uno de estos est\'imulos se cuenta con un mapa de 
\textit{z-scores} sobre la corteza. Estos \textit{z-scores} representan la
correlaci\'on entre la activaci\'on de cada punto de la corteza con el 
est\'imulo. Para comparar los resultados de este estudio con nuestra 
parcelaci\'on estudiaremos su superposici\'on. Definimos la siguiente 
relaci\'on: \\

\begin{equation}
\label{eq:super}
superposicion(A,B) = \frac{ 2 * area(A \cap B) }{area(A) + area(B)}
\end{equation}
 
Dadas dos superficies $A$ y $B$ se cumple: 
$0 \leq superposicion(A,B) \leq 1$. El $0$ representa que son superficies
disjuntas y el $1$ que est\'an totalmente superpuestas. Si alg\'un \'area 
posee un fuerte correlato funcional esperamos ver una superposici\'on alta
con alg\'un mapa de \textit{z-scores}.\\

Comparar las proyecciones del atlas an\'atomico con nuestras \'areas y 
la superposici\'on de estas \'ultimas con los \textit{z-scores} del 
estudio funcional nos permitir\'a hacer un primer acercamiento a 
analizar la relaci\'on entre la estructura, la anat\'omia y la funcionalidad
del cerebro.\\



