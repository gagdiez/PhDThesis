\section{Correlato con otras parcelaciones de la literatura}
\label{sec:res_correlato}

Nuestro m\'etodo divide la corteza cerebral bas\'andose en un criterio
puramente estructural. En la literatura actual existen parcelaciones
basadas en otros criterios. A continuaci\'on contrastaremos nuestros 
resultados contra un atlas anat\'omico basado en Desikan \cite{Desikan2006}
y un estudio funcional realizado sobre el sujeto por Bach \cite{Barch2013}.

\subsection{Correlaci\'on anat\'omica}
\label{sec:corr_anatomica}

En la secci\'on \ref{sec:corteza_nuestro} parcelamos el hemisferio 
izquierdo de un sujeto con nuestro m\'etodo. Vamos a comparar el resultado
obtenido con el atlas anat\'omico basado en Desikan et al. 
\cite{Desikan2006}. La figura \ref{fig:an2pa} muestra el resultado de
proyectar algunas regiones anat\'omicas sobre las obtenidas con nuestro
m\'etodo. La figura \ref{fig:an2pa2} muestra solo las regiones que estaban
contenidas en su mayor\'ia dentro de estas proyecciones. Luego calculamos
que porcentaje de las parcelaciones quedaba dentro de las proyecciones;
que porcentaje quedaba fuera y cuantas quedaban dentro de una sola 
proyecci\'on. El $90\%$ del \'area de las 9 proyecciones est\'a pintada y
solo un $5\%$ del \'area de nuestras parcelas queda fuera de las
proyecciones. Las parcelas de nuestro m\'etodo que se encuentran en una
sola proyecci\'on representan el $88\%$ del \'area pintada. Las parcelas
que est\'an en dos regiones representan el $12\%$ restante. \\

Repetimos el procedimiento con los resultados obtenidos utilizando el 
m\'etodo de Moreno-Dominguez. Proyectamos el atlas anat\'omico sobre la
parcelaci\'on mostrada en la figura \ref{fig:moreno_corteza2}.
En este caso el $75\%$ del \'area de las 9 proyecciones est\'a pintada y 
un $15\%$ del \'area de las parcelas quedan fuera de las proyecciones. \\

\begin{figure}[h!]
    \includegraphics[width=\textwidth]{img/anatomica2parcelation.png}
    \caption{Proyecci\'on del atlas anat\'omico (izquierda) a la 
             parcelaci\'on obtenida (derecha). Regiones: 
             1. \textit{pars triangularis}; 2. \textit{pars opercularis};
             3. \textit{caudal middle frontal}; 
             4. \textit{precentral (\'area motora)}; 
             5. \textit{post central}; 6. \textit{superior parietal}; 
             7. \textit{lateral occipital}; 8. \textit{fusiform};
             9. \textit{superior temporal}.  }
    \label{fig:an2pa}
\end{figure}

\begin{figure}[h!]
    \includegraphics[width=\textwidth]{img/anatomica2parcelation2.png}
    \caption{Proyecci\'on del atlas anat\'omico (izquierda) a la 
             parcelaci\'on obtenida (derecha). Solo las parcelas contenidas
             en mayor proporci\'on fueron pintadas. Regiones: 
             1. \textit{pars triangularis}; 2. \textit{pars opercularis};
             3. \textit{caudal middle frontal}; 
             4. \textit{precentral (\'area motora)}; 
             5. \textit{post central}; 6. \textit{superior parietal}; 
             7. \textit{lateral occipital}; 8. \textit{fusiform};
             9. \textit{superior temporal}.  }
    \label{fig:an2pa2}
\end{figure}

\subsection{Correlaci\'on funcional}
\label{sec:corr_funcional}

Dentro del estudio funcional basado en Bach \cite{Barch2013} nos
enfocamos en las respuestas a los siguientes est\'imulos: mover la mano;
mover el pie; mover la lengua; reconocer formas y caras vistas con
anterioridad; clasificar un cuento breve dentro de dos categor\'ias y
resolver problemas aritm\'eticos simples. Por cada uno de estos est\'imulos
se cuenta con un mapa de \textit{z-scores}. Estos mapas representan la
correlaci\'on entre la activaci\'on de cada punto de la corteza con el 
est\'imulo. \\

\begin{figure}[h!]
    \includegraphics[width=\textwidth]{img/32k_labels.png}
    \caption{Parcelaci\'on obtenida utilizando nuestro m\'etodo. Se 
             etiquetaron algunas \'areas de inter\'es para comparar
             con un estudio funcional.}
    \label{fig:32k}
\end{figure}


\begin{figure}[h!]
    \includegraphics[width=\textwidth]{img/32k_z5.png}
    \caption{Proyecci\'on de 14 \'areas obtenidas con nuestro m\'etodo
             sobre los \textit{z-scores} mayores a $5$ de las activaciones
             funcionales.}
    \label{fig:32k_z5}
\end{figure}

\begin{figure}[h!]
    \includegraphics[width=\textwidth]{img/32k_z7.png}
    \caption{Proyecci\'on de 14 \'areas obtenidas con nuestro m\'etodo
             sobre los \textit{z-scores} mayores a $7$ de las activaciones
             funcionales.}
    \label{fig:32k_z7}
\end{figure}


Queremos comparar los mapas de \textit{z-scores} con una de las 
parcelaciones obtenidas mediante nuestro m\'etodo. Para ello primero
interpolamos la parcelaci\'on utilizada en la secci\'on anterior a la
superficie en que se encuentran los mapas. El resultado de dicha 
interpolaci\'on puede verse en la figura \ref{fig:32k}. Algunas regiones
de inter\'es fueron etiquetadas. La figura \ref{fig:32k_z5} muestra la
proyecci\'on de estas regiones sobre los \textit{z-scores} $ \geq 5$ de
los distintos mapas. La figura \ref{fig:32k_z7} muestra la proyecci\'on
sobre los \textit{z-scores} $ \geq 7$. \\

Utilizamos la ecuaci\'on \ref{eq:super} para estudiar la superposici\'on
entre nuestras regiones proyectadas y los \textit{z-scores}. La tabla 
\ref{tb:zscore5} muestra la superposici\'on entre las superficies de 
nuestra parcelaci\'on y los \textit{z-scores} $ \geq 5$.  La tabla 
\ref{tb:zscore7} muestra la superposici\'on entre las superficies de 
nuestra parcelaci\'on y los \textit{z-scores} $ \geq 7$. En cada columna
se resalt\'o el valor mas grande. Esto es, el valor de la parcela que
mayor superficie compart\'ia con cada mapa de activaci\'on.  \\
 
\begin{table}[]
\centering
\begin{tabular}{|l|l|l|l|l|l|l|}
\hline
   & pie   & mano  & lengua & matem\'atica & cuento & formas \\ \hline
1  & 0     & 0     & 0      & 0.276      & 0      & 0      \\ \hline
2  & 0     & 0     & 0.311  & 0          & 0      & 0      \\ \hline
3  & 0     & 0     & 0      & 0          & 0      & 0      \\ \hline
4  & 0     & 0.172 & 0      & 0          & 0      & 0      \\ \hline
5  & 0.177 & 0.321 & 0      & 0          & 0      & 0      \\ \hline
6  & 0.201 & 0     & 0      & 0          & 0      & 0      \\ \hline
7  & 0.231 & 0.044 & 0      & 0          & 0      & 0      \\ \hline
8  & 0     & {\bf 0.590} & 0.139  & 0          & 0      & 0      \\ \hline
9  & 0     & 0     & {\bf 0.610}   & 0          & 0      & 0      \\ \hline
10 & {\bf 0.330}  & 0     & 0      & 0          & 0      & 0      \\ \hline
11 & 0     & 0     & 0      & 0          & {\bf 0.606}  & 0      \\ \hline
12 & 0     & 0     & 0      & 0          & 0.454  & 0      \\ \hline
13 & 0     & 0     & 0      & 0          & 0      & {\bf 0.491}  \\ \hline
14 & 0     & 0     & 0      & 0          & 0      & 0.154  \\ \hline
\end{tabular}
\caption{Relaci\'on entre las \'areas de cada parcela y los mapas de
         activaci\'on con $zscore > 5$. Los valores m\'aximos de cada 
         columna fueron resaltados.}
\label{tb:zscore5}         
\end{table}


\begin{table}[]
\centering

\begin{tabular}{|l|l|l|l|l|l|l|}
\hline
   & pie   & mano  & lengua & matem\'atica & cuento & formas \\ \hline
1  & 0     & 0     & 0      & {\bf 0.419} & 0     & 0      \\ \hline
2  & 0     & 0     & 0.323  & 0          & 0      & 0      \\ \hline
3  & 0     & 0     & 0      & 0          & 0      & 0      \\ \hline
4  & 0     & 0.115 & 0      & 0          & 0      & 0      \\ \hline
5  & 0.141 & 0.341 & 0      & 0          & 0      & 0      \\ \hline
6  & 0.095 & 0     & 0      & 0          & 0      & 0      \\ \hline
7  & {\bf 0.377} & 0.040 & 0      & 0          & 0      & 0      \\ \hline
8  & 0     & {\bf 0.669} & 0.147  & 0          & 0      & 0      \\ \hline
9  & 0     & 0     & {\bf 0.629}  & 0          & 0      & 0      \\ \hline
10 & 0.240 & 0     & 0      & 0          & 0      & 0      \\ \hline
11 & 0     & 0     & 0      & 0          & {\bf 0.715}  & 0      \\ \hline
12 & 0     & 0     & 0      & 0          & 0.458  & 0      \\ \hline
13 & 0     & 0     & 0      & 0          & 0      & {\bf 0.428}  \\ \hline
14 & 0     & 0     & 0      & 0          & 0      & 0.154 \\ \hline
\end{tabular}
\caption{Relaci\'on entre las \'areas de cada parcela y los mapas de
         activaci\'on con $zscore > 7$. Los valores m\'aximos de cada 
         columna fueron resaltados.}
\label{tb:zscore7}         
\end{table}

