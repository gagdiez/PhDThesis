Dipy es una librer\'ia para python que contiene, entre otras herramientas,
varios algoritmos para generar \textit{streamlines}. Para este trabajo elegimos
utilizar la implementaci\'on de \textit{LocalTracking} (LT de aqu\'i en mas) que
se encuentra en el paquete \textit{dipy.tracking.local}; y una implementaci\'on 
propia (MSL de aqu\'i en m\'as). Ambos algoritmos poseen una estructura 
similar: Encuadran la imagen de difusi\'on en un modelo; crean un objeto que les
permita seleccionar una direcci\'on hacia donde caminar en base a la posici\'on
actual y se mueven hasta cumplir un criterio de parada.\\

En ambos casos modelamos la informaci\'on de dMRI usando el \textit{Constrained
Spherical Deconvolve Model}. La principal diferencia entre los algoritmos surge
en la forma en que seleccionan c\'omo avanzar. Dado el conjunto de direcciones 
iniciales, esto es, las direcciones posibles a tomar desde la semilla,
LT intenta en sucesivas repeticiones del experimento elegir una distinta, 
usando as\'i todas al menos una vez. MSL por otro lado selecciona una al
azar cada vez que repite el experimento. A su vez, LT utiliza un criterio de 
parada basado en la anisotrop\'ia de la difusi\'on; MSL usa una mascara ya 
predefinida. \\
