\section{Correlato con otras parcelaciones de la literatura}
\label{sec:cont_correlato}

Nuestro m\'etodo divide la corteza cerebral bas\'andose en el 
agrupamiento de tractogramas. El criterio usado es puramente estructural.
Solo usamos informaci\'on sobre la estructura f\'isica de la materia
blanca. En la literatura actual existen parcelaciones basadas en otros
criterios. Desikan et al. \cite{Desikan2006} dividen la corteza usando
como referencia las circunvoluciones del cerebro. Peinfield 
\cite{Penfield1954} parcela la corteza en base a las respuestas a 
est\'imulos el\'ectricos sobre la misma. La parcelaci\'on de Desikan es
anat\'omica, mientras que la de Peinfield es funcional. Una pregunta
interesante es si nuestra parcelaci\'on posee correlaci\'on con otras
anat\'omicas o funcionales. Estudiaremos esto contrastando el resultado
de nuestro m\'etodo contra dos parcelaciones. Una funcional basada en 
Bach et al. \cite{Barch2013} y una anat\'omica basada en Desikan
\cite{Desikan2006}.\\

Para comparar nuestro resultado contra el atlas anat\'omico 
proyectaremos sus parcelas sobre las nuestras. Luego estudiaremos cuantas
de nuestras regiones quedan bien delimitadas dentro de las proyecciones.
Esto es, que proporci\'on de las \'areas en una proyecci\'on se
encuentran efectivamente dentro de la proyecci\'on. \\

Para estudiar la relaci\'on funcional usaremos un estudio hecho por
Bach et al. \cite{Barch2013}.
En el mismo utilizan Resonancia Magn\'etica Funcional
(fMRI) para estudiar la respuesta sobre la corteza a est\'imulos
particulares. La Resonancia Magn\'etica Funcional es un tipo especial de
resonancia magn\'etica que mide el nivel de oxigeno en sangre 
\cite{Ogawa1990}. Nosotros nos enfocamos en las respuestas a los siguientes
est\'imulos: mover la mano; mover el pie; mover la lengua; reconocer formas
y caras vistas con anterioridad; clasificar un cuento breve dentro de dos
categor\'ias y resolver problemas aritm\'eticos simples. Cada uno de estas
tareas se encuentra detallada en el trabajo de Bach et al. \cite{Barch2013}.
Por cada uno de estos est\'imulos se cuenta con un mapa de 
\textit{z-scores} sobre la corteza. Estos \textit{z-scores} representan la
correlaci\'on entre la activaci\'on de cada punto de la corteza con el 
est\'imulo. Para comparar los resultados de este estudio con nuestra 
parcelaci\'on estudiaremos su superposici\'on. Definimos la siguiente 
relaci\'on: \\

\begin{equation}
\label{eq:super}
superposicion(A,B) = \frac{ 2 * area(A \cap B) }{area(A) + area(B)}
\end{equation}
 
Dadas dos superficies $A$ y $B$ se cumple: 
$0 \leq superposicion(A,B) \leq 1$. El $0$ representa que son superficies
disjuntas y el $1$ que est\'an totalmente superpuestas. Si alg\'un \'area 
posee un fuerte correlato funcional esperamos ver una superposici\'on alta
con alg\'un mapa de \textit{z-scores}.\\

Comparar las proyecciones del atlas an\'atomico con nuestras \'areas y 
la superposici\'on de estas \'ultimas con los \textit{z-scores} del 
estudio funcional nos permitir\'a hacer un primer acercamiento a 
analizar la relaci\'on entre la estructura, la anat\'omia y la funcionalidad
del cerebro.\\
