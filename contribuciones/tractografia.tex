\section{Estudio de la convergencia de los tractogramas}
\label{sec:convergencia}

Recordemos de la secci\'on \ref{sec:tractogramas} que los tractogramas se generan
mediante un procedimiento Monte Carlo. Por ende, los tractogramas son inherentemente
estoc\'asticos. Esto genera algunas preguntas interesantes: ¿Al repetir
el experimento, podremos obtener el mismo tractograma? y ¿Cu\'antas part\'iculas
son necesarias para ello?. En particular nos interesa contestar estas preguntas
para la implementaci\'on que utilizamos. A continuaci\'on presentamos el 
pseudoc\'odigo del algoritmo implementado. Para mayores detalles referirse 
al anexo. \\


\settowidth\mylen{procedure2 tracrograma(}
\addtolength\mylen{\parindent}

\begin{algorithm}
\caption{Algoritmo de tractograf\'ia utilizado}\label{alg:localtracking}
\begin{algorithmic}[1]

\Procedure{tracrograma(anatomica: img. anat\'omica, \\ \hspace*{\mylen}
                              dMRI: img. de difusi\'on, \\ \hspace*{\mylen}
                              S: semilla, P: num. part\'iculas)}{}

\State $\emph{particulas} \gets 0 $
\State \emph{csd} $\gets$ Constrained Spherical Deconvolution Model of \emph{dMRI}. 

\State \emph{mapa\_probabilistico} $\gets$ Mapa desde los coeficientes
                                           Spherical Harmonics en \emph{csd}

\State \emph{mapa\_visitas} $\gets$ matriz con mismas dimensiones que la anat\'omica
\State \emph{mapa\_visitas[:,:]} $\gets$ \emph{0} 

\While {particulas$++$ $\leq$ P}

    \State \emph{posicion\_actual} $\gets$ posicion de S
    
    \While {posicion\_actual en materia blanca}

        \State \emph{direccion\_actual} $\gets$ elegir direccion desde \emph{mapa\_probabilistico} 
        
        \State \emph{posicion\_actual} $\gets$ avanzar solo un voxel en \emph{direccion\_actual} 
        
        \State \emph{mapa\_visitas[ posicion\_actual ]} $\gets$ \emph{mapa\_visitas[ posicion\_actual ]} + 1
    
    \EndWhile
\EndWhile

\State \emph{tractograma} $\gets$ mapa\_visitas 

\For {\emph{t} in \emph{tractograma}}
    \State \emph{t} $\gets$ \emph{log(t+1)} / log(P+1)

\EndFor

\State \Return \emph{tractograma} 
 
\EndProcedure
\end{algorithmic}

\end{algorithm}

Este algoritmo es una instanciaci\'on del presentado en la secci\'on 
\ref{sec:tractogramas}. La imagen de difusi\'on se enmarca en el modelo
\textit{Constrained Spherical Deconvolution} usando la forma propuesta por 
Aganj et al. \cite{Aganj2010}; el mapa de transiciones probabil\'isticas
se recupera usando \textit{Spherical Harmonics} \cite{Descoteaux2007} y el 
tractograma se calcula usando la transformaci\'on propuesta por
Moreno-Dominguez et al. \cite{Moreno-Dominguez2014}. \\

Para determinar la estabilidad de los tractogramas generados por este algoritmo
y el n\'umero de part\'iculas necesario para que eso suceda utilizamos la t\'ecnica
estad\'istica de \textit{bootstrap} \cite{Efron1982}. Bootstrap es una forma de
aproximar la distribuci\'on del muestreo de un estad\'istico en base a calcular
el mismo utilizando sucesivos remuestreos de los datos con repeticiones. Esto es
especialmente \'util cuando el n\'umero de muestras que se posee de la poblaci\'on
no es significativamente alto. \\

En nuestro caso situamos mas de setecientas semillas en el \'Area de Broca y luego
generamos quince mil \textit{streamlines} por cada una. Luego calculamos el 
tractograma medio y la varianza de cada voxel utilizando mil submuestras aleatorias
del mismo tama\~no. Repetimos esto con varios tama\~nos de submuestra para
estudiar la variabilidad de los estad\'isticos.\\

\begin{algorithm}
\caption{Estimando estabilidad de los tractogramas}\label{alg:localtracking}
\begin{algorithmic}[1]

\Procedure{estabilidad(S: semilla)}{}

\State \emph{streamlines} $\gets$ $\emptyset$

\Loop{ 15000 veces }
    \State \emph{stream} $\gets$ crear streamline desde S 
    \State \emph{streamlines} $\gets$ agregar \emph{stream}
\EndLoop

\State \emph{medias} $\gets$ $\emptyset$
\State \emph{varianzas} $\gets$ $\emptyset$

\For{\emph{ss\_size} in [200, 500, 800, $\dots$]}

    \State \emph{subsample} $\gets$ $\emptyset$

    \Loop{ 10000 veces }
        \State \emph{tractograma} $\gets$ tomar \emph{ss\_size} streams de \emph{streamlines} y crear tractograma
        \State \emph{subsample} $\gets$ agregar \emph{tractograma} 
    \EndLoop
    
    \State \emph{medias} $\gets$ agregar tractograma medio de \emph{subsample}
    \State \emph{varianzas} $\gets$ agregar tractograma varianza de \emph{subsample}
\EndFor

\State \Return \emph{medias}, \emph{varianzas} 
 
\EndProcedure
\end{algorithmic}
\end{algorithm}



