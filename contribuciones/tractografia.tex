\section{Estudio de la convergencia de los tractogramas}
\label{sec:convergencia}

El diagrama de la figura \ref{fig:diagrama} divide el proceso de 
parcelaci\'on de la corteza en varios pasos. El tercero de ellos consiste
en generar tractogramas. Un tractograma para una semilla $s$ es una imagen
donde cada voxel $v$ representa la probabilidad de que $v$ est\'e 
conectado a $s$ mediante un conjunto de axones. En la secci\'on 
\ref{sec:tractogramas} presentamos una forma de crear tractogramas
mediante un procedimiento Monte Carlo generando \textit{streamlines}. Un
\textit{streamline} representa el movimiento aleatorio de una part\'icula
de agua dentro de la materia blanca. Los tractogramas creados de esta
manera son inherentemente estoc\'asticos. Esto genera algunas preguntas
interesantes: ¿Al repetir el experimento, podremos obtener el mismo
tractograma? y ¿Cu\'antas part\'iculas son necesarias para ello?. Para
contestar estas preguntas utilizamos la t\'ecnica estad\'istica de 
\textit{bootstrap} \cite{Efron1982}. \\

\settowidth\mylen{procedure2 tractograma(}
\addtolength\mylen{\parindent}

\begin{algorithm}[h!]
\caption{Algoritmo de tractograf\'ia utilizado}\label{alg:localtracking}
\begin{algorithmic}[1]

\Procedure{tractograma(anatomica: img. anat\'omica, \\ \hspace*{\mylen}
                              dMRI: img. de difusi\'on, \\ \hspace*{\mylen}
                              S: semilla, P: num. part\'iculas)}{}

\State \emph{csd} $\gets$ Constrained Spherical Deconvolution Model of \emph{dMRI}. 

\State \emph{mapa\_probabilistico} $\gets$ Mapa desde los coeficientes
                                           Spherical Harmonics en \emph{csd}

\State \emph{mapa\_visitas} $\gets$ matriz nula con mismas dimensiones que la anat\'omica

\For {p in [1$\dots$P]}

    \State $streamline_p$ $\gets \emptyset$
    \State \emph{posicion\_actual} $\gets$ posicion de S
    
    \While {posicion\_actual en materia blanca}

        \State \emph{direccion\_actual} $\gets$ elegir direccion desde
                                               \emph{mapa\_probabilistico} 
        
        \State \emph{posicion\_actual} $\gets$ avanzar solo un voxel en
                                               \emph{direccion\_actual} 
        
        \State $streamline_p$ $\gets$ agregar \emph{posicion\_actual}    
        
    \EndWhile
    
    \For {\emph{pos} in $streamline_p$}
        \State \emph{mapa\_visitas[pos]} $\gets$ \emph{mapa\_visitas[pos]} + 1
    \EndFor
    
\EndFor

\State \emph{tractograma} $\gets$
                                 \emph{log(mapa\_visitas+1)} / log(P+1)    

\State \Return \emph{tractograma} 
 
\EndProcedure

\end{algorithmic}
\label{alg:itract}
\end{algorithm}

\vspace{0.3cm}

\textit{Bootstrap} permite estimar la distribuci\'on de un estad\'istico
en base a calcularlo sobre remuestreos con repeticiones de una 
poblaci\'on. El m\'etodo es especialmente \'util cuando el n\'umero de
muestras que se posee no es significativamente alto. Nosotros queremos
estimar la distribuci\'on de la media de los \textit{tractogramas} en 
funci\'on de la cantidad de \textit{streamlines}. En la literatura
se usan hasta $100000$ \textit{streamlines} por cada semilla 
\cite{Anwander2006}. \textit{Bootstrap} nos permite realizar este estudio
con muchas menos. \\

Para entender mejor la relaci\'on entre \textit{streamlines} y
tractogramas presentamos en el algoritmo \ref{alg:itract} el pseudoc\'odigo
de la implementaci\'on de tractograf\'ia utilizada. Dicho algoritmo es una
instanciaci\'on del presentado en la secci\'on \ref{sec:tractogramas}. La
imagen de difusi\'on se enmarca en el modelo \textit{Constrained Spherical
Deconvolution} usando la forma propuesta por Aganj et al. 
\cite{Aganj2010}; el mapa de transiciones probabil\'isticas se recupera
usando \textit{Spherical Harmonics} \cite{Descoteaux2007} y el tractograma
se calcula usando la ecuaci\'on \ref{eq:normalizacion} propuesta en el 
trabajo de Moreno-Dominguez et al. \cite{Moreno-Dominguez2014}. \\

\begin{algorithm}
\caption{Procedimiento de \textit{Bootstraping}}\label{alg:localtracking}
\begin{algorithmic}[1]

\Procedure{estabilidad(S: semilla)}{}

\State \emph{streamlines} $\gets$ $\emptyset$

\Loop{ 15000 veces }
    \State \emph{stream} $\gets$ crear streamline desde S 
    \State \emph{streamlines} $\gets$ agregar \emph{stream}
\EndLoop

\State \emph{medias} $\gets$ $\emptyset$
\State \emph{varianzas} $\gets$ $\emptyset$

\For{\emph{ss\_size} in [200, 500, 800, $\dots$]}

    \State \emph{subsample} $\gets$ $\emptyset$

    \Loop{ 10000 veces }
        \State \emph{tractograma} $\gets$ tomar \emph{ss\_size} streams de
                                          \emph{streamlines} y crear
                                          tractograma
        \State \emph{subsample} $\gets$ agregar \emph{tractograma} 
    \EndLoop
    
    \State \emph{medias} $\gets$ agregar tractograma medio de 
                                 \emph{subsample}
    \State \emph{varianzas} $\gets$ agregar tractograma varianza de 
                                    \emph{subsample}
\EndFor

\State \Return \emph{medias}, \emph{varianzas} 
 
\EndProcedure

\end{algorithmic}
\label{alg:bootstrap}
\end{algorithm}

Usando el algoritmo \ref{alg:itract} generamos $15000$ 
\textit{streamlines} para distintas semillas en el \'area de Broca. Luego
calculamos el tractograma medio y la varianza de cada voxel utilizando
$10000$ submuestras aleatorias del mismo tama\~no. Repetimos esto con
varios tama\~nos de submuestra para estudiar la variabilidad del 
estad\'istico. Este procedimiento se puede ver en el algoritmo 
\ref{alg:bootstrap}.\\

Crear los tractogramas es uno de los pasos necesarios para parcelar la
corteza. Por la forma en que est\'an definidos, los tractogramas son
inherentemente estoc\'asticos. El m\'etodo aqu\'i presentado nos permite
estudiar la estabilidad del algoritmo de tractograf\'ia implementado. De
esta manera podemos comprobar si dado un n\'umero suficiente de 
part\'iculas los tractogramas de una semilla convergen a un mismo 
resultado. \\
